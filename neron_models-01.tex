\chapter{Motivation}

Sei $(R,\m)$ diskreter Bewertungsring, $\K=\Quot(R)$ und $E_\K$
elliptische Kurve über $\K$. Eine Weierstraßgleichung zu $E_\K$
\begin{gather*}
  0 = Y^2 Z + \c_1 XYZ + \c_3 YZ^2 - X^3 - \c_2X^2 Z - \c_4 XZ^2 - \c_6 Z^3
  \eqqcolon F\in\K[X,Y,Z]  
\end{gather*}
mit Koeffizienten in $R$ definiert ein projektives $R$-Schema
$W=V_+(F)\subset\P_R^n$ der Dimension~2. Dessen generische Faser
ist wieder $E_\K$ und die spezielle Faser ist die Kurve, die durch das
reduzierte Polynom $(F\mod\m)$ über $\k(\m)$ erzeugt wird.
$W$ enthält also Informationen zur Reduktion unserer elliptischen
Kurve.

Da elliptische Kurven besonders durch ihre Gruppenstruktur
interessant sind, stellt sich die Frage, ob sich die
Gruppenstruktur $E_\K\times_\K E_\K\to E_\K$ auf ein hinreichend
gutartiges (hier: glattes, separiertes) $R$-Schema wie $W$ erweitern
lässt.
Eine Forderung an ein solches $R$-Schema ist dann:
\begin{quote}
  \itshape
  Die Gruppenverknüpfung auf der generischen Faser erweitert sich zu
  einer Gruppenverknüpfung $E\times_R E\to E$ auf ganz $E$.
\end{quote}
Insbesondere sind dann für $R$-Schemata $Y$ alle Schnitte $E(Y)$
Gruppen. Per Konstruktion ist die kanonische Abbildung
$E(Y)\to E_\K(Y_\K)$, die durch den Basiswechsel induziert ist, ein
Gruppenmorphismus, der allerdings im Allgemeinen weder surjektiv noch
injektiv ist. Für eine \enquote{schöne} Erweiterung der
Gruppenstruktur fordern also die stärkere Eigenschaft:
\begin{quote}
  \itshape
  Für alle \enquote{gutartigen} (hier: glatten) $R$-Schemata $Y$ mit
  generischer Faser $Y_\K$ soll die kanonische Abbildung
  $E(Y)\to E_\K(Y_\K)$ ein (Gruppen-)Isomorphismus sein.
\end{quote}
Dies motiviert die Definition des Néron-Modells, welches
ein $R$-Modell von $E_\K$ ist, das ebendiese Erweiterungseigenschaften
erfüllt.

Das Hauptziel dieser Arbeit wird sein, die Existenz von Néron-Modellen
für elliptische Kurven zu zeigen und die Konstruktion zu skizzieren.
Wir beschränken uns auf Dedekindringe $R$ mit perfekten
Restklassenkörpern, für allgemeine Dedekindringe siehe auch
\cite{nonperfect}.

In \autoref{chap:gruppenstrukturenaufschemata} werden die nötigen
Grundlagen zu Gruppenschemata eingeführt, wonach dann in
\autoref{chap:neronmodelle} Néron-Modelle allgemein definiert und
einige ihrer Eigenschaften behandelt werden.
Hier wird auch ein Blick darauf geworfen, wann bzw. welcher Teil eines
Néron-Modells einer abelschen Gruppenstruktur wieder abelsch ist.

In \autoref{chap:weierstraßmodelle} wird die Konstruktion von $W$ von
oben genauer erläutert und die Eigenschaften einer solchen Kurve über
$R$ untersucht. Insbesondere werden wir feststellen, dass der glatte
Teil von $W$ tatsächlich die Gruppenverknüpfung erweitert und im
glatten (bzw. später auch regulären) Fall sogar das Néron-Modell von
$E_\K$ ist.

Ist $W$ nicht regulär, müssen erst weitere Modifikationen an $W$
vorgenommen werden, um eine hinreichend gutartige, insb. reguläre,
Kurve über $R$ zu erhalten. Solche gutartigen Kurven über
einem Dedekindring werden arithmetische Fläche genannt und in
\autoref{chap:arithmetischeflächen} näher auf ihre Eigenschaften und
die Konstruktion für elliptische Kurven untersucht.

Zuletzt wird sich in \autoref{chap:exneronmodelle} zeigen, dass der
glatte Teil einer minimal gewählten arithmetischen Fläche zu $E_\K$
bereits das Néron-Modell von $E_\K$ ist, womit die Konstruktion und
insbesondere der Existenzbeweis abgeschlossen ist.

\autoref{chap:ausblick} enthält dann noch ein paar Beispiele
bzw. Anwendungen und den Ausblick auf eine Klassifizierung
elliptischer Kurven nach der Form ihres minimalen Modells.

\section*{Notation}
Es werden Kenntnisse zur Garben-~und Schematheorie benötigt,
insbesondere zu den Begriffen glatt, flach, eigentlich, separiert.
Die Notation richtet sich nach \cite{wedhorn}.

% Mit affiner Varietät ist stets ein affines, integeres $\K$-Schema
% endlichen Typs gemeint.
Eine Kurve über einem Körper $\K$ sei immer eine projektive Varietät
der Dimension~1,
und eine elliptische Kurve über $\K$ ist eine glatte Kurve $E_\K$
vom Genus~1 zusammen mit einem $\K$-Schnitt $O\in E_\K(\K)$.
Elliptische Kurven sind $\K$"=Gruppenschemata mit Nullelement $O$
\cite{silverman}.

Es werden nur elliptische Kurven über Körpern der Charakteristik
ungleich~2, 3 betrachten und die Restklassenkörper der Dedekindringe
seien perfekt.

Im Folgenden noch einige weitere verwendete Begriffe:

% DEDEKINDRINGE
\begin{Definition}[Dedekindring]\label{def:dedekindring}
  \optcite[S.\,40]{hartshorne}\optcite[Definition B.84]{wedhorn}
  Ein Integritätsring $R$ heißt Dedekindring, wenn er eine der
  äquivalenten Bedingungen erfüllt
  \begin{enumerate}[label=(\roman*)]
  \item $R$ ist regulär und $\dim(R)\leq 1$.
  \item $R$ ist noethersch, normal und $\dim(R)\leq 1$.
  \item $R$ ist Körper oder
    jede Lokalisierung nach einem Maximalideal ist ein diskreter
    Bewertungsring
    (d.\,h. lokal, noethersch, normal von Dimension~1).
  \end{enumerate}
\end{Definition}
Für Schemata über Dedekindringen werden die Fasern über dem
generischen Punkt als generische Fasern, die über den geschlossenen
Punkten als spezielle Fasern bezeichnet.
\begin{Definition}[Dedekindschema]
  Ein Dedekindschema ist ein noethersches, integeres Schema $X$, das
  eine der äquivalenten Eigenschaften erfüllt
  \begin{enumerate}[label=(\roman*)]
  \item Die offenen affinen Unterschemata sind Spektren von Dedekindringen.
  \item $\dim(X)\leq 1$ und $X$ ist regulär
    (in Dimension 1 äquivalent zu normal nach
    \cite[Corollary 6.39, Proposition 6.40]{wedhorn}).
  \end{enumerate}
\end{Definition}
Sei $S$ allgemein ein irreduzibles Schema. Ein $S$-Schema wird auch als
$S$-Modell seiner generischen Faser bezeichnet.

Folgende Äquivalenz und Aussage zu Glattheit werden häufig verwendet:
\begin{Bemerkung}\label{thm:eigglatt}
  Sei $f\colon X\to Y$ Morphismus von Schemata.
  Für $f$ endlich präsentiert ist $f\colon X\to Y $ glatt in $x$
  äquivalent zu $f$ flach in $x$ und $X_{f(x)}$ glatt über $\k(f(x))$ nach
  \cite[8.5, Proposition 17]{bosch}.
  
  Ist $X$ ein $\K$-Schema lokal endlichen Typs sind glatt und
  geometrisch regulär äquivalent nach \cite[Corollary 6.32]{wedhorn}.
  Ist $\K$ perfekter Körper (z.\,B. $\Char(\K)=0$), so ist dies sogar
  äquivalent zu regulär nach \cite[Remark 6.33]{wedhorn}.
    
  Glatt impliziert reduziert, lokal von endlichem Typ und flach
  \cite[für flach s.][Theorem 14.22]{wedhorn}.
\end{Bemerkung}

%%% Local Variables:
%%% mode: latex
%%% TeX-master: "neron_models"
%%% End:
