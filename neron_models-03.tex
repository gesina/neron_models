\chapter{Néron-Modelle}\label{chap:neronmodelle}
Sei $R$ im gesamten Kapitel ein Dedekindring mit Quotientenkörper
$\K$.

Nachdem jetzt die nötigen Grundlagen für Gruppenschemata vorhanden
sind, lassen sich Néron-Modelle wie zuvor motiviert definieren.
Dazu wird erst die formale Definition angegeben und für
Gruppenschemata spezialisiert. Danach werden wichtige allgemeine
Eigenschaften untersucht, die direkt aus der universellen Eigenschaft
folgen, unter anderem die Eindeutigkeit und Lokalität.
Zuletzt wird noch einen kurzen Blick auf Néron-Modelle
allgemeiner abelscher Varietäten geworfen.

\section{Definition}
Das Néron-Modell eines $\K$-Gruppenschemas ist – falls es existiert –
eine besonders gutartige (genauer: glatte, separierte) Erweiterung der
Gruppenstruktur auf ein $R$-Gruppenschema, das
minimal mit dieser Eigenschaft gewählt ist.

In diesem Abschnitt wird vorerst die allgemeine Definition eines
Néron-Modells nach \cite[Definition~1.2/1]{neron} anhand der
Erweiterungsforderung eingeführt. Daraus leiten sich bereits einige
Eigenschaften ab, die zu einer alternativen Definition für
$\K$-Gruppenschemata führen, welche die Gruppenstruktur wieder mit
einbezieht.

\begin{Definition}[Néron-Modell]
  \optcite[1.2, Definition 1]{neron}
  Sei $R$ Dedekindring, $\K=\Quot(R)$ und
  $X_\K$ ein glattes, separiertes $\K$-Schema endlichen Typs.
  Ein Néron-Modell von $X_\K$ über $R$ ist ein glattes, separiertes
  $R$-Modell $X$ von $X_\K$ mit der \NAbbEig
  \begin{quote}
    Für jedes weitere glatte $R$-Schema $Y$ wird jeder
    $\K$-Morphismus $Y_\K\to X_\K$ auf den generischen Fasern durch
    einen eindeutigen $R$-Morphismus $Y\to X$ erweitert.
  \end{quote}
  In anderen Worten, die kanonische Einschränkung $X(Y)\to
  X_\K(Y_\K)$ ist ein Isomorphismus.
\end{Definition}

% ERWEITERUNG DER GRUPPENSCHEMASTRUKTUR
Die Motivation als minimale Erweiterung von Gruppenschemata ist wie
folgt gerechtfertigt:
\begin{Lemma}\label{thm:gruppenschemaerweiterung}
  \optcite[1.2, Proposition 6]{neron}
  Sei $X$ Néron-Modell seiner generischen Faser $X_\K$ über $R$. Ist
  $X_\K$ ein $\K$"=Gruppenschema, so wird dessen $\K$"=Gruppenstruktur
  durch die \NAbbEig eindeutig auf eine $S$"=Gruppenschemastruktur auf
  $X$ erweitert.
  \begin{proof}[Beweisskizze]
    Die Gruppenverknüpfung $\mu_\K\colon X_\K\times_\K X_\K\to X_\K$
    erweitert sich wegen der \NAbbEig zu einem Morphismus
    $\mu\colon X\times_R X\to X$, da $X\times_R X$ wieder glatt ist.
    Genauso liften das Inverse, der Einheitsschnitt und, wegen der
    Eindeutigkeit, die Diagramme aus \ref{def:gruppenschema}, was
    $\mu$ zur Gruppenverknüpfung auf $X$ macht.
  \end{proof}
\end{Lemma}

% NÉRON-MODELL K-GRUPPENSCHEMA
\begin{Lemma}\label{nerongruppenschemaglatt}
  \optcite[1.2, Remark 7]{neron} 
  Ist $X_\K$ ein $\K$"=Gruppenschema und $X$ ein
  glattes $R$-Modell und Gruppenschema von $X_\K$, das die
  \NAbbEig erfüllt, so ist $X$ ein Néron-Modell von $X_\K$.
  \begin{proof}
    $X_\K$ ist als Faser eines glatten Schemas glatt. Bleibt zu
    zeigen, dass $X_\K$ und $X$ separiert sind.
    
    Ein Gruppenschema $X$ ist genau dann separiert, wenn der
    Einheitsschnitt ${\epsilon\colon\Spec(R)\to X}$ eine abgeschlossene
    Immersion ist nach
    \cite[Lemma~38.6.1]{stacksproject} oder \cite[Lemma~7.1/2]{neron}.
    $\K$"=Gruppenschemata sind entsprechend immer separiert
    \cite[Lemma~38.7.3]{stacksproject}. 

    Sei wegen \ref{thm:neronmodelllokal} ohne Einschränkung $R$ diskreter
    Bewertungsring mit Restklassenkörper $k$.    
    Es wird im Folgenden gezeigt, dass topologisch
    $\im(\epsilon)=\overline{\im(\epsilon_\K)}$,
    also dass $\overline{\im(\epsilon_\K)}$ aus genau dem Punkt
    $\epsilon(\Spec\K)$ in der generischen Faser und dem Punkt
    $\epsilon(\Spec k)$ in der speziellen Faser besteht.
    Dann ist $\im(\epsilon)$ topologisch abgeschlossen in $X$ und der
    Schnitt $\epsilon$ demnach abgeschlossene Immersion, da er
    wegen der Glattheit von $X$ bereits Homeomorphismus und als
    Schnitt surjektiv auf den Halmen ist.
    
    Auf der generischen Faser gilt
    $\overline{\im(\epsilon_\K)}\cap X_\K=\im(\epsilon_\K)$,
    da $\epsilon_\K$ abgeschlossene Immersion ist.
    Außerdem stimmen hier $\im(\epsilon_\K)=\{\epsilon_\K(\Spec\K)\}$
    und $\im(\epsilon)\cap X_\K$ wegen der \NAbbEig punktweise
    überein.

    Auf der speziellen Faser $X_k$ ist zu zeigen, dass für jeden
    Punkt $x\in\overline{\im(\epsilon_\K)}\cap X_k$ bereits
    $x\in\im(\epsilon)$ gilt. Sei also $x$ ein solcher Punkt.
    Das schematische Bild $\text{Im}(\epsilon_\K)$ von $\epsilon_\K$ ist
    nach \cite[Remark~10.32]{wedhorn} genau
    $\overline{\im(\epsilon_\K)}=\overline{\{\epsilon_\K(\Spec(\K))\}}$
    mit reduzierter Unterschemastruktur, was
    $\overline{\im(\epsilon_\K)}$ mit einer integren
    Unterschemastruktur ausstattet.
    Sei $U$ eine offene, affine Umgebung von $x$ in $X$ und
    $\Spec(A)=U\cap\overline{\im(\epsilon_\K)}$ offenes, affines
    Unterschema von $\overline{\im(\epsilon_\K)}$.
    Später werden die Inklusionen $R\subset A\subset\K$ gezeigt.
    Nachdem $R$ als diskreter Bewertungsring der größte Unterring
    von $\K$ ist, folgt daraus dann $R=A$ und es ergibt sich zusammen
    mit der Immersion zu $\Spec(A)$ ein Schnitt
    \begin{gather*}
      X(R)\ni e\colon
      \Spec(R)\xrightarrow{\sim}\Spec(A)\hookrightarrow X\;,
    \end{gather*}
    der $\epsilon_\K\colon\Spec(\K)\to X_\K$ erweitert.
    Dieser muss allerdings wegen der Eindeutigkeit aus der \NAbbEig
    gleich $\epsilon\in X(R)$ sein, also liegt $x$ in
    $\im(e)=\im(\epsilon)$.
    Nachdem $x\in \overline{\im(\epsilon_\K)}\cap X_k$ beliebig
    gewählt war, kann $\overline{\im(\epsilon_\K)}\cap X_k$ nur
    den Punkt $\epsilon(\Spec k)$ enthalten.

    Betrachte also $A$ wie oben, um $R\subset A\subset\K$ zu
    zeigen. Es ist $e_\K\coloneqq\epsilon_\K(\Spec(\K))$ der
    generische Punkt von $\overline{\im(\epsilon_\K)}$ und
    entsprechend auch vom offenen Unterschema $\Spec(A)$. Da
    $\overline{\im(\epsilon_\K)}$ und $\Spec(A)$ integer sind, ist
    $\O_{\overline{\Im(\epsilon_\K)}, e_\K}=\O_{A,e_\K}$ Körper.
    Es folgt eine Sequenz von Körperhomomorphismen auf den Halmen
    \begin{gather*}
      \Id_\K\colon
      \K \xrightarrow{\pi_\K^\#}
      \O_{\Im(\epsilon_\K), e_\K} = \O_{A,e_\K} = \Quot(A)
      \xrightarrow{\epsilon_\K^\#} \K
    \end{gather*}
    weshalb $\Quot(A)\cong\K$ und daher $A\hookrightarrow\K$.
    Für den zu $\pi|_{\Spec(A)}$ assoziierten
    Ringhomomorphismus $\pi^\#=\pi_\K^\#|_R$ kommutiert außerdem wegen
    $\pi\circ\epsilon_\K=\Id_{\Spec(\K)}$ das folgende Diagramm
    \begin{center}
      \begin{tikzcd}
        A\arrow[r, hook]\arrow[from=d, "{\pi^\#}"]
        & \Quot(A)\arrow[d, "{\wr}"{swap}, "{\epsilon_\K^\#}"]\\
        R\arrow[r, hook]
        & \K
      \end{tikzcd}
    \end{center}
    weshalb $\pi^\#\colon R\hookrightarrow A$ gelten muss.
    
    Die Mengen $\overline{\im(\epsilon_\K)}=\im(\epsilon_\K)$ stimmen
    also überein, insbesondere ist $\im(\epsilon_\K)$ abgeschlossene
    Immersion.
  \end{proof}
\end{Lemma}

% Elliptische Kurven sind abelsche Varietäten und nach
% \ref{thm:exneronmodellabvarietaet} hat somit jede elliptische
% Kurve ein Néron-Modell. Wir werden uns damit befassen, wie sich diese
% Modelle konkret angeben lassen mithilfe der Weierstraßgleichung der
% Kurve.

\begin{SatzDefinition}[Néron-Modelle elliptischer Kurven]
  \label{def:neronmodellellkurve}
  Sei $E_\K$ eine elliptische Kurve, das heißt ein
  glattes, projektives, insbesondere separiertes, integres
  $\K$-Schema endlichen Typs vom Geschlecht~1.
  Dann ist ein Néron-Modell von $E_\K$ nach
  \ref{thm:gruppenschemaerweiterung} und \ref{nerongruppenschemaglatt}
  ein glattes $R$-Modell und Gruppenschema $E/R$ mit der
  \NAbbEig:
  \begin{quote}
    Für jedes weitere glatte $R$-Schema $Y$ wird jeder
    $\K$-Morphismus ${\phi_\K\colon Y_\K\to E_\K}$ auf den generischen
    Fasern durch einen eindeutigen $R$-Morphismus ${Y\to E}$ erweitert.
  \end{quote}
  \optcite[Chapter IV.5]{silverman2}
\end{SatzDefinition}
\begin{Bemerkung}\label{thm:dichtefaser}
  Das $\phi_\K$ aus \ref{def:neronmodellellkurve} kann auch als
  $\K$-rationale Abbildung gewählt werden. Da $E_\K$ projektiv, also
  eigentlich, ist, erweitert sich $\phi_\K$ nach
  \ref{thm:rationalzumorphismus} zu einem Morphismus.

  Im Fall eines diskreten Bewertungsrings $R$ ist der generische Punkt
  $(0)$ von $\Spec(R)$ offen, da das Komplement
  $\Spec(R)\setminus\{(0)\}$ aus nur endlich vielen abgeschlossenen
  Punkten besteht. Entsprechend ist die generische Faser $Y_\K$ jedes
  $R$-Schemas $Y$ offen.
  Ist $Y$ glatt – also insbesondere flach und reduziert – über $R$,
  ist $Y_\K$ dicht in $Y$ nach \cite[Proposition~III.9.7]{hartshorne}.
  Also präsentiert der Morphismus ${\phi_\K\colon X_\K\to E_\K}$ eine
  rationale Abbildung (vgl. \ref{def:ratabb})
  \begin{gather*}
    \phi\colon X\dra E
  \end{gather*}
  Die \NAbbEig bedeutet, dass diese sich (eindeutig) zu einem
  Morphismus $X\to E$ erweitert.
\end{Bemerkung}


\section{Eigenschaften}
\subsection{Allgemeine Eigenschaften}
Wie oben sei $R$ wieder Dedekindring mit $\K=\Quot(R)$.

% WARUM SEPARIERTHEIT
Ein Grund, weshalb für ein Néron-Modell Separiertheit gefordert wird,
ist die Eindeutigkeit von Morphismen für separierte Schemata im Sinne
des folgenden Satzes. Für den Beweis siehe
\cite[Corollary~9.9]{wedhorn}.
\begin{Satz}\label{thm:erweindeutig}
  Seien $Y$ und $X$ $R$-Schemata, $X$ separiert, $Y$
  reduziert.
  Dann ist ein Morphismus $\phi\colon Y\to X$ durch jede
  Einschränkung auf eine dichte Teilmenge von $Y$ eindeutig bestimmt.
  Insbesondere sind, falls $Y$ flach über $X$ ist, Erweiterungen von
  Morphismen der generischen Fasern eindeutig.
  % \begin{proof}
  %   Sei $x\in\Dom(\phi)$.
  %   Wir werden induktiv nach der Dimension von $x$ vorgehen. Für
  %   $\dim_Y(x)=0$ ist $x$ generischer Punkt und $\phi$ ist hier nach
  %   Voraussetzung festgelegt.
  %   Sei $\phi$ eindeutig definiert in $y\in\Dom(\phi)$ für
  %   $\dim_Y(y)=(n-1)$ und sei $\dim_Y(x)=n$.
  %   Wir können nun eine Generalisierung $y\in\Spec(\O_{Y,x})\subset Y$
  %   von $x$ mit $\dim_Y(y)=(n-1)$ betrachten.
  %   Im geschlossenen, irreduziblen, reduzierten Unterschema $V_y$ mit
  %   topologischem Raum $\overline{\{y\}}\subset Y$ ist
  %   $\dim_{V_y}(x)=1$.
  %   $V_y$ ist integer, $\Spec(\O_{V_y,x})=\{x,y\}$ und daher 
  %   $\O_{V_y,x}$ ein Bewertungsring mit
  %   $y=(0)\in\Spec(\O_{V_y,x})$ und $\Quot(\O_{V_y,x})=\k(y)$.
  %   Nach dem Bewertungskriterium für Separiertheit von $W$ ist
  %   $\phi|_{\Spec(\O_{V_y,x})}$ die eindeutige Erweiterung von
  %   $\phi|_{\Spec(\k(y))}$. Da $\phi$ auf $y$ bereits eindeutig war
  %   und $Y$ reduziert ist, ist $\phi$ auch eindeutig auf $x$.

  %   Ist $Y$ flach, so liegen alle generischen Punkte von $Y$ in der
  %   generischen Faser nach \cite[Proposition III.9.7]{hartshorne}.
  % \end{proof}
\end{Satz}
Somit ist die Eindeutigkeit der Erweiterung eines Schnitts bereits
durch die Separiertheit des Néron-Modells gegeben. Außerdem ist nun
bekannt, dass die kanonische Einschränkung $X(Y)\to X_\K(Y_\K)$ der
$R$-Schnitte eines Néron-Modells $X$ nicht nur für glatte, sondern
auch für reduzierte Schemata $Y$ injektiv ist.

% EINDEUTIGKEIT, BASISWECHSEL
Nun ein paar weitere grundlegende Eigenschaften von Néron-Modellen.
Hierbei ist zu beachten, dass sich Néron-Modelle nicht so gutartig
verhalten, wie es die universelle Eigenschaft vermuten lässt.
Unter anderem sind sie zwar stabil unter Produktbildung (was hier
nicht gebrauchen und nicht gezeigen wird), allerdings nicht unter
beliebigem Basiswechsel.
\begin{Satz}\label{thm:eigneronmodell}
  \optcite[1.2, Proposition 2]{neron}
  \optcite[Proposition IV.5.2]{silverman2}
  Sei $X$ ein Néron-Modell seiner generischen Faser $X_\K$ über
  $R$. Dann gilt
  \begin{enumerate}[label=(\alph*)]
  \item $X$ ist eindeutig bis auf kanonische Isomorphie.
  \item Für einen étalen Basiswechsel $R\to R'$ mit $K'$
    Funktionenkörper von $R'$ ist $X_{R'}=X\times_R R'$ Néron-Modell
    seiner generischen Faser $X_{\K'}=X_\K\times_{\K}\K'$ über $R'$,
    das heißt étaler Basiswechsel kommutiert mit der Bildung von
    Néron-Modellen.
  \end{enumerate}
  \begin{proof}
    Die erste Eignschaft folgt direkt aus der universellen
    Eigenschaft, für die zweite sind weitere Argumente zur Stabilität
    unter Basiswechsel nötig. 
    \begin{enumerate}[resume*,start=1]
    \item Sei $X'$ ein weiteres Néron-Modell von $X_\K$.
      Dann liefert die \NAbbEig zur Identität
      ${\Id_{X_\K}\colon X_\K\to X_\K}$ eindeutige Morphismen
      ${\phi\colon X'\to X}$ und ${\psi\colon X\to X'}$.
      Die Verknüpfungen ${\phi\circ\psi\colon X\to X}$ und
      ${\psi\circ\phi\colon X'\to X'}$ sind jeweils die Identität auf
      der generischen Faser $X_\K$, welche nach der Eindeutigkeit in der
      \NAbbEig eindeutig durch die Identitäten auf
      $X$ und auf $X'$ erweitert wird. Also sind $\phi,\psi$ inverse
      Isomorphismen.
    \item
      $X_{R'}$ ist glatt, separiert und von endlichem Typ, das heißt
      lokal von endlichem Typ und quasikompakt, da alles 
      stabil unter Basiswechsel ist. Es ist somit nur die
      \NAbbEig zu prüfen.
      Sei also $Y'$ glattes $R'$-Schema mit generischer Faser
      $Y_{\K'}'$ und ${\phi_{\K'}\colon Y_{\K'}'\to X_{\K'}}$ ein
      $\K'$-Morphismus. 
      \begin{description}
      \item[Existenz:] 
        $Y'$ ist glattes $R$-Schema durch die Komposition
        \begin{gather*}
          Y'\xrightarrow{\text{glatt}}\Spec(R')
          \xrightarrow{\text{glatt}}\Spec(R)
        \end{gather*}
        glatter Abbildungen.
        Außerdem lässt sich der $\K$-Morphismus
        \begin{gather*}
          \phi_{\K} \coloneqq \proj_{X_\K}\circ\phi_{\K'}
          \colon
          Y_{\K'}' \xrightarrow{\phi_{\K'}} X_{\K'}=X\times_\K\K'
          \xrightarrow{\proj_{X_\K}} X_{\K}
        \end{gather*}
        mit der \NAbbEig von $X$ zu einem eindeutigen
        $R$-Morphismus
        \begin{gather*}
          \phi_R\colon Y'\longto X
        \end{gather*}
        erweitern.
        So ergibt sich ein kanonischer eindeutiger $R$-Morphismus
        (beziehungsweise ein $R'$-Morphismus)
        $\phi_{R'} = (\phi_R\;,\;\pi_{R'})$ auf dem Faserprodukt 
        \begin{center}
          \begin{tikzcd}
            Y'
            \arrow[drr, bend left, "\pi_{R'}"]
            \arrow[ddr, bend right, "{\phi_R}"]
            \arrow[dr, dashed, "{\exists!\phi_{R'}}"]\\
            & X\times_{R} R'
            \arrow[dr, phantom, "\ulcorner", very near start]
            \arrow[r, "\proj_{R'}"]\arrow[d, "\proj_X"]
            & \Spec(R')\arrow[d]\\
            & X \arrow[r]
            & \Spec(R)
          \end{tikzcd}
        \end{center}
        $\phi_{R'}$ erweitert $\phi_{\K'}$, denn
        \begin{align*}
          \phi_{R'}|_{Y_{\K'}'}
          &= \left(
            \phi_R|_{Y_{\K'}'}\;,\;
            \pi_{R'}|_{Y_{\K'}'}\colon
            Y_{\K'}'\hookrightarrow Y'\to\Spec(R')
            \right)\\
            % \pi|_{Y_{\K'}'})\colon Y'\times_{R'}\K'\to Y'\to\Spec(R')
            % \overset{\text{Faserprod.}}{=}
            % \pi_{\K'}\colon Y'\times_{R'}\K'\to \Spec(K')\to\Spec(R')
          &\overset{\mathllap{\text{Faserprod.}}}{=} \left(
            \phi_R|_{Y_{\K'}'}\;,\;
            \pi_{\K'}\colon
            Y_{\K'}'\to \Spec(K')\hookrightarrow\Spec(R')
            \right)\\
          &\overset{\mathllap{\text{Def. }\phi_R}}{=} \left(
            \phi_{\K}\;,\; \pi_{\K'}
            \right)\\
          &\overset{\mathllap{\text{Def. }\phi_\K}}{=} \left(
            \proj_{X_\K}\circ\phi_{\K'}\;,\;
            \proj_{\K'}\circ\phi_{\K'}
            \right)\\
          &= \phi_{\K'}
        \end{align*}
      \item[Eindeutigkeit:]
        Für jeden weiteren Lift $\psi_{R'}$ von $\phi_{\K'}$ ist
        $\proj_{X}\circ\psi_{R'}$ ein Lift von $\phi_{K}$ und es gilt
        aufgrund der Eindeutigkeit in der \NAbbEig
        ${\proj_{X}\circ\psi_{R'}=\phi_{R}}$.
        Damit lässt $\psi$ ebenfalls das obige Diagramm kommutieren
        und ist nach der universellen Eigenschaft des Faserprodukts
        bereits gleich $\phi_{R'}$.
        \qedhere
      \end{description}
    \end{enumerate}
  \end{proof}
\end{Satz}

% LOKALITÄT
Eine sehr wichtige und praktische Eigenschaft ist die Lokalität von
Néron-Modellen wie folgt:
\begin{Satz}\label{thm:neronmodelllokal}
  Für ein $R$-Schema $X$ endlichen Typs ist äquivalent
  \begin{enumerate}[label=(\roman*)]
  \item $X$ ist Néron-Modell seiner generischen Faser
  \item $X\times_R\Spec(\O_{\Spec(R),s})$ ist Néron-Modell
    seiner generischen Faser über $\O_{R,s}=R_s$ für alle
    abgeschlossenen Punkte $s\in\Spec(R)$.
  \end{enumerate}
  Das heißt, es kann ohne Einschränkung immer davon ausgegangen
  werden, dass $R$ lokal ist, also diskreter Bewertungsring.
\end{Satz}

Im Folgenden wird nur die Implikation
\emph{(ii)}$\Rightarrow$\emph{(i)} benötigt, daher sei für die
andere Richtung auf \cite[Proposition~1.2/4]{neron} verwiesen.
Erst ist dazu ein Lemma und ein Korollar zur Erweiterung von
Morphismen auf offene Umgebungen nötig.

% ERWEITERUNG V. MORPHISMEN AUF OFFENE TEILMENGEN
\begin{Lemma}\label{thm:morphismuserweiterung}
  % http://math.stackexchange.com/questions/1259876/extending-a-morphism-of-schemes
  \optcite[Exercise 3.2.4]{liu}
  Sei $S$ ein lokal noethersches Schema,
  $X$ und $Y$ $S$-Schemata,
  $X$ endlichen Typs.
  Seien weiterhin $y\in Y$ und 
  ein $S$-Morphismus ${f_y\colon \Spec\O_{Y,y}\to X}$ gegeben.
  Dann gibt es eine offene Umgebung ${U\subset Y}$ von $y$, auf der
  $f_y$ von einem Morphismus ${f\colon U\to X}$ erweitert wird.
  \begin{proof}
    Es reicht, sich auf den affinen Fall $X=\Spec(B)$ zu beschränken.
    Denn $\O_{Y,y}$ ist lokal mit Maximalideal $y$, also ist $y$ in
    jeder Zariski-abgeschlossenen, nichtleeren Menge von $\Spec\O_{Y,y}$
    enthalten. Anders gesagt die einzige offene Umgebung von $y$ in
    $\Spec\O_{Y,y}$ ist $\Spec\O_{Y,y}$ selbst.
    Somit ist das offene Urbild jeder offenen, affinen Umgebung
    $\Spec(B)\subset X$ von $f_y(y)$ ganz $\Spec\O_{Y,y}$.

    $X$ ist nach Voraussetzung von endlichem Typ und $S$ lokal
    noethersch.
    Es kann also eine genügend kleine, offene, affine Umgebung
    $\Spec(B)\subset X$ von $f_y(y)$ und eine offene Teilmenge
    $\Spec(R)\subset S$ ausgewählt werden,
    so dass $R$ noethersch und $B$ eine $R$-Algebra endlichen Typs
    – wegen noethersch auch endlich präsentiert – ist. Das heißt $B$
    hat die Form
    \begin{gather*}
      B=R[X_0,\dotsc, X_n]/(g_0,\dotsc, g_r)\qquad n,r\in\N
    \end{gather*}
    Wähle weiterhin eine genügend kleine, offene, affine Umgebung
    $\Spec(A)\subset Y$ von $y$ aus, dass $A$ ebenfalls $R$-Algebra ist.
    Hier gilt $y\in\Spec\O_{Y,y}=\Spec(A_y)\subset\Spec(A)$.
    
    Der Morphismus ${f_y\colon\Spec(A_y)\to\Spec(B)\subset X}$
    korrespondiert also zu einem Ringhomomorphismus
    ${\Gamma(f_y)\colon B\to A_y}$.
    Insgesamt ergibt sich
    \begin{center}
      \begin{tikzcd}[row sep=0pt]
        \phi\colon R\left[X_0,\dotsc, X_n\right] \arrow[r,two heads]
        &\frac{R\left[X_0,\dotsc, X_n\right]}{\left(g_0,\dotsc, g_r\right)}
        \arrow[r,"{\Gamma(f_y)}"]
        &A_y\\
        X_i \arrow[rr,mapsto]&& \frac{\alpha_i}{\beta_i} &i=0,\dotsc, n\\
        g_j \arrow[rr,mapsto]&& \frac{\alpha_j'}{\beta_j'} &j=0,\dotsc, r
      \end{tikzcd}
    \end{center}
    Die Bedingung $\frac{\alpha_j'}{\beta_j'}=\Gamma(f_y)(g_j)=0\in A_y$
    heißt, dass es $\gamma_j$ gibt, so dass
    $\alpha_j'\gamma_j=0\in A$.
    Definiere
    $\gamma\coloneqq \prod_{i=0}^n \beta_i \prod_{j=0}^r \gamma_j$.
    Dann gilt für den Morphismus
    \begin{align*}
      \phi_\gamma\colon
      R\left[X_0,\dotsc, X_n\right]
      &\longto
        A_\gamma\\
      X_i
      &\longmapsto
        \dfrac{
        \alpha_i\cdot
        \prod_{k\neq i}^n \beta_k
        \prod_{j=0}^r \gamma_j
        }{f}
        \qquad i=0,\dotsc,n
    \end{align*}
    dass $(g_0, \dotsc, g_r)\subset\ker(\phi_f)$ und das folgende
    Diagramm kommutiert
    \begin{center}
      \begin{tikzcd}
        R\left[X_0,\dotsc, X_n\right] \arrow[r,"\phi_\gamma"]
        \arrow[d, two heads]
        & A_\gamma\arrow[d]\\
        B \arrow[r, "\Gamma(f_y)"{swap}]\arrow[ur,"\overline\phi_\gamma"]
        & A_y
      \end{tikzcd}
    \end{center}
    Insbesondere faktorisiert ${\Gamma(f_y)\colon B\to A_\gamma\to A_y}$
    über $A_\gamma$, was heißt, dass $f_y$ über einen Morphismus
    ${f\colon D(\gamma)\to\Spec(B)}$ faktorisiert:
    \begin{gather*}
      f_y\colon
      \Spec\O_{Y,y}=\Spec(A_y)
      \hookrightarrow \Spec(A_\gamma)=D(\gamma)
      \overset{f}\longto \Spec(B)
    \end{gather*}
    $D(\gamma)$ ist eine offene Umgebung von $y$, womit die Existenz
    gezeigt ist.
  \end{proof}
\end{Lemma}

\begin{Korollar}\label{thm:allgmorphismuserweiterung}
  Sei $S$ Dedekindschema,
  $\pi\colon Y\to S$ integres, flaches $S$-Schema
  und $X$ separiertes $S$-Schema endlichen Typs.
  Sei für ein $s\in S$ ein $S$-Morphismus
  \begin{gather*}
    f_s\colon Y\times_S\O_{S,s}\to X\times_S\O_{S,s}
  \end{gather*}
  gegeben.
  Dann gibt es eine offene Umgebung $U\subset Y$ von
  ${Y\times_S\O_{S,s}}$, auf der $f_s$ eindeutig von einem Morphismus
  ${f\colon U\to X}$ erweitert wird.
  Insbesondere gilt dies für $Y$ glatt, da $Y$ dann disjunkte
  Vereinigung integrer, flacher $S$-Schemata ist.
  \begin{proof}
    Sei $y\in Y\times_S\O_{S,s}\subset Y$ und
    $s'\coloneqq\pi(y)\in\Spec\O_{S,s}$ die zugehörige Spezialisierung
    von $s$. Dann ist $y\in Y\times_S\O_{S,s'}$ und $\pi$ induziert einen
    Ringhomomorphismus ${\O_{S,s'}\to\O_{Y,y}}$ auf den Halmen. Der
    zugehörige Morphismus
    ${\Spec\O_{Y,y}\to\Spec\O_{S,s'}\subset\Spec\O_{S,s}}$
    von affinen $R$-Schemata ist die Einschränkung von $\pi$ auf
    ${\Spec\O_{Y,y}\subset Y}$, insbesondere  liegt ganz $\Spec\O_{Y,y}$
    im Urbild von $\Spec\O_{S,s}$.
    Das zeigt, dass ${\Spec\O_{Y,y}\subset Y\times_S\O_{S,s}}$ für jedes
    ${y\in Y\times_S\O_{S,s}}$, und die Einschränkung von $f_s$ liefert
    $R$-Morphismen
    \begin{gather*}
      f_{s,y}\colon \Spec\O_{Y,y}\longto X\times_S\O_{S,s}\subset X
      \qquad \text{für alle } y \in Y\times_S\O_{S,s}
    \end{gather*}
    Jeder davon erweitert sich nach \ref{thm:morphismuserweiterung}
    zu einem Morphismus auf einer offenen Umgebung ${U_y\subset Y}$ von
    $y$.
    Diese Erweiterungen sind eindeutig wegen Separiertheit.
    Die Definitionsmengen $U_y$ sowie ihre Schnitte ${U_y\cap
    U_{y'}\subset Y}$, ${y,y'\in Y}$, sind offene, integre
    Unterschemata mit demselben generischen Punkt $\eta$ wie $Y$, der
    entsprechend dicht in allen $U_y$ liegt.
    Da $Y$ nach Voraussetzung flach ist, liegt $\eta$ außerdem in
    ${Y\times_S\O_{S,s}}$ nach \cite[Proposition~14.14]{wedhorn}, sprich
    im Definitionsbereich von $f_s$. Mit $X$ separiert sind deswegen nach
    \ref{thm:erweindeutig} alle Erweiterungen von $f_s|_{U_y}$ auf
    ganz $U_y$ und von $f_s|_{U_y\cap U_{y'}}$ auf ${U_y\cap U_{y'}}$
    eindeutig durch $f_s|_{\{\eta\}}$ bestimmt.
    Insbesondere sind die Erweiterungen $f_{s,y}$ eindeutig und
    stimmen auf den Schnitten ihrer Definitionsmengen überein.
    Daher verkleben sie zu einem eindeutigen Morphismus
    \begin{gather*}
      f\colon U=\bigcup_{\mathclap{y\in Y\times_S\O_{S,s}}} U_y\longto X
    \end{gather*}
    der $f_s$ erweitert.
  \end{proof}
\end{Korollar}
% \begin{Lemma}\label{thm:limesoffenemengen}
%   \cite[1.2, Lemma 5]{neron}
%   Für ein Basisschema $S$, einen Punkt $s\in S$ und endlich
%   präsentierte $S$-Schemata $X$, $Y$ ist die kanonische
%   Abbildung
%   \begin{gather*}
%     \varinjlim \Hom_{S'}(X\times_S S', Y\times_S S')
%     \longto
%     \Hom_{\O_{R,s}}(X\times_S \O_{R,s}, Y\times_S \O_{R,s})
%   \end{gather*}
%   bijektiv, wobei der direkte Limes über die offenen Umgebungen
%   $S'\subset S$ von $s$ läuft.
% \end{Lemma}

\begin{proof}[Beweis von \ref{thm:neronmodelllokal}]
  Vorerst ist zu zeigen, dass $X$ glatt und separiert ist.

  Letzteres gilt, da das Bewertungskriterium für Separiertheit
  erfüllt ist: Sei $(A,\m)$ Bewertungsring und über
  ${\pi\colon\Spec(A)\to\Spec(R)}$ eine $R$-Algebra mit ${\pi(\m)=s}$.
  Dann faktorisiert $\pi$ über
  ${\Spec(A)\to\Spec(R_s)\to\Spec(R)}$ und für jeden Morphismus
  \begin{gather*}
    \Spec(\Quot(A))\to X
  \end{gather*}
  faktorisiert eine Erweiterung ${\Spec(A)\to X}$ über
  ${X\times_R\O_{R,s}}$:
  \begin{gather*}
    \Spec(A)\to X_s\hookrightarrow X\times_R\O_{R,s}\hookrightarrow X
  \end{gather*}
  Hier ist sie eindeutig wegen
  Separiertheit von ${X\times_R\O_{R,s}}$.

  Für Glattheit auf $X$ ist nur noch nötig, dass $X$ flaches
  $R$-Schema ist, da $R$ noethersch und $X$ nach Voraussetzung von
  endlichem Typ ist, also endlich präsentiert.
  Für ein ${x\in X_s\subset X}$, ${s\in\Spec(R)}$, ist der Halm
  ${\O_{X,x}=\O_{(X\times_R\O_{R,s}),x}}$ wegen der Glattheit von
  ${X\times_R\O_{R,s}}$ bereits flach über $R_s$.

  Es bleibt die \NAbbEig zu zeigen.
  Sei ${\K=\Quot(R)}$ und betrachte ein glattes $R$-Schema $Y$ sowie
  einen $\K$-Morphismus ${f_\K\colon Y_\K\to X_\K}$ der
  generischen Fasern.
  Als glattes $R$-Schema ist $Y$ flach, reduziert und insbesondere
  regulär, weshalb die irreduziblen Komponenten alle offen und
  disjunkt sind. Es ist also ohne Einschränkung anzunehmen, dass
  $Y$ integer ist.
  Nach Voraussetzung erweitert sich $f_\K$ für einen abgeschlossenen
  Punkt $s\in\Spec(R)$ eindeutig zu einem Morphismus
  \begin{gather*}
    f_s\colon Y\times_R\O_{R,s}\to X\times_R\O_{R,s}\;,
  \end{gather*}
  da $X\times_R\O_{R,s}$ Néron-Modell seiner generischen Faser $X_\K$
  ist.
  % Nach \ref{thm:limesoffenemengen} ist $u_s$ die Einschränkung
  % einer eindeutigen Äquivalenzklasse im direkten Limes
  % $\varinjlim \Hom_{S'}(X\times_S S', Y\times_S S')$ über die
  % offenen Umgebungen $S'\subset S$ von $s$.
  % Daher gibt es eine offene Umgebung $S'(s)$ von $s$, über der $u_s$
  % eindeutig auf einen $R$-Morphismus $u(s)\colon S'(s)\to X$
  % erweitert wird.
  % % ??? Warum kann ich eine offene Menge wählen?
  Nach \ref{thm:allgmorphismuserweiterung} gibt es eine offene
  Umgebung $U_s$ von $Y\times_R\O_{R,s}$, auf der
  $f_s|_{Y\times_R\O_{R,s}}$ wiederum
  eindeutig auf einen Morphismus $u_s\colon U_s\to X$ erweitert
  wird.
  Die $u_s$ stimmen wieder wegen Separiertheit von $X$
  nach \ref{thm:erweindeutig} auf dem Schnitt ihrer
  Definitionsbereiche überein, die zudem ganz $Y$ überdecken
  ($\Spec(R)$ besteht genau aus seinen
  abgeschlossen Punkten und dem generischen Punkt, und letzterer ist
  in jeder offenen Teilmenge enthalten).
  Daher liefert Verkleben der eindeutigen $u_s$ einen
  eindeutig bestimmten Morphismus $f\colon Y\to X$, der
  $f_\K$ erweitert.
  Das heißt die \NAbbEig ist erfüllt.
\end{proof}


\subsection{Néron-Modelle abelscher Varietäten}
Nachdem das Hauptaugenmerk der Arbeit explizit auf Néron-Modellen
abelscher Varietäten liegt, seien hier einige Eigenschaften aufgeführt.
Wichtig ist unter anderem, dass die Existenz in diesem Fall allgemein
gesichert ist, was für elliptische Kurven später explizit nachgewiesen
wird.
$R$ sei in diesem Abschnitt wieder ein Dedekindring und
$\K\coloneqq\Quot(R)$.
% ALLG. EXISTENZ
% \begin{Satz}[Existenz von Néron-Modellen]\label{thm:existenz}
%   \cite[1.3, Theorem 1]{neron}
%   Sei $\Rsh$ die strikte Henselisierung von $R$ und $\Ksh$ der
%   Quotientenkörper. Ein glattes $\K$"=Gruppenschema $X_\K$ endlichen
%   Typs hat genau dann ein Néron-Modell über $R$, wenn $X_\K(\Ksh)$
%   beschränkt in $X_\K$ ist.
%   %   eigentlich => beschränkt
% \end{Satz}
Zur Existenz von Néron-Modellen im Fall abelscher Varietäten besagt
\cite[Corollary~1.3/2]{neron} vorerst ohne explizite Konstruktion:
\begin{Satz}\label{thm:exneronmodellabvarietaet}
  Abelsche Varietäten über $\K$ haben ein Néron-Modell über $R$.
  % \begin{proof}
  %   Abelsche Varietäten über $\K$ sind insbesondere glatte und
  %   eigentliche $\K$"=Gruppenschemata, siehe
  %   \ref{def:abelschevarietaet}.
  %   Nachdem eigentlich beschränkt impliziert, ist die Bedingung aus
  %   \ref{thm:existenz} erfüllt.
  % \end{proof}
\end{Satz}

% Für die folgenden Aussagen ist beim Übergang zum allgemeinen Fall
% eines allgemeinen Dedekindschemas die Eigenschaft zusammenhängend
% nötig.

Abelsche Varietäten unterscheiden sich von generellen glatten,
separierten $\K$"=Gruppenschemata dadurch, dass die Gruppen ihrer
Schnitte abelsch sind. Da diese Eigenschaft für Gruppenschemata
sehr wünschenswert ist, fragt sich, inwieweit das Néron-Modell einer
abelschen Varietät wieder abelsch ist. 
\begin{Satz}\label{thm:abelscheneronmodelle}
  Sei $A_\K$ abelsche Varietät über $\K$ mit guter Reduktion,
  das heißt es gibt ein glattes, eigentliches $R$-Modell $A$ von $A_\K$.
  Dann ist $A$ Néron-Modell von $A_\K$ und die induzierte
  Gruppenschemastruktur macht $A$ zu einem abelschen Schema.
  \begin{proof}[Beweisskizze] Für den vollständigen Beweis siehe
    \cite[Proposition~1.4/2]{neron}.
    Nach \ref{thm:exneronmodellabvarietaet} existiert ein
    Néron-Modell $X$ von $A_\K$, das nach
    \ref{thm:gruppenschemaerweiterung} eine Gruppenschemastruktur
    trägt.
    Es bleibt zu zeigen, dass $A$ isomorph zu $X$ ist
    mit zusammenhängenden Fasern (dann ist $A$ ist abelsches Schema,
    da $A$ nach Voraussetzung bereits glatt und eigentlich ist).
    \begin{description}[font=\normalfont\itshape]
    \item[$A$ isomorph zu $X$:]
      Die \NAbbEig von $X$ liefert zur Identität $\Id_{A_\K}$ einen
      $R$-Morphismus $f\colon A\to X$, der bereits offene Immersion ist
      \cite[siehe][4.3/1~(ii) oder 4.4/1]{neron}.
      % ?? Wieso ist die Erweiterung von id hier offene Immersion?
      Es ist zu zeigen, dass das Bild von $f$ ganz $X$ ist, woraus
      wegen Reduziertheit von $X$ bereits folgt, dass $f$ ein
      Isomorphismus ist.
      $X$ ist glatt, also nach
      \ref{thm:eigglatt} flaches und lokal endlich präsentiertes
      $R$-Schema.
      Als solches ist es (universell) offen nach
      \cite[Theorem~14.33]{wedhorn} und somit zusammenhängend nach
      \cite[Proposition~3.24]{wedhorn},
      da die generische Faser $A_\K$ zusammenhängend und $\Spec(R)$
      ein integres Schema ist.
      Es ist also nur zu zeigen, dass das Bild von $f$ abgeschlossen
      ist, da es dann als offene und abgeschlossene Teilmenge des
      zusammenhängenden $X$ schon ganz $X$ sein muss.
      $A$ ist nach Voraussetzung eigentlich, weswegen ${A\to R}$
      universell abgeschlossen und insbesondere der Basiswechsel
      ${A\times_R X\to X}$ abgeschlossen ist.
      Nachdem $X$ separiert ist, ist der Graph
      ${\Gamma_f\colon A\to A\times_R X}$ und daher das Bild von
      \begin{gather*}
        f\colon A\xrightarrow{\Gamma_f} A\times_R X
        \xrightarrow{f\times_R\Id} X
      \end{gather*}
      abgeschlossen in $X$.
    \item[$A$ hat zusammenhängende Fasern:]
      Ohne Einschränkung sei $R$ diskreter Bewertungsring nach
      \ref{thm:neronmodelllokal}.
      Dann ist nur noch die spezielle Faser über dem abgeschlossenen
      Punkt zu überprüfen, die nach \cite[{}5.5.1]{EGAIII-1} zusammenhängend
      ist.
      % ??? –> Zariski's Connectedness Thm/Stein-Faktoriesierung?
      % \cite[Exercise 12.29]{wedhorn}
      \qedhere
    \end{description}
  \end{proof}
\end{Satz}

Mit diesem Ergebnis kann für Néron-Modelle abelscher Varietäten
eine Aussage dazu getroffen werden, wo die induzierte
Gruppenschemastruktur abelsch ist.
\begin{Satz}\label{thm:teilwabelscheneronmodelle}
  Sei $A_\K$ abelsche Varietät über $\K$,
  $X$ ihr Néron-Modell über $R$ und
  $S\subset\Spec(R)$ die Menge des generischen Punkts zusammen mit
  allen abgeschlossenen Punkte $s\in\Spec(R)$, über denen $A_\K$ gute
  Reduktion hat, also für die $A_\K$ ein glattes, eigentliches
  $R_s$-Modell hat.
  $S$ ist entsprechend offen und dicht in $\Spec(R)$.
  Dann ist $X\times_R S$ abelsches Schema über $S$.
  \begin{proof}[Beweisskizze] Siehe \cite[Theorem~1.4/3]{neron} für
    den vollständigen Beweis.
    Es gilt zu zeigen, dass $A_\K$ in einer Umgebung $U_s$ jedes
    Punktes $s\in S$ gute Reduktion hat, also die generische Faser eines
    eigentlichen, glatten Schemas über $U_s$ ist.
    Diese Schemata sind dann nach \ref{thm:abelscheneronmodelle}
    bereits jeweils über $U_s$ Néron-Modelle und abelsch.
    Wegen der \NAbbEig und der resultierenden Eindeutigkeit von
    Néron-Modellen über den offenen Teilmengen von $\Spec(R)$
    lassen sie sich zu einem abelschen $S$-Schema verkleben.

    In den abgeschlossenen Punkten von $S$ hat $A_\K$ nach Definition
    gute Reduktion, welche sich auf eine offene Umgebung erweitern
    lässt.
    % Gute Reduktion in s\in S <=> X\times_S\O_{S,s} eigentlich
    % (\ref{thm:neronmodell}, \ref{thm:abelscheneronmodelle}
    % Erweiterung von Morphismen auf Umgebung:
    % \ref{thm:allgmorphismuserweiterung}
    % ?? Wieso erhält Morphismuserweiterung Eigentlichkeit?
    
    Für den generischen Punkt kann der schematische
    Abschluss $A$ des projektiven (s.~\ref{thm:abvarietaetprojektiv})
    $A_\K$ in $\P_R^n$ betrachtet werden, das heißt das schematische
    Bild von
    \begin{gather*}
      A_\K\hookrightarrow\P_\K^n\to\P_R^n
    \end{gather*}
    Hier ist es explizit der Abschluss des Bildes mit reduzierter
    Unterschemastruktur nach \cite[Remark~10.32]{wedhorn}, da $A_\K$
    reduziert ist.
    Dieser ist glatt über dem generischen Punkt von $\Spec(R)$
    und somit auch über einer offenen Umgebung $S'$ dessen.
    Insgesamt ist es ein glattes, projektives, also eigentliches,
    $S'$-Modell von $A_\K$ um den generischen Punkt, was gesucht war.
  \end{proof}
\end{Satz}

% \section{Beispiele}

% \begin{Satz}
%   \cite[1.2, Proposition 8]{neron}
%   Ein abelsches Schema über $\Spec(R)$ ist Néron-Modell seiner
%   generischen Faser.
% \end{Satz}

% \begin{Satz}
%   \cite[1.2, Criterion 9]{neron}
%   Ein glattes, separiertes $\Spec(R)$-Schema endlichen Typs ist genau
%   dann ein Néron-Modell seiner generischen Faser, wenn es die
%   Erweiterungseigenschaft für étale Punkte erfüllt.
% \end{Satz}

% ----------


%%% Local Variables:
%%% mode: latex
%%% TeX-master: "neron_models"
%%% End:
