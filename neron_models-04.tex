\chapter{Weierstraßmodelle}\label{chap:weierstraßmodelle}
In diesem Kapitel werden wir näher auf die Konstruktion eines
projektiven $R$-Schemas $W$ aus einer Weierstraßgleichung über dem
Quotientenkörper eines Dedekindrings eingehen.
Ein solches Schema ist zwar eigentlich – erfüllt also die \NAbbEig – ist
aber nicht notwendigerweise glatt und die Gruppenverknüpfung
erweitert sich nicht immer zu einem Morphismus auf ganz $W$.
Letzteres kann man beheben, indem man sich auf das Unterschema der
glatten Punkte $W^0$ von $W$ beschränkt. Denn hier erweitert sich die
Gruppenverknüpfung und macht $W^0$ zum Gruppenschema, wie wir in
\autoref{chap:weierstraßgruppe} sehen werden.

Vorweg sei schon gesagt, dass $W^0$ nur groß genug für die
\NAbbEig ist, wenn $W$ regulär ist. Allerdings wird sich aus
der späteren Konstruktion ergeben, dass $W^0$ immer ein großer Teil
des Néron-Modells ist.

Für unsere Konstruktion eines Néron-Modells beginnen wir in jedem Fall
mit einem Weierstraßmodell wie unten beschrieben.

\section{Konstruktion}
\subsection{Weierstraßgleichungen}
Jede elliptische Kurve $E_\K$ über einem beliebigen Körper $\K$ kann
durch eine homogene Gleichung der Form
\begin{gather*}
  0 = Y^2 Z + \c_1 XYZ + \c_3 YZ^2 - X^3 - \c_2X^2 Z - \c_4 XZ^2 - \c_6 Z^3
  \eqqcolon F\in\K[X,Y,Z]
\end{gather*}
mit Koeffizienten $\c_1,\dotsc,\c_6\in\K$ und Nullpunkt $O=[0,1,0]$
dargestellt werden, das heißt sie ist isomorph zur projektiven Kurve
$V_+(F)\subset\P_\K^2$.
Umgekehrt ist jede glatte Kurve, die durch eine
solche Gleichung definiert wird, eine elliptische Kurve mit genanntem
Nullpunkt nach \cite[Proposition III.3.1]{silverman}.
Gleichungen dieser Form werden \emph{Weierstraßgleichungen} genannt.

Für $\Char(\K)\neq 2,3$ lässt sich eine Weierstraßgleichung durch
homogene Koordinatentransformation vereinfachen zur Form
\begin{gather*}
  Y^2 Z = X^3 + \beta XZ^2 + \gamma Z^3
  \quad\text{bzw. dehomogenisiert}\quad
  y^2 = x^3 + \beta x + \gamma
\end{gather*}
wobei $\beta,\gamma\in\K$.
Für Berechnungen wird im Folgenden von dieser vereinfachten Form
ausgegangen. Für $\Char(\K)=2,3$ sei verwiesen auf
\cite[Appendix: Elliptic Curves in Characteristics 2 and 3]{silverman}.


% DISKRIMINANTE
Zu einer Weierstraßgleichung ist die \emph{Diskriminante} und die
\emph{$j$-Invariante} assoziiert, welche für eine Gleichung in obiger
Form definiert sind als 
\begin{align*}
  \Delta &= -16\left(4\beta^3 + 27\gamma^2\right)
  &j &= (-16)\cdot 2^8\cdot 3^3\cdot\beta^3
\end{align*}
Eine Weierstraßgleichung ist genau dann regulär (definiert also eine
elliptische Kurve), wenn $\Delta\neq 0$ gilt,
und im singulären Fall gibt es genau einen singulären Punkt
nach \cite[Proposition III.1.4]{silverman}.


\subsection{Weierstraßmodelle}
% (MINIMALE) WEIERSTRASSGLEICHUNG ÜBER $R$
Sei im folgenden Abschnitt $R$ Dedekindring,
$\K=\Quot(R)$ und $E_\K$ elliptische Kurve über $\K$.

Uns interessiert nun, ob und wie sich eine Weierstraßgleichung mit
Koeffizienten in $R$ finden lässt, also entsprechend ein $R$-Modell
von $E_\K$.
Bleiben wir bei unserer Annahme $\Char(\K)\neq2,3$, sind die
Isomorphismen zwischen Weierstraßkurven zu $E_\K$ genau durch
Koordinatenwechsel der Form
\begin{gather*}
  [x,y,z] \longmapsto [a^{-2n}x, a^{-3n}y, z] \qquad a\in\K, n\in\Z
\end{gather*}
mit dem Koeffizientenwechsel
\begin{align*}
  \beta&\mapsto a^{4n}\beta
  &\gamma&\mapsto a^{6n}\gamma
  &\Delta&\mapsto a^{12n}\Delta
\end{align*}
gegeben, siehe \cite[1.5, Lemma 2]{neron}
oder \cite[Chapter VII.1]{silverman}.
Sind $\beta=\frac{\beta_1}{\beta_2}$,
$\gamma=\frac{\gamma_1}{\gamma_2}$ mit $\beta_i,\gamma_i\in R$,
erzeugt ein solcher Koordinatenwechsel mit $a=\gamma_2\beta_2$ eine
Weierstraßgleichung mit Koeffizienten in $R$.
% Beschränken wir unsere Betrachtung auf elliptische Kurven über
% Bewertungskörpern $\K$ mit diskretem Bewertungsring $R$, können wir
% die Weierstraßgleichung durch homogenen Koordinatenwechsel bezüglich
% der Ringstruktur formulieren:
% Sei $\pi$ ein Uniformierer.
% Die Koeffizienten haben die Form $\beta=u_\beta\pi^n_\beta$, $\gamma=u_\gamma\pi^n_\gamma$ und
% entsprechend liegen beide nach Multiplikation mit $\phi^n$ in $R$ für
% genügend großes $n$. Zudem kommt der homogene Koordinatenwechsel
% \begin{gather*}
%   [x,y,z] \longmapsto [\pi^{-2n}x, \pi^{-3n}y, z]
% \end{gather*}
% dem Koeffizientenwechsel
% \begin{align*}
%   \beta&\mapsto \pi^{4n}\beta
%   &\gamma&\mapsto \pi^{6n}\gamma
%   &\Delta&\mapsto \pi^{12n}\Delta
% \end{align*}
% also der Multiplikation der Koeffizienten mit positiven Potenzen von
% $\pi$ gleich
% (\cite[1.5, Lemma 2]{neron} oder \cite[Chapter VII.1]{silverman}).
% Ein Anwenden dieses Koordinatenwechsels mit genügend hohem $n$ ergibt
% also eine Gleichung mit Koeffizienten in $R$.

\begin{Definition}[Weierstraßmodell]
  Wir nennen ein projektives $R$-Schema $W=V_+(F)\subset\P_R^2$, das
  von einer homogenen Weierstraßgleichung $0=F\in R[X,Y,Z]$ von $E_\K$
  mit Koeffizienten in $R\subset\K$ erzeugt wird,
  ein Weierstraßmodell von $E_\K$ über $R$.
\end{Definition}

Ist $R$ ein diskreter Bewertungsring, können wir die Bewertung der
Diskriminante durch einen Koordinatenwechsel manipulieren. Daher macht
es Sinn, eine bzgl. der gegebenen diskreten Bewertung minimale
Form der Weierstraßgleichung in $R$ zu betrachten.
\begin{Definition}[minimale Weierstraßgleichung]
  \optcite[Chapter 1.5, S.\,22]{neron}
  \optcite[Chapter VII.1]{silverman}
  Sei $R$ ein diskreter Bewertungsring.
  Eine Weierstraßgleichung ($\Char(\K)\neq2,3$) von $E_\K$
  \begin{gather*}
    Y^2Z = X^3 + \beta XZ^2 + \gamma Z^3
  \end{gather*}
  heißt minimal, wenn für die Koeffizienten $\beta, \gamma\in R$ gilt
  und die Bewertung $\ord_\K(\Delta)$ der Diskriminante minimal
  bzgl. homogenem Koordinatenwechsel ist.
  Wir bezeichnen ein Weierstraßmodell von $E_\K$ über $R$ als minimal,
  wenn es von einer minimalen Weierstraßgleichung von $E_\K$ erzeugt
  ist.
\end{Definition}
Es stellt sich heraus, dass sich die spezielle Faser des
Weierstraßmodells einer elliptische Kurve anhand der minimalen
Bewertung der Diskriminante und der $j$-Invariante in eine von zehn
Kategorien klassifizieren lässt. Siehe dazu auch \autoref{chap:ausblick}. 

% -----

\section{Gruppenschemastruktur}\label{chap:weierstraßgruppe}
\subsection{Allgemeine Eigenschaften}
Im Folgenden ein paar nützliche Eigenschaften von Weierstraßmodellen,
die wir unter anderem für den Nachweis der Gruppenstruktur auf dem
glatten Teil eines Weierstraßmodells brauchen werden.

% INTEGER, PROJEKTIV, FLACH
\begin{Lemma}\label{thm:eigweierstrassmodelle}
  Ein Weierstraßmodell $W=V_+(F)$ zu einer Weierstraßgleichung $0=F$
  von $E_\K$ über einem Dedekindring $R$ mit $\K=\Quot(R)$ ist
  integer,
  projektiv, insbesondere quasikompakt und noethersch,
  endlich präsentiert, insbesondere von endlichem Typ, und
  flach.
  \begin{proof}
    \begin{description}
    \item[projektiv] $W$ ist per Konstruktion projektiv. Als
      abgeschlossenes Unterschema des quasikompakten, noetherschen Raums
      $\P_R^n$ ist es quasikompakt und noethersch.
    \item[integer]
      $W=V_+(F)$ ist genau dann irreduzibel, wenn $F\in R[X,Y,Z]$
      irreduzibel ist.
      Da nach Voraussetzung $E_\K\cong V_+(F)\subset\P_\K^2$
      elliptische Kurve ist, vor allem irreduzibel, ist $F$ irreduzibel
      als Polynom in $\K[X,Y,Z]$. Nach Algebra impliziert das bereits
      die Irreduzibilität von $F\in R[X,Y,Z]$.
      % Dass $W$ irreduzibel ist, kann auf der affinen, offenen
      % Standardüberdeckung des $\P_R^2$ durch drei Kopien des $\A_R^2$
      % überprüft werden, da diese nicht-leeren Schnitt haben. Hier
      % erhalten wir die Kurven im $\A_R^2$, die durch die
      % dehomogenisierten Formen der Weierstraßgleichung $f(X,Y,Z)=0\in
      % R[X,Y,Z]$, d.\,h. $f_Z\in R[\frac{X}{Z},\frac{Y}{Z}]=R[x,y]$,
      % $f_Y\in R[x,z]$, $f_X\in R[y,z]$, erzeugt werden.
      % Diese stimmen nach Konstruktion mit den Erzeugern für die affinen
      % Kurven zu $E_\K$ überein und sind daher irreduzibel über $\K$,
      % also auch irreduzibel über $R$.
      % Entsprechend ist die gewählte affine, offene Überdeckung von $W$
      % irreduzibel und reduziert und somit $W$ integer.
    \item[endlich präsentiert]
      $W$ ist offen überdeckt von seinen drei dehomogenisierten,
      affinen Teilmengen $V(F_i)=\Spec(R[x,y]/(F_i))$, wobei $F_X,F_Y,F_Z$
      Dehomogenisierungen von $F$ nach $X,Y,Z$ seien. Diese sind jeweils
      endlich präsentiert, also ist $W$ endlich präsentiert.
      Entsprechend folgt, dass $W$ von endlichem Typ ist.
    \item[flach]
      Wir werden zeigen, dass aus integer und nichtleerer generischer
      Faser $E_\K$ folgt, dass der eindeutige generische Punkt $\eta$
      von $W$ auf den generischen Punkt von $R$ geschickt wird.
      Das zeigt, dass $W$ flach ist
      nach \cite[Proposition III.9.7]{hartshorne}, denn
      $\Spec(R)$ ist integer, regulär und von Dimension~1 und
      $W$ ist reduziert, womit alle Voraussetzungen des Satzes erfüllt
      sind.
      
      Nachdem $W$ integer ist, hat es einen eindeutigen generischen
      Punkt $\eta$.
      Angenommen, dieser wird durch die Abbildung, die $W$ zum
      $R$-Schema macht, auf einen abgeschlossenen Punkt $\m$ von $\Spec(R)$
      geschickt.
      Dann liegt $\eta$ im abgeschlossenen Urbild von $\m$ –
      aber der Abschluss $\overline{\{\eta\}}=X$ von $\eta$ ist darin
      enthalten. Entsprechend wäre die generische Faser $E_\K$
      leer, was ein Widerspruch ist.
      Also muss der generische Punkt auf den generischen Punkt geschickt
      werden.
    \end{description}
  \end{proof}
\end{Lemma}
% Für Regularität siehe auch \cite[Lemma IV.9.5]{silverman2}

% ÄQ GLATTHEIT
% \begin{Lemma}\label{thm:glattüberdvr}
%   Sei $R$ diskreter Bewertungsring,
%   $\K=\Quot(R)$,
%   $X$ irreduzibles,
%   lokal endlich präsentiertes
%   (für lokal noethersch äquivalent:
%   lokal endlichen Typs \cite[Remark 10.36]{wedhorn})
%   $R$-Schema mit nichtleerer, generischer Faser $X_\K$.
%   Dann ist $X$ genau dann glattes $R$-Schema, wenn die Fasern glatt
%   sind.
%   \begin{proof}
%     %     siehe auch \cite[Proposition IV.2.9]{silverman2}
%     Ist $X$ glatt, so auch jede Faser, da Glattheit stabil unter
%     Basiswechsel ist.

%     Seien also die Fasern von $X$ glatt, insb. reduziert.
%     Dann ist $X$ ebenfalls reduziert, da Reduziertheit auf den Halmen
%     geprüft werden kann.
%     Außerdem ist $X$ nach Voraussetzung irreduzibel, daher integer,
%     und (lokal) endlich präsentiert.
%     Wir wissen, dass $X$ glatt ist, wenn es lokal endlich präsentiert
%     und flach über $R$ ist und seine Fasern glatt sind
%     \cite[8.5, Proposition 17]{bosch}.
%     Es ist also Flachheit zu zeigen.

%     Wir werden zeigen, dass aus integer und nichtleerer generischer
%     Faser $X_\K$ folgt, dass der eindeutige generische Punkt $\eta$
%     von $X$ auf den generischen Punkt von $R$ geschickt wird.
%     Das zeigt, dass $X$ flach ist
%     \cite[Proposition III.9.7]{hartshorne}
%     (hierbei wird zusätzlich benutzt, dass $\Spec(R)$ integer,
%     regulär und von Dimension 1 ist).

%     Nachdem $X$ integer ist, hat es einen eindeutigen generischen
%     Punkt $\eta$.
%     Angenommen, dieser wird durch die Abbildung, die $X$ zum
%     $R$-Schema macht, auf einen geschlossenen Punkt $\m$ von $\Spec(R)$
%     geschickt.
%     Dann liegt $\eta$ im geschlossenen Urbild von $\m$ –
%     aber der Abschluss $\overline{\{\eta\}}=X$ von $\eta$ ist darin
%     enthalten. Entsprechend wäre die generische Faser $X_\K$
%     leer, was ein Widerspruch ist.
%     Also muss der generische Punkt auf den generischen Punkt geschickt
%     werden.
%   \end{proof}
% \end{Lemma}

\begin{Lemma}\label{thm:weierstrassglatt}
  \optcite[Remark IV.5.4.1]{silverman2}
  \optcite[Chapter VII.5 für die Definition von guter Reduktion]{silverman}
  % Gute Reduktion auf elliptischen Kurven heißt, dass die spezielle
  % Faser von $W$ regulär (also glatt also ebenfalls elliptische
  % Kurve) ist
  % \optcite[Chapter VII.5]{silverman}.
  Sei $(R,\m)$ diskreter Bewertungsring,
  $\K=\Quot(R)$,
  der Restklassenkörper $k=R/\m$ perfekt,
  $E_\K$ elliptische Kurve über $\K$ und
  $W$ ein Weierstraßmodell von $E_\K$ über $R$
  mit spezieller Faser $W_\m$ über $k$.
  Dann gilt
  \begin{enumerate}[label=(\alph*)]
  \item $W$ ist genau dann glatt, wenn die spezielle Faser von $W$ glatt
    ist.
  \item Ist $W$ nicht glatt, so ist der eindeutige, abgeschlossene,
    singuläre Punkt $\gamma$ der speziellen Faser der eindeutige
    Punkt, in dem $W$ nicht glatt ist, d.\,h.
    \begin{gather*}
      W^0\coloneqq W\setminus \{\gamma\}
    \end{gather*}
    ist das offene Unterschema der glatten Punkte von $W$.
    $W^0$ ist glattes $R$-Modell von $E_\K$.
  \end{enumerate}
  \begin{proof}
    \begin{enumerate}[label=(\alph*)]
    \item $W$ ist flaches, lokal endlich präsentiertes
      $R$-Schema (s. \ref{thm:eigweierstrassmodelle}).
      Nach \cite[8.5, Proposition 17]{bosch} ist es also genau dann
      glatt, wenn seine Fasern glatt sind.
      Nachdem $W_\K=E_\K$ nach Voraussetzung glatt ist, ist $W$ genau
      dann glatt, wenn $W_\m$ glatt ist.
    \item Für Schemata lokal endlichen Typs über einem Körper sind
      glatt und geometrisch regulär äquivalent
      \cite[Corollary 6.32]{wedhorn}.
      Da $k$ perfekt ist, ist dies für $k$-Schemata sogar
      äquivalent zu regulär
      \cite[Remark 6.33]{wedhorn}.
      Schemata lokal endlichen Typs über einem Körper sind genau dann
      regulär, wenn sie in allen abgeschlossenen Punkten regulär sind
      \cite[Remark 6.25 (3)]{wedhorn}. Insbesondere sind alle
      singulären Punkte abgeschlossen.

      $W_\m$ ist als Weierstraßkurve über $k$ ein $k$-Schema
      lokal endlichen Typs, also treffen obige Aussagen zu.
      Nach \cite[Proposition III.1.4]{silverman} ist unter den
      abgeschlossenen Punkten von $W_\m$ maximal ein singulärer Punkt
      $\gamma$. Demnach kann $W_\m$ maximal in einem abgeschlossenen
      Punkt $\gamma\in W_\m$ singulär (bzw. nicht glatt) sein.

      Angenommen, $W_\m$ habe den singulären Punkt $\gamma$.
      $W_\m=W\setminus E_\K$ ist als Urbild des abgeschlossenen Punktes
      $\m$ von $\Spec(R)$ ein abgeschlossenes Unterschema von $W$.
      Also ist der Punkt $\gamma$, der abgeschlossen in $W_\m$ ist, auch
      abgeschlossener Punkt von $W$.
      
      Das Unterschema $W^0=W\setminus\gamma$ ist dementsprechend offen.
      $W^0$ ist irreduzibel, da der Abschluss $\overline{W^0}=W$ in
      $W$ irreduzibel ist nach \ref{thm:eigweierstrassmodelle}.
      Es ist als Unterschema von $W$ ebenfalls reduziert, endlich
      präsentiert und flach (s. \ref{thm:eigweierstrassmodelle}).
      Es hat wie $W$ die (nichtleere) glatte, generische Faser $E_\K$ und
      per Konstruktion die spezielle Faser $W_{\m}\setminus\{\gamma\}$,
      welche nach Voraussetzung beide regulär über perfekten Körpern,
      also glatt, sind.
      Damit ist $W^0$ wieder nach \cite[8.5, Proposition 17]{bosch}
      glatt über $R$ bzw. glattes $R$-Modell von $E_\K$ und das offene
      Unterschema der glatten Punkte von $W$.
    \end{enumerate}
  \end{proof}
\end{Lemma}

\begin{Bemerkung}\label{thm:weierstraßmodellglatt}
  Für den Fall eines diskreten Bewertungsringes $(R,\m=(\pi))$ liefert
  \ref{thm:weierstrassglatt} ein sehr einfaches Kriterium für die
  Glattheit eines Weierstraßmodell $W$ von $E_\K$.
  $W$ ist genau dann glatt, wenn seine spezielle Faser glatt
  (bzw. regulär) ist.
  Dies lässt sich recht leicht an der Weierstraßgleichung ablesen.
  
  Die spezielle Faser $W_{\m}=W\times_R (R/\m)$ ist definiert
  durch die Reduktion der Weierstraßgleichung von $W$ modulo dem
  Maximalideal $\m$ von $R$, also ebenfalls eine Weierstraßkurve.
  Entsprechend ist $W_{\m}$ (und damit $W$) genau dann glatt, wenn
  eine der äquivalenten Bedingungen erfüllt ist
  \begin{enumerate}[label=(\roman*)]
  \item $0\neq\Delta_{\m}=\bar\Delta\in R/\m$
  \item $\Delta\not\in\m$ bzw. $\pi\nmid\Delta$
  \item $\ord_R(\Delta)=0$ bzw. $\Delta\in R^\times$
  \end{enumerate}
  wobei $\Delta_{\m}$ Diskriminante von $W_{\m}$ und $\Delta$
  Diskriminante von $W$ ist,
  \cite[vgl.][Proposition VII.5.1 (a)]{silverman}.
  Die letzte Bedingung ist einfach nachzuprüfen.
\end{Bemerkung}

\subsection{Gruppenstruktur auf Weierstraßmodellen}
Das Folgende Lemma weist die Gruppenstruktur auf dem glatten Teil
eines Weierstraßmodells nach. Es wird nur eine kurze Beweisskizze
geliefert, da die Aussage nur ein Spezialfall von
\ref{thm:exneronmodelle} ist.
\begin{Lemma}\label{thm:erweiterunggruppenstruktur}
  Sei $(R,\m)$ wieder diskreter Bewertungsring,
  $k=R/\m$ perfekt,
  $\K=\Quot(R)$.
  Sei $W$ ein Weierstraßmodell von $E_\K$ über $R$ und
  $W^0$ das offene $R$-Unterschema der glatten Punkte von $W$ 
  (ebenfalls $R$-Modell von $E_\K$ nach \ref{thm:weierstrassglatt}).
  $W^0$ ist ein glattes $R$"=Gruppenschema
  mit Addition, Inversenbildung und neutralem Schnitt
  \begin{align*}
    \mu_0\colon W^0\times_R W^0 &\longto W^0
    &i_0\colon W^0&\longto W^0
    &\epsilon_0\colon R&\longto W^0
  \end{align*}
  welche die Gruppenstruktur (Addition $\mu$, Inversenbildung $i$ und
  neutraler Schnitt $\epsilon$)
  der abelschen Varietät $E_\K$ erweitern.
  \optcite[Theorem IV.5.3 (c)]{silverman2}

  % % ??? Referenzen
  % $W^0(R)\to W(R) \cong E_\K(\K)$ ist ein Gruppenisomorphismus,
  % falls $W$ regulär ist.
  % \optcite[Theorem IV.5.3 (b)]{silverman2}
  
  \begin{proof}[Beweisskizze] Sei wieder $\Char(\K)\neq 2,3$.
    \ref{thm:erweiterunggruppenstruktur} folgt als Spezialfall von
    \ref{thm:exneronmodelle}.
    Für das Nachweisen der Gruppenstruktur ist zu zeigen, dass die von
    $E_\K$ induzierten, rationalen Abbildungen $\mu\colon W\times_R
    W\dra W$ und $i\colon W\dra W$ auf $W^0$ definiert sind.

    $i$ ist in projektiven Koordinaten (also als graduierter
    Ringmorphismus der zugrundeliegenden graduierten Ringe)
    dargestellt als
    \begin{gather*}
      i\colon [X,Y,Z]\mapsto [X,-Y,Z]
    \end{gather*}
    Dies definiert einen Isomorphismus auf ganz $\P_R^2$, ist also
    insbesondere auf $W^0$ definiert.

    Für $\mu$ kann die zugehörige Darstellung als Morphismus
    graduierter Polynomringe über $\K$ auf einen Morphismus von
    Polynomringen über $R$ erweitert werden \cite[vgl. Formeln für
    Addition in][Group Law Algorithm III.2.3]{silverman}.
    Berechnungen wie in \cite[Proposition 2.5]{silverman} zeigen, dass
    $\mu$ auf $W\times_R W$ höchstens im singulären Punkt nicht
    definiert ist, falls dieser existiert.
    Dies kann auf der affinen Überdeckung von $W$ nachgerechnet werden.    
    % \begin{enumerate}
    % \item $W^0=W\setminus\{\gamma\}$: \ref{thm:weierstrassglatt}
    %     %   \cite[Remark IV.5.4.2]{silverman2}
    %     %   Die nicht-glatten Punkte von $W$ sind die singulären Punkte
    %     %   der Fasern.
    %     %   Die generische Faser ist regulär (ell. Kurve), die spezielle
    %     %   Faser ist durch eine Weierstraßgleichung über $k$
    %     %   definiert. Sie hat also maximal einen singulären Punkt $\gamma$.
    %     %   \cite[Proposition III.1.4 (a)]{silverman}
    % \item $W^0$ ist $R$-Modell von $E_\K$: \ref{thm:weierstrassglatt}
    %     %   $W$ ist $R$-Modell von $E_\K$ per Definition,
    %     %   $W^0=W\setminus\{\gamma\}$ hat dieselbe generische Faser
    % \item Es treffen alle Voraussetzungen für
    %   \ref{thm:ratpkteregulaeremodelle} auf $W$ zu
    %   ($W$ ist projektiv, also eigentlich, von Dimension 2 mit
    %   projektiven Kurven als Fasern, und nach
    %   \ref{thm:weierstrassglatt} genau dann glatt in einem Punkt,
    %   wenn die Fasern dort regulär sind).
    %   Also gilt $E_\K(\K)=W(R)=W^0(R)$.
    %     %   \cite[Corollary IV.4.4]{silverman2}
    %     %   $E_\K(\K)=W(R)=W^0(R)$
    %     %   (s. \ref{thm:ratpkteregulaeremodelle}):
    %     %   \begin{enumerate}
    %     %   \item $E_\K(\K)=W(R)$: $W$ eigentlich
    %     %   \item $W(R)\subset W^0(R)$:
    %     %     nach \cite[Proposition IV.4.3 (b)]{silverman2}
    %     %   \item $W(R)=W^0(R)$: Alle Bilder von Schnitten $W(R)$ sind
    %     %     glatt
    %     %   \end{enumerate}
    % \item rationale Abbildungen
    %   $\mu\colon W\times_R W\dra W$,
    %   $i\colon W\dra W$:
    %   von generischer Faser $E_\K$ induziert, erfüllen Gruppenaxiome
    %   auf (größter, offener) Definitionsmenge (wo es Morphismus ist)
    % \item $\mu$ definiert auf $W^0$:
    %   \begin{enumerate}
    %   \item Aufteilen in affine Überdeckung
    %     ($W_Z=W\cap U_Z=\{[x,y,z]\in W| z\neq0\}$,
    %     $W_Y=W\cap U_Y=\{[x,y,z]\in W| y\neq0\}$)
    %     $W_Z\times W_Z$, $W_Z\times W_Y$, $W_Y\times W_Z$,
    %     $W_Y\times W_Y$
    %   \item $\mu$ auf $W_Z\times W_Z\setminus(\gamma,\gamma)$
    %     definiert:
    %     Schreibe rationale Abbildung von $\mu$ auf und finde
    %     Gleichungen für Unterschema, an denen der Nullpunkt getroffen
    %     würde
    %   \end{enumerate}
    % \item $i$ definiert auf $W^0$:
    %   $i\colon [X,Y,Z]\mapsto [X,-Y,Z]$ ist Morphismus in $\P_R^2$,
    %   also definiert auf ganz $W$
    % \end{enumerate}
  \end{proof}
\end{Lemma}

% IDENTITÄTSKOMPONENTE
\begin{Bemerkung}[{\cite[Chapter 1.5, S.\,23]{neron},
  \cite[Corollary IV.9.1]{silverman2},
  \cite[S. 46]{tate}}]
  Für ein minimales Weierstraßmodell $W$ von $E_\K$ über $R$
  ist $W^0$ isomorph zur Identitätskomponente des Néron-Modells von
  $E_\K$ über $R$.
\end{Bemerkung}

% Im Folgenden sei $E_\K$ immer eine elliptische Kurve über dem
% Quotientenkörper $\K$ des Dedekindringes $R$.

Im glatten Fall ist das Néron-Modell wie versprochen bereits das
Weierstraßmodell mit der induzierten Gruppenstruktur.
\begin{Korollar}\label{thm:neronmausweierstrassgl}
  \optcite[Corollary IV.6.3]{silverman2}
  Sei $R$ hier Dedekindring mit perfekten Restklassenkörpern, $\K=\Quot(R)$,
  $E_\K$ elliptische Kurve über $\K$, $W$ Weierstraßmodell von $E_\K$
  über $R$.
  Ist $W$ glatt, so ist $W$ bereits das Néron-Modell von $E_\K$ über
  $R$.
  \begin{proof}
    Nach \ref{thm:neronmodelllokal} müssen wir nur prüfen, ob die 
    $W\times_R R_s$ für abgeschlossene Punkte $s\in\Spec(R)$
    Néron-Modelle von $E_\K$ über $R_s$ sind.
    Ist $W$ glatt, so ist auch $W\times_R R_s$ glattes
    Weierstraßmodell von $E_\K$ über $R_s$, da alles stabil unter
    Basiswechsel ist.
    Sei daher $R$ ohne Einschränkung bereits lokal, also diskreter
    Bewertungsring.
    
    % TODO!!! KAPUTT

    Für die \NAbbEig machen wir uns zunutze, dass $W$ nach
    \ref{thm:erweiterunggruppenstruktur} ein $R_s$-Gruppenschema ist.
    Sei $Y$ ein weiteres glattes $R$-Schema mit generischer Faser
    $Y_\K$ und $\phi_\K\colon Y_\K\to E_\K$ ein Morphismus.
    In \ref{thm:erwrrational} zeigen wir, dass $\phi_\K$ eine
    eindeutige $R$-rationale Abbildung $\phi\colon Y\dra W$ liefert.
    Nachdem $W$ eigentliches $R$-Gruppenschema und $Y$ glatt ist, ist
    $\phi$ nach \ref{thm:rationalzumorphismus} ein Morphismus, der
    $\phi_\K$ erweitert, und die \NAbbEig ist erfüllt.
  \end{proof}
\end{Korollar}

\begin{Lemma}\label{thm:erwrrational}
  Sei $R$ ein Dedekindring, $\K=\Quot(R)$, $W$ allgemein ein
  eigentliches $R$-Schema, $Y$ lokal noethersches, flaches, reguläres,
  integeres
  $R$-Schema und $\phi_\K\colon Y_\K\to W_\K$ ein Morphismus auf den
  generischen Fasern.
  Dann erweitert sich $\phi_\K$ eindeutig zu einer
  $R$-rationalen Abbildung $\phi\colon Y\dra W$.
  Dasselbe gilt für $Y$ glatt, da es dann als reguläres Schema die
  disjunkte Vereinigung seiner irreduziblen Komponenten ist.
  \begin{proof}
    $\Spec(R)$ und $Y$ sind nach Voraussetzung lokal noethersch,
    und der Morphismus ${\pi\colon Y\to R}$ ist flach über $R$. Daher
    gilt für ein $x\in Y$ nach \cite[Theorem 4.3.12]{liu}
    \begin{gather}\label{eq:stalkdimension}
      \dim\O_{Y,x} = \dim R_{\pi(x)} + \dim\O_{Y_{\phi(x)}, x}
    \end{gather}
    Für $s\in\Spec(R)$ und den generischen Punkt $\eta_{s}$ einer
    irreduziblen Komponente der Faser $Y_s$ von $Y\to R$ gilt
    $\dim\O_{Y_s, \eta_{s}}=0$, also $\dim\O_{Y,\eta_{s}}=\dim R_s=1$.
    Da $Y$ regulär über $R$ ist, ist $\O_{Y,\eta_{s}}$ diskreter
    Bewertungsring.
    Der Quotientenkörper von $\O_{Y,\eta_{s}}$ ist der lokale Ring
    eines generischen Punkts $\eta$ von $Y$. Da $Y$ flach
    ist, liegt $\eta$ nach \cite[Proposition III.9.7]{hartshorne} in
    der generischen Faser.
    Wir können also das Bewertungskriterium für Eigentlichkeit für $W$
    anwenden:
    \begin{center}
      \begin{tikzcd}
        \eta\arrow[r, equal]
        &\Spec(\k(\eta))
        \arrow[r, "{\phi_\K|_{\eta}}"] \arrow[d]
        & E_\K \arrow[r, hook]
        & W\arrow[d]\\
        \eta\cup\eta_{s}\arrow[r, equal]
        &\Spec(\O_{Y,\eta_{s}}) \arrow[rr]
        \arrow[urr, dashed, "{\exists!}", out=20, in=215]
        && \Spec(R)
      \end{tikzcd}
    \end{center}
    Damit erweitert sich $\phi_\K$ zu einem eindeutigen Morphismus
    $\Spec(\O_{Y,\eta_{s}})\to W$, der sich nach
    \ref{thm:morphismuserweiterung} und \ref{thm:erweindeutig}
    eindeutig auf einen Morphismus in einer Umgebung von $\eta_{s}$
    erweitert.
    
    Die so gewonnenen Morphismen in Umgebungen der generischen Punkte
    der Fasern können wiederum wegen der Eindeutigkeit aus
    \ref{thm:erweindeutig} verklebt werden
    und wir erhalten eine eindeutige $R$-rationale Abbildung
    $\phi\colon U\dra W$ auf einem offenen Unterschema $U\subset Y$.
    $U$ enthält alle generischen Punkte der Fasern von $Y\to R$ und
    ist somit $R$-dicht.
    $\phi$ ist demnach eindeutige $R$-rationale Erweiterung von
    $\phi_\K$.
  \end{proof}
\end{Lemma}
\begin{Bemerkung}
  Im Fall diskreter Bewertungsringe kann die Glattheit anhand der Fasern
  überprüft werden:
  Wir wissen aus \ref{thm:weierstraßmodellglatt},
  dass ein Weierstraßmodell $W$ von $E_\K$ über $R$ genau dann glatt
  ist, wenn $\ord_R(\Delta)=0$ für die Diskriminante der
  Weierstraßgleichung gilt.
  Somit ist beispielsweise das Weierstraßmodell $W$ über $\Z_{(5)}$
  zur (hier dehomogenisierten) Weierstraßgleichung
  \begin{gather*}
    y^2=x^3+x
  \end{gather*}
  einer elliptischen Kurve $E_\Q$ über $\Q$ glatt, da
  $\Delta=-16\cdot4\cdot1^3$ eine Einheit in $\Z_{(5)}$ ist.
  $W$ ist deshalb bereits das Néron-Modell von $E_\Q$.

  Nach \ref{thm:abelscheneronmodelle} gilt auch, dass die
  Gruppenstruktur auf glatten Weierstraßmodellen die eines abelschen
  Schemas ist.
  Dies ist gleich aus der Konstruktion klar, da die Fasern des
  Weierstraßmodells in diesem Fall ja alle elliptische Kurven sind.
\end{Bemerkung}

%%% Local Variables:
%%% mode: latex
%%% TeX-master: "neron_models"
%%% End:
