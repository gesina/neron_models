\documentclass[english, german, parskip=half]{scrreprt}
% \usepackage{mathpazo}
\usepackage{fontspec}
\setmainfont{Latin Modern Roman}
\usepackage[shorthands=off]{babel}
\usepackage[autostyle=try]{csquotes}

\usepackage[style=alphabetic, backend=biber]{biblatex}
\bibliography{neron_models.bib}

\usepackage{mathtools}
\usepackage{amssymb}
\usepackage{amsfonts}
\usepackage{amsthm}
\usepackage{dsfont}
\usepackage{tikz-cd}
\usetikzlibrary{babel}
\hyphenation{wo-raus}

\usepackage{enumitem}

\usepackage[colorlinks=true, linkcolor=blue]{hyperref}
\hypersetup{
  pdfauthor={Gesina Schwalbe},
  pdftitle={Néron Modelle}
}

% DEFINITIONS
\newtheorem{Satz}{Satz}[chapter]
\newtheorem{Theorem}[Satz]{Theorem}
\newtheorem{Proposition}[Satz]{Proposition}
\newtheorem{Lemma}[Satz]{Lemma}
\newtheorem{Korollar}[Satz]{Korollar}
\theoremstyle{definition}
\newtheorem{Definition}[Satz]{Definition}
\newtheorem{LemmaDefinition}[Satz]{Lemma/Definition}
\newtheorem{Beispiel}[Satz]{Beispiel}
\theoremstyle{remark}
\newtheorem{Bemerkung}[Satz]{Bemerkung}

\newcommand*{\N}{\mathds{N}}
\newcommand*{\Z}{\mathds{Z}}
\newcommand*{\K}{\ensuremath{K}} % Körper
\newcommand*{\algK}{\ensuremath{\overline K}} % algebraischer Abschluss
\DeclareMathOperator{\Quot}{Quot} % Quotientenkörperoperator
\DeclareMathOperator{\Spec}{Spec} % Spektrum
\newcommand*{\A}{\ensuremath{\mathds{A}}} % affiner Raum
\renewcommand*{\P}{\ensuremath{\mathds{P}}} % projektiver Raum
\renewcommand*{\m}{\mathfrak{m}} % Maximalideal
\renewcommand*{\k}{\kappa} % Restklassenkörperoperator
\newcommand*{\n}{\eta} % generischer Punkt
\renewcommand*{\O}{\mathcal{O}} % Garbe
\newcommand*{\I}{\mathcal{I}} % Idealgarbe
\newcommand*{\Om}[1]{\ensuremath{\Omega_{#1}^1}} % relative 1-Formen
\renewcommand*{\d}{\ensuremath{\operatorname{d}}} % (universelles) Differential
\DeclareMathOperator{\rk}{rk} % Rang (einer Matrix)
\newcommand*{\longto}{\longrightarrow}
\newcommand*{\longfrom}{\longleftarrow}
%\newcommand*{\Xn}{\underline{X}} % Variablenmenge
\newcommand*{\degs}{\operatorname{\deg}_s} % Separabilitätsgrad
\newcommand*{\degi}{\operatorname{\deg}_i} % Inseparabilitätsgrad
\DeclareMathOperator{\ord}{ord} % diskrete Bewertung
\DeclareMathOperator{\Div}{Div} % Divisorengruppe
\DeclareMathOperator{\Pic}{Pic} % Picard Gruppe
\renewcommand{\div}{\operatorname{div}}
\DeclareMathOperator{\Char}{char} % Charakteristik
\DeclareMathOperator{\Hom}{Hom} % Homomorphismenmenge
\DeclareMathOperator{\End}{End} % Endomorphismenmenge
\DeclareMathOperator{\Aut}{Aut} % Automorphismenmenge
\newcommand{\Id}{\mathrm{id}} % Identitätsabbildung
\DeclareMathOperator{\im}{im} % image
%\newcommand{\tlambda}{\tilde\lambda} % alternative Abbildung \lambda
%\newcommand{\F}{F} % endliche Untergruppe
\DeclareMathOperator{\genus}{genus} % Kurvengeschlecht

% TITEL
\titlehead{Seminar Elliptische Kurven \\
  von Prof. Kerz,
  SS2016, Universität Regensburg}
\subject{Bachelorarbeit}
\title{Néron-Modelle Elliptischer Kurven 
  und Entwicklung solcher aus dem Weierstrass-Modell}
\author{Gesina Schwalbe}

\begin{document}
$\m$\\
$\k$
\maketitle
\tableofcontents

\chapter{Motivation}

\chapter{Gruppenschemata}
% Abelsche Varietät X/K: eigentliches, glattes, irreduzibles K-Gruppenschema
\begin{Definition}[$S$-Gruppenschema]
  Sei $S$ ein Schema. Ein \emph{Gruppenschema über $S$} ist ein
  $S$-Schema $\pi\colon G\to S$ zusammen mit $S$-Morphismen
  \begin{align*}
    \mu\colon G\times_S G &\longto G
    &&\text{(Gruppenverknüpfung)}\\
    \sigma_0\colon S&\longto G 
    &&\text{(Neutrales Element (Schnitt von $p$))}\\
    i\colon G &\longto G    
    &&\text{(Inverses)}
  \end{align*}
  für die folgende Diagramme kommutieren
  \begin{description}[labelwidth=4cm]
  \item[Neutrales Element]
    % \begin{gather*}
    %   \Id = \mu\circ((\sigma_0\circ\pi)\times\Id) =
    %   \mu\circ(\Id\times(\sigma_0\circ\pi))
    % \end{gather*}
    % d.\,h. es kommutieren
    % \begin{center}
    \begin{tikzcd}
     G \arrow[r, "{(\pi,\Id)}"] \arrow[d, "{\Id}"]
     & S\times_S G \arrow[d, "{\sigma_0\times\Id}"] \\
     G \arrow[from=r, "\mu"]
     & G\times_S G
    \end{tikzcd}
    und
    \begin{tikzcd}
      G \arrow[r, "{(\Id,\pi)}"] \arrow[d, "\Id"]
      & G\times_S S \arrow[d, "{\sigma_0\times\Id}"] \\
      G \arrow[from=r, "\mu"]
      & G\times_S G
    \end{tikzcd}
    % \end{center}
  \item[Inverses]
    % \begin{gather*}
    %   \sigma_0\circ\pi = \mu\circ(i,\Id)=\mu\circ(\Id,i)
    % \end{gather*}
    % d.\,h. es kommutieren
    % \begin{center}
    \begin{tikzcd}
     G \arrow[r, "{(i,\Id)}"] \arrow[d, "{\pi}"]
     & G\times_S G \arrow[d, "{\mu}"] \\
     S \arrow[r, "\sigma_0"]
     & G
    \end{tikzcd}
    und
    \begin{tikzcd}
     G \arrow[r, "{(\Id,i)}"] \arrow[d, "{\pi}"]
     & G\times_S G \arrow[d, "{\mu}"] \\
     S \arrow[r, "\sigma_0"]
     & G
    \end{tikzcd}
    % \end{center}
  \item[Assoziativität]
    % \begin{gather*}
    %   \mu\circ(\Id\circ\mu)=\mu\circ(\mu\circ\Id)
    % \end{gather*}
    % d.\,h. es kommutieren
    % \begin{center}
    \begin{tikzcd}
     G\times_S G\times_S G 
     \arrow[r, "{(\mu,\Id)}"] \arrow[d, "{(\Id,\mu)}"]
     & G\times_S G \arrow[d, "{\mu}"] \\
     G\times_S G \arrow[r, "\mu"]
     & G
    \end{tikzcd}
    % \end{center}
\end{description}
\end{Definition}
\begin{Bemerkung}
  Ein Gruppenschema $X$ ist keine Gruppe auf seinen Punkten, da das
  Faserprodukt topologisch nicht mit dem Produkt der zugrundeliegenden
  topologischen Räume übereinstimmt.
  Dafür sind die Fasern Gruppen und man kann Punkte miteinander
  verknüpfen, sofern sie auf derselben Faser liegen.
\end{Bemerkung}


\appendix
\chapter{Definitionen}
\begin{Definition}[flacher Morphismus]
Ein Morphismus $f\colon X\to Y$ von Schemata ist flach, wenn alle
induzierten Ringhomomorphismen $f_x^\#\colon \O_{S,f(x)}\to\O_{X,x}$  auf
den Halmen flach sind.
\cite[siehe][]{bosch, goodreduction}
\end{Definition}

\begin{Definition}[unverzweigter Morphismus]
  Ein Morphismus $f\colon X\to S$ von Schemata heißt unverzweigt im
  Punkt $x\in X$, falls es eine offene Umgebung
  $U\subset X$ von $x$ gibt und eine geschlossene $S$-Immersion
  $j\colon U\cong\O_W/\I\to W\subset\A_S^n$ in ein offenes Unterschema
  des affinen $n$-Raums mit zugehöriger Idealgarbe $\I\subset\O_W$,
  für die gilt
  \begin{enumerate}[label=(\alph*)]
  \item $\I$ ist in einer Umgebung von $z\coloneqq j(x)$ endlich erzeugt.
  \item $\Om{\A_S^n/S}$ ist erzeugt von den $\d g\in\Om{\A_S^n/S}$
    mit $g\in\I$.
  \end{enumerate}
  $f$ heißt unverzweigt, falls es in jedem Punkt $x\in X$
  unverzweigt ist.
  \begin{Bemerkung}
    Ist $X$ lokal endlich repräsentiert, sind äquivalent
    \begin{enumerate}[label=(\roman*)]
    \item $X$ unverzweigt in $x$,
    \item $\Om{X/S,x}=0$,
    \item $X_{f(x)}\to \Spec(\k(f(x)))$ unverzweigt.
    \end{enumerate}
    \cite[siehe][8.4, Theorem 3]{bosch}
  \end{Bemerkung}
\end{Definition}

\begin{Definition}[glatter Morphismus]
  Ein Morphismus $f\colon X\to S$ von Schemata heißt glatt im Punkt
  $x\in X$ von relativer Dimension $r$, wenn es eine offene Umgebung
  $U\subset X$ von $x$ gibt und eine geschlossene Immersion $j\colon
  U\cong\O_W/\I\to W\subset\A_S^n$ in ein offenes Unterschema des affinen
  $n$-Raums mit zugehöriger Idealgarbe $\I\subset\O_W$, für die
  gilt
  \begin{enumerate}[label=(\alph*)]
  \item $\I$ wird in einer Umgebung von $z\coloneqq j(x)$ von $n-r$ Schnitten
    $g_{r+1},\dotsc,g_{n}$ erzeugt.
  \item Die $\d g_{r+1}(z),\dotsc,\d g_{n}(z)\in 
    \Om{\A_S^n/S,z} \otimes_{\O_{\A_S^n,z}} \k(z)
    \cong \Om{\A_S^n/S,z}/\m_z\Om{\A_S^n/S,z}$ 
    sind linear unabhängig über $\k(z)$.
    Dies ist äquivalent zu
    \begin{gather*}\tag{Jacobi Bedingung}
      \rk\left(
        \frac{\partial g_j}{\partial t_i}\d t_i 
      \right)_{\substack{j=r+1,\dotsc,n\\i=1,\dotsc,n\phantom{+1}}}
      = n-r
    \end{gather*}
  \end{enumerate}
  Ist $f$ glatt in jedem Punkt heißt $f$ glatt bzw. $X$ ist glattes
  $S$-Schema.  
  \cite[siehe][8.5, Definition 1]{bosch}

  $f$ glatt in $x$ ist äquivalent zu $f$ flach in $x$ und $X_{f(x)}$
  glatt über $\k(f(x))$.
  \cite[siehe][8.5, Proposition 17]{bosch}
\end{Definition}

\begin{Definition}[eigentlicher Morphismus]
% Verallgemeinerung von projektiv; Äquivalent zu kompakt
  Ein Morphismus $f\colon X\to Y$ heißt eigentlich, wenn er eine der
  folgenden äquivalenten Bedingungen erfüllt
  \begin{enumerate}[label=(\roman*)]
  \item $f$ ist separiert, von endlichem Typ und universell
    geschlossen (d.\,h. jeder Basiswechsel ist geschlossen)
  \item $f$ ist quasi-separiert, von endlichem Typ und für jeden
    Bewertungsring $R$ gilt
    \begin{gather*}
      (\bullet)\circ\Spec(\operatorname{incl})\colon
      \Hom_Y\left(\Spec(R), X\right) \overset{\sim}{\longto} 
      \Hom_Y\left(\Spec\left(\Quot(R)\right), X\right)\;.
    \end{gather*}
    In anderen Worten, es findet sich für jedes kommutative Diagramm
    wie folgt ein entsprechender, eindeutiger Lift
    \begin{center}
      \begin{tikzcd}
        \Spec(\Quot(R)) \arrow[r]\arrow[d]
        & X \arrow[d, "f"] \\
        \Spec(R) \arrow[r]\arrow[to=ur, dashed, "\exists!"]
        & Y
      \end{tikzcd}
    \end{center}
  \end{enumerate}
  Endliche sowie projektive Morphismen sind eigentlich.
  \cite[siehe][9.5, Remark 5 und Theorem 9]{bosch}

  \begin{Bemerkung}[Stein Faktorisierung]
    Jeder eigentliche Morphismus $f\colon X\to Y$ faktorisiert über
    einen eigentlichen, surjektiven Morphismus $g\colon X\to
    \Spec(f_*\O_X)$ mit zusammenhängenden Fasern und einen endlichen
    Morphismus $h\colon \Spec(f_*\O_X)\to Y$.
    \cite[siehe][9.5, Theorem 12]{bosch}
  \end{Bemerkung}
\end{Definition}

\begin{Definition}[étaler Morphismus]% Äquivalent zu lokal umkehrbar
  Ein Morphismus $f\colon X\to Y$ von $S$-Schemata heißt étale in
  einem Punkt $x\in X$, falls er eine der äquivalenten Bedingungen
  erfüllt 
  \begin{enumerate}[label=(\roman*)]
  \item $f$ glatt von relativer Dimension 0 in $x$
  \item $X$ glatt in $x$, $Y$ glatt in $f(x)$ und 
    $(f^*\Om{Y/S})_x \overset{\sim}{\longto} \Om{X/S,x}$
  \end{enumerate}
  Er heißt étale, falls er étale in jedem Punkt von $X$ ist.
  \cite[siehe][8.5, Definition 1 und Corollary 12]{bosch}  

  $f$ ist genau dann étale, wenn $f$ glatt und unverzweigt ist.
  \cite[siehe][8.5, Proposition 6]{bosch}

  Offene Immersionen sind genau die étalen Immersionen.
  \cite[siehe][8.5, Lemma 7]{bosch}
\end{Definition}

\begin{Definition}[Gute Reduktion]
%$X/S$ hat Gute Reduktion nach dem Primideal $P\in S$ heißt, dass die
%Faser $X_P$ Glattheit erhält.
Sei $(A,\m)$ ein diskreter Bewertungsring, $\K=\Quot(A)$ und $X/\K$
ein eigentliches glattes $\K$-Schema.
Wir sagen $X$ hat \emph{gute Reduktion}, falls es ein eigentliches
glattes $A$-schema $Y$ gibt, so dass $X\cong Y_{\K}$ ($Y_{\K}$
generische Faser), d.\,h. das folgende Diagramm ist kartesisch
\begin{center}
\begin{tikzcd}
  X \arrow[r]\arrow[d]
  \arrow[dr, phantom, "\ulcorner", very near start]
  & Y \arrow[d] \\
  \k(\n) \arrow[r]
  & \Spec(A)
\end{tikzcd}
\end{center}
\cite[siehe][Definition 3.1]{goodreduction}
\end{Definition}



\nocite{*}
\printbibliography
\end{document}
