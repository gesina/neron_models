\documentclass[german]{scrreprt}
\KOMAoptions{parskip=half}

% PACKAGES
% \usepackage{mathastext} % uses text font; bad arrows -> NO
% \usepackage{euler} % for egyptienne
% \usepackage{sfmath}\usepackage{sansmathaccent} % sans serif
% \usepackage[math]{anttor} % letters look slightly crippled in comparison
% \usepackage[math]{kurier}
% \usepackage{mathpazo}
%\usepackage[charter]{mathdesign} % not working :-(

\usepackage{fontspec}
\setmainfont{Latin Modern Roman}
\setsansfont{Latin Modern Sans}
%\setmainfont{Charis SIL}
%\setsansfont{Source Sans Pro}
%\setsansfont{Fira Sans}
%\setsansfont{Bitstream Vera Sans} % too wide

\usepackage{babel}
\usepackage[autostyle=try]{csquotes}

\usepackage[style=alphabetic, backend=biber]{biblatex}
\bibliography{neron_models.bib}

\usepackage{mathtools}
\usepackage{amssymb}
\usepackage{amsfonts}
\usepackage{amsthm}
\usepackage{dsfont}
\usepackage{tikz-cd}
\usetikzlibrary{babel}
\hyphenation{Né-ron=Ab-bil-dungs-ei-gen-schaft Grup-pen-sche-ma-ta
  Treu-flach-heit wo-raus}

\usepackage{enumitem}
\usepackage{booktabs}

\usepackage{hyperref}
\usepackage{tabularx}

% SCHREIBHILFEN
\newcommand*{\optcite}[2][]{\cite[#1]{#2}}

% SONSTIGE DEFINITIONEN
\newcommand*{\NAbbEig}{Néron-Abbildungseigenschaft }

% MATHE DEFINITIONEN
\newtheorem{Satz}{Satz}[chapter]
\newtheorem{Theorem}[Satz]{Theorem}
\newtheorem{Proposition}[Satz]{Proposition}
\newtheorem{Lemma}[Satz]{Lemma}
\newtheorem{Korollar}[Satz]{Korollar}
\theoremstyle{definition}
\newtheorem{Definition}[Satz]{Definition}
\newtheorem{LemmaDefinition}[Satz]{Lemma/Definition}
\newtheorem{Beispiel}[Satz]{Beispiel}
%\theoremstyle{remark}
\newtheorem{Bemerkung}[Satz]{Bemerkung}

\renewcommand*{\epsilon}{\varepsilon} % richtiges Epsilon
\renewcommand*{\c}{\alpha} % Konstanten
\newcommand*{\N}{\mathds{N}} % natürliche Zahlen
\newcommand*{\Z}{\mathds{Z}} % ganze Zahlen
\newcommand*{\Q}{\mathds{Q}} % rationale Zahlen
\renewcommand*{\G}{\mathds{G}} % additives/multiplikatives Gruppenschema
\newcommand*{\K}{\ensuremath{K}} % Körper
\newcommand*{\algK}{\ensuremath{\overline K}} % algebraischer
\newcommand*{\Rh}{\ensuremath{R^{\text{h}}}} % Henselisierung
\renewcommand*{\Rsh}{\ensuremath{R^{\text{sh}}}} % (strikte) Henselisierung
\newcommand*{\Ksh}{\ensuremath{\K^{\text{sh}}}} % (strikte) Henselisierung
\newcommand*{\p}{\ensuremath{\mathfrak{p}}} % Primideal
\renewcommand*{\P}{\ensuremath{\mathfrak{P}}} % Primideal
% Abschluss
\renewcommand*{\Im}{\text{im}} % Bild
\newcommand*{\proj}{\text{proj}} % Projektion
\newcommand*{\dra}{\dashrightarrow} % rationale Abbildung
\DeclareMathOperator{\Dom}{Dom} % Definitionsbereich einer rat. Abb.
\DeclareMathOperator{\Quot}{Quot} % Quotientenkörperoperator
\DeclareMathOperator{\Nil}{Nil} % Nilradikal
\DeclareMathOperator{\Spec}{Spec} % Spektrum
\newcommand*{\A}{\ensuremath{\mathds{A}}} % affiner Raum
\renewcommand*{\P}{\ensuremath{\mathds{P}}} % projektiver Raum
\renewcommand*{\m}{\mathfrak{m}} % Maximalideal
\renewcommand*{\k}{\kappa} % Restklassenkörperoperator
\newcommand*{\n}{\eta} % generischer Punkt
\renewcommand*{\O}{\mathcal{O}} % Garbe
\newcommand*{\I}{\mathcal{I}} % Idealgarbe
\newcommand*{\Om}[1]{\ensuremath{\Omega_{#1}^1}} % relative 1-Formen
\renewcommand*{\d}{\ensuremath{\operatorname{d}}} % (universelles) Differential
\DeclareMathOperator{\rk}{rk} % Rang (einer Matrix)
\newcommand*{\longto}{\longrightarrow}
\newcommand*{\longfrom}{\longleftarrow}
%\newcommand*{\Xn}{\underline{X}} % Variablenmenge
\newcommand*{\degs}{\operatorname{\deg}_s} % Separabilitätsgrad
\newcommand*{\degi}{\operatorname{\deg}_i} % Inseparabilitätsgrad
\DeclareMathOperator{\ord}{ord} % diskrete Bewertung
\DeclareMathOperator{\Div}{Div} % Divisorengruppe
\DeclareMathOperator{\Pic}{Pic} % Picard Gruppe
\renewcommand{\div}{\operatorname{div}}
\DeclareMathOperator{\Char}{char} % Charakteristik
\DeclareMathOperator{\Hom}{Hom} % Homomorphismenmenge
\DeclareMathOperator{\End}{End} % Endomorphismenmenge
\DeclareMathOperator{\Aut}{Aut} % Automorphismenmenge
\newcommand{\Id}{\mathrm{id}} % Identitätsabbildung
\DeclareMathOperator{\im}{im} % image
%\newcommand{\tlambda}{\tilde\lambda} % alternative Abbildung \lambda
%\newcommand{\F}{F} % endliche Untergruppe
\DeclareMathOperator{\genus}{genus} % Kurvengeschlecht
\newcommand*{\Groups}{\text{(Groups)}} % Kategorie der Gruppen
\newcommand*{\Sch}[1][S]{\text{(Sch/#1)}} % Kategorie der #1-Schemata


%%% Local Variables:
%%% mode: latex
%%% TeX-master: "neron_models"
%%% End:


% TITEL
\hypersetup{
  pdfauthor={Gesina Schwalbe},
  pdftitle={Neron Modelle}
}
\titlehead{Betreuer: Prof.\,Dr.\,Kerz\\ Universität Regensburg}
\subject{Bachelorarbeit}
\title{Néron-Modelle Elliptischer Kurven 
  und Entwicklung solcher aus dem Weierstrass-Modell}
\author{Gesina Schwalbe}

\begin{document}

\maketitle
\tableofcontents

%\chapter{Motivation}

\chapter{Gruppenschemata}
\begin{Definition}[$S$-Gruppenschema]
  Sei $S$ ein Schema. Ein \emph{Gruppenschema über $S$} ist ein
  $S$-Schema $\pi\colon G\to S$ zusammen mit $S$-Morphismen
  \begin{align*}
    \mu\colon G\times_S G &\longto G
    &&\text{(Gruppenverknüpfung)}\\
    \sigma_0\colon S&\longto G 
    &&\text{(Neutrales Element (Schnitt von $p$))}\\
    i\colon G &\longto G    
    &&\text{(Inverses)}
  \end{align*}
  für die folgende Diagramme kommutieren
  \begin{description}[labelwidth=4cm]
  \item[Neutrales Element]
    % \begin{gather*}
    %   \Id = \mu\circ((\sigma_0\circ\pi)\times\Id) =
    %   \mu\circ(\Id\times(\sigma_0\circ\pi))
    % \end{gather*}
    % d.\,h. es kommutieren
    % \begin{center}
    \begin{tikzcd}
     G \arrow[r, "{(\pi,\Id)}"] \arrow[d, "{\Id}"]
     & S\times_S G \arrow[d, "{\sigma_0\times\Id}"] \\
     G \arrow[from=r, "\mu"]
     & G\times_S G
    \end{tikzcd}
    und
    \begin{tikzcd}
      G \arrow[r, "{(\Id,\pi)}"] \arrow[d, "\Id"]
      & G\times_S S \arrow[d, "{\sigma_0\times\Id}"] \\
      G \arrow[from=r, "\mu"]
      & G\times_S G
    \end{tikzcd}
    % \end{center}
  \item[Inverses]
    % \begin{gather*}
    %   \sigma_0\circ\pi = \mu\circ(i,\Id)=\mu\circ(\Id,i)
    % \end{gather*}
    % d.\,h. es kommutieren
    % \begin{center}
    \begin{tikzcd}
     G \arrow[r, "{(i,\Id)}"] \arrow[d, "{\pi}"]
     & G\times_S G \arrow[d, "{\mu}"] \\
     S \arrow[r, "\sigma_0"]
     & G
    \end{tikzcd}
    und
    \begin{tikzcd}
     G \arrow[r, "{(\Id,i)}"] \arrow[d, "{\pi}"]
     & G\times_S G \arrow[d, "{\mu}"] \\
     S \arrow[r, "\sigma_0"]
     & G
    \end{tikzcd}
    % \end{center}
  \item[Assoziativität]
    % \begin{gather*}
    %   \mu\circ(\Id\circ\mu)=\mu\circ(\mu\circ\Id)
    % \end{gather*}
    % d.\,h. es kommutieren
    % \begin{center}
    \begin{tikzcd}
     G\times_S G\times_S G 
     \arrow[r, "{(\mu,\Id)}"] \arrow[d, "{(\Id,\mu)}"]
     & G\times_S G \arrow[d, "{\mu}"] \\
     G\times_S G \arrow[r, "\mu"]
     & G
    \end{tikzcd}
    % \end{center}
\end{description}
\begin{Bemerkung}[Gruppe der $T$-wertigen Punkte]
  Diese Definition ist äquivalent dazu, dass die $T$-wertigen Punkte
  $G(T)=\Hom_S(T,G)$ eine Gruppenstruktur tragen, die funktoriell in
  $T$ ist für $S$-Schemata $T$ \cite[Chapter (4.15)]{wedhorn}. 
  In anderen Worten wir erhalten den Funktor 
  \begin{align*}
    \Sch &\longto \Groups\\
    T &\longmapsto G(T)
  \end{align*}
  Man erhält die Gruppenstruktur aus den oben angegebenen Morphismen
  durch
  \begin{align*}
    G(T)\times G(T) &\longto G(T)\\
    (\phi, \psi) &\longmapsto 
                   \left(
                   T\overset{(\phi,\psi)}\longto G\times_S G
                   \overset\mu\longto G
                   \right)
  \end{align*}
  wobei die Identität $T\longto S\overset{\sigma_0}\longto G$ ist und
  man Inverse durch Verknüpfung mit $i$ erhält.
  \cite[Proposition IV.3.2]{silverman2}
\end{Bemerkung}

\begin{Bemerkung}
  Ein Gruppenschema $X$ ist keine Gruppe auf seinen Punkten, da das
  Faserprodukt topologisch nicht mit dem Produkt der zugrundeliegenden
  topologischen Räume übereinstimmt.
  Dafür sind die Fasern Gruppen und man kann Punkte miteinander
  verknüpfen, sofern sie auf derselben Faser liegen.
\end{Bemerkung}

\begin{Bemerkung}\label{gruppenschemaaequivalenzen}
  Für ein $\K$-Gruppenschema endlichen Typs über einem Körper $\K$
  sind glatt und geometrisch reduziert äquivalent.
  \cite[8.5, Excercise 11]{bosch}
  
  In diesem Fall sind auch (geometrisch) irreduzibel und (geometrisch)
  zusammenhängend äquivalent, da aufgrund der Glattheit alle Halme
  lokal und regulär, also insb. integer, sind und $G$ lokal noethersch
  ist.
  \cite[s.][Exercise 3.16]{wedhorn} und \cite[Corollary 16.51]{wedhorn}
\end{Bemerkung}
\end{Definition}


\begin{Definition}[Abelsche Varietät]\label{abelschevarietaet}
Eine abelsche Varietät $X/\K$ ist ein eigentliches, glattes,
irreduzibles $\K$-Gruppenschema.
\cite[9.6, Definition 1]{bosch} oder \cite[Definition 16.52]{wedhorn}

(Hier ist glatt äquivalent zu geometrisch reduziert und irreduzibel
äquivalent zu zusammenhängend, 
siehe \ref{gruppenschemaaequivalenzen}.)

\begin{Bemerkung}
Abelsche Varietäten sind projektiv.
\cite[9.6, Proposition 4]{bosch}
\end{Bemerkung}
\end{Definition}

\begin{Definition}[abelsches Schema]
Ein abelsches $S$-Schema ist ein eigentliches, glattes $S$-Gruppenschema,
dessen Fasern abelsche Varietäten sind 
(also zusätzlich irreduzibel bzw. zusammenhängend,
s. \ref{abelschevarietaet} und \ref{gruppenschemaaequivalenzen},
nachdem eigentlich und glatt stabil unter Basiswechsel sind).
\end{Definition}

\chapter{$S$-Modelle}
\begin{Definition}[$S$-Modell]
Sei $S$ ein zusammenhängendes Dedekindschema mit (eindeutigem)
generischen Punkt $s$ und Funktionenkörper $\K=\k(s)$.
Sei $X$ ein $\K$-Schema.
Ein $S$-Schema $Y$ heißt $S$-Modell von $X$, falls
$Y\times_S\Spec(\K)=Y_K\cong X$ für die generische Faser gilt.
\end{Definition}

\begin{Definition}[arithmetische Fläche]
  Sei $R$ Dedekindring mit Quotientenkörper $\K$.
  Eine arithmetische Fläche $X$ über $R$ ist ein
  integrales, normales, exzellentes Schema, flach und von endlichem
  Typ über $R$,
  dessen generische Faser eine reguläre, zusammenhängende,
  projektive Kurve $X_\K/\K$ ist
  und dessen spezielle Fasern Vereinigungen von Kurven über den
  entsprechenden Restklassenkörpern sind.
  \cite[IV.4]{silverman2}
  
  $X$ ist also ein $X_\K$-Modell.
\end{Definition}

\begin{Satz}
  Sei $R$ Dedekindring mit Quotientenkörper $\K$,
  $X/R$ arithmetische Fläche, $X_\K/\K$ generische Faser.
  Dann gilt
  \begin{enumerate}[label=(\roman*)]
  \item Ist $X/R$ eigentlich, so gilt $X(R)=X_\K(\K)$.
  \item Ist $X/R$ regulär, so gilt $X(R)=X^0(R)$ für $X^0/R$ größtes
    glattes $R$-Unterschema von $X$.
  \end{enumerate}
  \cite[Corollary IV.4.4]{silverman2}

  Für ein reguläres, eigentliches $R$-Modell von $X_K$, das eine
  arithmetische Oberfläche ist (integral, normal, exzellent, flach,
  von endlichem Typ), gilt also
  \begin{gather*}
    X_\K(\K)=X^0(R)=X(R)
  \end{gather*}
\end{Satz}

\begin{Satz}
  Sei $R$ Dedekindring mit Quotientenkörper $\K$,
  $C/\K$ reguläre, projektive Kurve vom Geschlecht $g$.
  \begin{enumerate}[label=(\roman*)]
  \item Dann existiert ein eigentliches, reguläres $R$-Modell $X$ von $C$,
    das eine arithmetische Fläche darstellt
    (kurz: eigentliches, reguläres Modell für $C/\K$).
    \cite[Proposition IV.4.5(a)]{silverman2}
  \item Ist $g\geq1$ kann dieses minimal gewählt werden, d.\,h.
    für jedes andere eigentliche, reguläre Modell für $C/\K$ und jeden
    Isomorphismus der generischen Fasern ist die induzierte birationale
    Abbildung ein $R$-Isomorphismus.
    \cite[Proposition IV.4.5(b)]{silverman2}
  \item Für ein minimales, eigentliches, reguläres Modell $X$ für $C/\K$
    gilt, dass jeder $\K$-Automorphismus von $C$ sich zu einem
    $R$-Morphismus von $X$ erweitert, der glatte Punkte auf glatte
    Punkte abbildet.
    \cite[Proposition IV.4.6]{silverman2}
  \end{enumerate}
\end{Satz}

\chapter{Néron Modelle}
Sei $R$ im Folgenden Dedekindring mit Quotientenkörper $\K$.

\begin{Definition}[Néron Modell]
  Sei $X_\K$ ein glattes, separiertes $\K$-Schema endlichen Typs.

  Ein \emph{Néron Modell} von $X_\K$ ist ein glattes, separiertes
  $R$-Modell $X/R$ endlichen Typs mit der
  sog. Néron-Abbildungseigenschaft:
  \begin{quote}
    Für jedes weitere glatte $R$-Schema $Y$ wird jeder
    $\K$-Morphismus $Y_\K\to X_\K$ auf den generischen Fasern durch
    einen eindeutigen $R$-Morphismus $Y\to X$ erweitert.
  \end{quote}
  \cite[1.2, Definition 1]{neron}

  Spezialfall:
  Sei $X_\K$ eine elliptische Kurve, d.\,h. ein
  glattes, projektives (also separiertes), integrales, reguläres
  $\K$-Schema endlichen Typs vom Genus 1.
  
  Dann ist ein \emph{Néron Modell} von $X_\K$ ist ein glattes
  $R$-Modell und Gruppenschema $X/R$ mit der
  Néron-Abbildungseigenschaft:
  \begin{quote}
    Für jedes weitere glatte $R$-Modell und Gruppenschema $Y$ wird jeder
    $\K$-Morphismus $Y_\K\to X_\K$ auf den generischen Fasern durch
    einen eindeutigen $R$-Morphismus $Y\to X$ erweitert.
  \end{quote}
  \cite[Chapter IV.5]{silverman2}

\end{Definition}

\appendix
\chapter{Definitionen}
\begin{Definition}[Dedekindring]
Ein Integritätsring $R$ heißt Dedekindring, wenn er eine der
äquivalenten Bedingungen erfüllt
\begin{enumerate}[label=(\roman*)]
\item regulär und $\dim(A)\leq 1$
\item noethersch, normal und $\dim(A)\leq 1$
\item $A$ Körper oder
  jede Lokalisierung nach einem Maximalideal ist ein diskreter
  Bewertungsring
  (d.\,h. lokal, noethersch, normal von Dimension 1)
\item jedes echte Ideal ist endliches Produkt von Primidealen
  (diese ist dann eindeutig bis auf Reihenfolge) 
\end{enumerate}
\cite[S.\,40]{hartshorne} und \cite[Definition B.84]{wedhorn}

Faktorielle Dedekindringe sind genau die Hauptidealringe.
\cite[Proposition B.85]{wedhorn}
\end{Definition}

\begin{Definition}[Dedekindschema]
Ein Dedekindschema ist ein noethersches, integrales Schema $X$, das
eine der äquivalenten Eigenschaften erfüllt
\begin{enumerate}[label=(\roman*)]
\item die offenen affinen Unterschemata sind Spektren von Dedekindringen
\item $\dim(X)\leq 1$ und $X$ regulär
  (in Dimension 1 äquivalent zu normal
  \cite[Corollary 6.39, Proposition 6.40]{wedhorn})
\end{enumerate}
Ein integrales Schema mit endlicher Überdeckung durch Dedekindschemata
ist selbst Dedekindschema.
\cite[Chapter (7.13)]{wedhorn}

\begin{Bemerkung}
Die Bedingung, dass $S$ integral sein muss, liefert einen eindeutigen
generischen Punkt $\eta$ und einen Funktionenkörper
$\K=\k(\eta)=\O_{S,\eta}$, so dass jeder Schnitt und jeder Halm ein
Unterring von $\K$ ist, d.\,h. es gibt für jeden Punkt $x\in X$ einen
Morphismus $\Spec(\K)\to\Spec(\k(x))$ und für jede offene Teilmenge
$U\subset X$ einen Morphismus $\Spec(\K)\to\Spec(\Gamma(U,\O_X))$.

Verzichtet man auf diese Bedingung, erhält man eine Familie
generischer Punkte $\eta_i$ der endlich vielen irreduziblen
Komponenten (beachte: $S$ noethersch). Dieser Fall kann analog
behandelt werden, indem man $\K=\bigoplus_i \k(\eta_i)$ setzt.
\end{Bemerkung}

\begin{Bemerkung}[generische und spezielle Fasern]
  Ist $S=\Spec(R)$ mit $R$ Dedekindring, so hat ein $S$-Schema die
  \emph{generische Faser} über dem generische Punkt $\eta=(0)$ und die
  \emph{speziellen Fasern} über den verbleibenden Maximalidealen
  (geschlossenen Punkten).
\end{Bemerkung}
\end{Definition}

\begin{Definition}[reduziert, geometrisch reduziert]
  Ein Schema heißt reduziert, falls alle Halme reduzierte Ringe sind
  (d.\,h. $\Nil(0)=(0)$).

  Ein $\K$-Schema heißt geometrisch reduziert, wenn für jeden
  Basiswechsel zu einer Körpererweiterung von $\K$ die generische
  Faser reduziert ist. 
\end{Definition}

\begin{Definition}[lokal von endlichem Typ]
Ein Morphismus $f\colon X\to Y$ von Schemata heißt lokal von endlichem
Typ, wenn lokal auf affinen offenen Teilmengen $V\subset Y$ die
induzierte Abbildung $f_U^\#: f_\ast\O_X(V)\to \O_Y(V)$
(wobei $U=f^{-1}(V)$) $f_\ast\O_X(V)=\O_X(U)$ zu einer \emph{endlich erzeugten}
$\O_Y(V)$-Algebra macht.

Er heißt von endlichem Typ, wenn er lokal von endlichem Typ und
quasikompakt (Urbilder quasikompakter Mengen sind quasikompakt) ist.
\end{Definition}

\begin{Definition}[Varietät]
Eine affine Varietät über einem Körper $\K$ ist eine Unterprägarbe der Prägarbe 
$U\mapsto \Hom(U,\K)$ über den geschlossenen Punkten eines
geschlossenen, irreduziblen Unterschemas des $\A_\K^n$.
Eine affine Prävarietät ist eine quasikompakte, zusammenhängende
Prägarbe, die lokal affine Varietät ist.

Für integrale $\K$-Schemata endlichen Typs sind die geschlossenen
Punkte isomorph zu den $\K$-Schnitten 
$\Hom_{\Spec(\K)}(\Spec(\K),X)$.
Dadurch gibt es eine Kategorienäquivalenz
\begin{align*}
  \left\{\text{Integrale $\K$-Schemata endlichen Typs}\right\}
  &\longto
  \left\{\text{Prävarietäten über $\K$}\right\}\\
  (X,\O_X) 
  &\longmapsto
    \left( X(\K), \O_{X(\K)}\subset \Hom(\bullet, \K) \right)
\end{align*}
wobei $f\in\O_{X(\K)}(U)$ der Abbildung 
\begin{align*}
  U\cap X(\K) &\longto \K\\
  x  &\longmapsto f(x)
  \coloneqq \bar{f_x}\in\O_{X,x}/\m_x\cong \K
\end{align*}
entspricht (Analogon zum Einsetzhomomorphismus für Polynomringe).
Der inverse Funktor ist gegeben durch
\begin{align*}
  (P,\O_P) \longmapsto (t(P), t_\ast\O_P) 
\end{align*}
wobei $t\colon P\to t(P)$, $x\mapsto \{x\}$. Auf affinen Varietäten
$V$ wird dies zu
\begin{gather*}
  (V, \O_V) \mapsto (\Spec(\O_V(V)), \O_{\Spec(\O_V(V))})
\end{gather*}
\cite[Theorem 3.37]{wedhorn}

Daher werden wir im folgenden mit Prävarietät ein integrales
$\K$-Schema endlichen Typs meinen und mit affiner Varietät ein affines,
integrales $\K$-Schema endlichen Typs.

Eine algebraische Varietät sei im folgenden ein separiertes, integrales
Schema über einem algebraisch abgeschlossen Körper $\K$
(alternativ: über einem Körper $\K$, wobei die Basiserweiterung auf
den algebraischen Abschluss weiterhin integral ist).
\end{Definition}

\begin{Definition}[Kurve, elliptische Kurve]
  Eine Kurve über einem Körper $\K$ ist eine projektive Varietät der
  Dimension 1,
  also ein projektives, integrales $\K$-Schema endlichen Typs.

  Eine elliptische Kurve über $\K$ ist eine reguläre Kurve vom Genus 1.
  Elliptische Kurven sind glatt \cite[Proposition III.3.1]{silverman}.
\end{Definition}

\begin{Definition}[Regulär]
Ein Schema $X$ heißt regulär im Punk $x\in X$, wenn sein Halm
$\O_{X,x}$ ein regulärer (lokaler) Ring ist
(d.\,h. $\dim_{\O_{X,x}/\m_x}(\m_x/\m_x^2)=\dim(\O_{X,x})$).
\end{Definition}

\begin{Definition}[flacher Morphismus]
Ein Morphismus $f\colon X\to Y$ von Schemata ist flach, wenn alle
induzierten Ringhomomorphismen $f_x^\#\colon \O_{S,f(x)}\to\O_{X,x}$  auf
den Halmen flach sind.
\cite[]{bosch, goodreduction}
\end{Definition}

\begin{Definition}[unverzweigter Morphismus]
  Ein Morphismus $f\colon X\to S$ von Schemata heißt unverzweigt im
  Punkt $x\in X$, falls es eine offene Umgebung
  $U\subset X$ von $x$ gibt und eine geschlossene $S$-Immersion
  $j\colon U\cong\O_W/\I\to W\subset\A_S^n$ in ein offenes Unterschema
  des affinen $n$-Raums mit zugehöriger Idealgarbe $\I\subset\O_W$,
  für die gilt
  \begin{enumerate}[label=(\alph*)]
  \item $\I$ ist in einer Umgebung von $z\coloneqq j(x)$ endlich erzeugt.
  \item $\Om{\A_S^n/S}$ ist erzeugt von den $\d g\in\Om{\A_S^n/S}$
    mit $g\in\I$.
  \end{enumerate}
  $f$ heißt unverzweigt, falls es in jedem Punkt $x\in X$
  unverzweigt ist.
  \begin{Bemerkung}
    Ist $X$ lokal endlich repräsentiert, sind äquivalent
    \begin{enumerate}[label=(\roman*)]
    \item $X$ unverzweigt in $x$,
    \item $\Om{X/S,x}=0$,
    \item $X_{f(x)}\to \Spec(\k(f(x)))$ unverzweigt.
    \end{enumerate}
    \cite[8.4, Theorem 3]{bosch}
  \end{Bemerkung}
\end{Definition}

\begin{Definition}[glatter Morphismus]
  Ein Morphismus $f\colon X\to S$ von Schemata heißt glatt im Punkt
  $x\in X$ von relativer Dimension $r$, wenn es eine offene Umgebung
  $U\subset X$ von $x$ gibt und eine geschlossene Immersion $j\colon
  U\cong\O_W/\I\to W\subset\A_S^n$ in ein offenes Unterschema des affinen
  $n$-Raums mit zugehöriger Idealgarbe $\I\subset\O_W$, für die
  gilt
  \begin{enumerate}[label=(\alph*)]
  \item $\I$ wird in einer Umgebung von $z\coloneqq j(x)$ von $n-r$ Schnitten
    $g_{r+1},\dotsc,g_{n}$ erzeugt.
  \item Die $\d g_{r+1}(z),\dotsc,\d g_{n}(z)\in 
    \Om{\A_S^n/S,z} \otimes_{\O_{\A_S^n,z}} \k(z)
    \cong \Om{\A_S^n/S,z}/\m_z\Om{\A_S^n/S,z}$ 
    sind linear unabhängig über $\k(z)$.
    Dies ist äquivalent zu
    \begin{gather*}\tag{Jacobi Bedingung}
      \rk\left(
        \frac{\partial g_j}{\partial t_i}\d t_i 
      \right)_{\substack{j=r+1,\dotsc,n\\i=1,\dotsc,n\phantom{+1}}}
      = n-r
    \end{gather*}
  \end{enumerate}
  Ist $f$ glatt in jedem Punkt heißt $f$ glatt bzw. $X$ ist glattes
  $S$-Schema.  
  \cite[8.5, Definition 1]{bosch}

  $f$ glatt in $x$ ist äquivalent zu $f$ flach in $x$ und $X_{f(x)}$
  glatt über $\k(f(x))$.
  \cite[8.5, Proposition 17]{bosch}

  \begin{Bemerkung}
    Glatt impliziert reduziert und lokal von endlichem Typ.
  \end{Bemerkung}
\end{Definition}

\begin{Definition}[eigentlicher Morphismus]
% Verallgemeinerung von projektiv; Äquivalent zu kompakt
  Ein Morphismus $f\colon X\to Y$ heißt eigentlich, wenn er eine der
  folgenden äquivalenten Bedingungen erfüllt
  \begin{enumerate}[label=(\roman*)]
  \item $f$ ist separiert, von endlichem Typ und universell
    geschlossen (d.\,h. jeder Basiswechsel ist geschlossen)
  \item $f$ ist quasi-separiert, von endlichem Typ und für jeden
    Bewertungsring $R$ gilt
    \begin{gather*}
      X(R)=\Hom_Y\left(\Spec(R), X\right) 
      \overset{\sim}{\underset{\circ\Spec(\operatorname{incl})}{\longto}}
      \Hom_Y\left(\Spec\left(\Quot(R)\right), X\right)=X(\Quot(R))\;.
    \end{gather*}
    In anderen Worten, es findet sich für jedes kommutative Diagramm
    wie folgt ein entsprechender, eindeutiger Lift
    \begin{center}
      \begin{tikzcd}
        \Spec(\Quot(R)) \arrow[r]\arrow[d]
        & X \arrow[d, "f"] \\
        \Spec(R) \arrow[r]\arrow[to=ur, dashed, "\exists!"]
        & Y
      \end{tikzcd}
    \end{center}
  \end{enumerate}
  Endliche sowie projektive Morphismen sind eigentlich.
  \cite[9.5, Remark 5 und Theorem 9]{bosch}

  \begin{Bemerkung}[Stein Faktorisierung]
    Jeder eigentliche Morphismus $f\colon X\to Y$ faktorisiert über
    einen eigentlichen, surjektiven Morphismus $g\colon X\to
    \Spec(f_*\O_X)$ mit zusammenhängenden Fasern und einen endlichen
    Morphismus $h\colon \Spec(f_*\O_X)\to Y$.
    \cite[9.5, Theorem 12]{bosch}
  \end{Bemerkung}
\end{Definition}

\begin{Definition}[étaler Morphismus]% Äquivalent zu lokal umkehrbar
  Ein Morphismus $f\colon X\to Y$ von $S$-Schemata heißt étale in
  einem Punkt $x\in X$, falls er eine der äquivalenten Bedingungen
  erfüllt 
  \begin{enumerate}[label=(\roman*)]
  \item $f$ glatt von relativer Dimension 0 in $x$
  \item $X$ glatt in $x$, $Y$ glatt in $f(x)$ und 
    $(f^*\Om{Y/S})_x \overset{\sim}{\longto} \Om{X/S,x}$
  \end{enumerate}
  Er heißt étale, falls er étale in jedem Punkt von $X$ ist.
  \cite[8.5, Definition 1 und Corollary 12]{bosch}  

  $f$ ist genau dann étale, wenn $f$ glatt und unverzweigt ist.
  \cite[8.5, Proposition 6]{bosch}

  Offene Immersionen sind genau die étalen Immersionen.
  \cite[8.5, Lemma 7]{bosch}
\end{Definition}

\begin{Definition}[Gute Reduktion]
%$X/S$ hat Gute Reduktion nach dem Primideal $P\in S$ heißt, dass die
%Faser $X_P$ Glattheit erhält.
Sei $(A,\m)$ ein diskreter Bewertungsring, $\K=\Quot(A)$ und $X/\K$
ein eigentliches glattes $\K$-Schema.
Wir sagen $X$ hat \emph{gute Reduktion}, falls es ein eigentliches
glattes $A$-schema $Y$ gibt, so dass $X$ isomorph zur generischen
Faser $Y_{\K}$ von $Y$ ist. D.\,h. das folgende Diagramm ist
kartesisch
\begin{center}
\begin{tikzcd}
  X \arrow[r]\arrow[d]
  \arrow[dr, phantom, "\ulcorner", very near start]
  & Y \arrow[d] \\
  \k(\n) \arrow[r]
  & \Spec(A)
\end{tikzcd}
\end{center}
\cite[Definition 3.1]{goodreduction}
\end{Definition}



\nocite{*}
\printbibliography
\end{document}
