\documentclass[german]{scrreprt}
\KOMAoptions{parskip=half}

% PACKAGES
% \usepackage{mathastext} % uses text font; bad arrows -> NO
% \usepackage{euler} % for egyptienne
% \usepackage{sfmath}\usepackage{sansmathaccent} % sans serif
% \usepackage[math]{anttor} % letters look slightly crippled in comparison
% \usepackage[math]{kurier}
% \usepackage{mathpazo}
%\usepackage[charter]{mathdesign} % not working :-(

\usepackage{fontspec}
\setmainfont{Latin Modern Roman}
\setsansfont{Latin Modern Sans}
%\setmainfont{Charis SIL}
%\setsansfont{Source Sans Pro}
%\setsansfont{Fira Sans}
%\setsansfont{Bitstream Vera Sans} % too wide

\usepackage{babel}
\usepackage[autostyle=try]{csquotes}

\usepackage[style=alphabetic, backend=biber]{biblatex}
\bibliography{neron_models.bib}

\usepackage{mathtools}
\usepackage{amssymb}
\usepackage{amsfonts}
\usepackage{amsthm}
\usepackage{dsfont}
\usepackage{tikz-cd}
\usetikzlibrary{babel}
\hyphenation{Né-ron=Ab-bil-dungs-ei-gen-schaft Grup-pen-sche-ma-ta
  Treu-flach-heit wo-raus}

\usepackage{enumitem}
\usepackage{booktabs}

\usepackage{hyperref}
\usepackage{tabularx}

% SCHREIBHILFEN
\newcommand*{\optcite}[2][]{\cite[#1]{#2}}

% SONSTIGE DEFINITIONEN
\newcommand*{\NAbbEig}{Néron-Abbildungseigenschaft }

% MATHE DEFINITIONEN
\newtheorem{Satz}{Satz}[chapter]
\newtheorem{Theorem}[Satz]{Theorem}
\newtheorem{Proposition}[Satz]{Proposition}
\newtheorem{Lemma}[Satz]{Lemma}
\newtheorem{Korollar}[Satz]{Korollar}
\theoremstyle{definition}
\newtheorem{Definition}[Satz]{Definition}
\newtheorem{LemmaDefinition}[Satz]{Lemma/Definition}
\newtheorem{Beispiel}[Satz]{Beispiel}
%\theoremstyle{remark}
\newtheorem{Bemerkung}[Satz]{Bemerkung}

\renewcommand*{\epsilon}{\varepsilon} % richtiges Epsilon
\renewcommand*{\c}{\alpha} % Konstanten
\newcommand*{\N}{\mathds{N}} % natürliche Zahlen
\newcommand*{\Z}{\mathds{Z}} % ganze Zahlen
\newcommand*{\Q}{\mathds{Q}} % rationale Zahlen
\renewcommand*{\G}{\mathds{G}} % additives/multiplikatives Gruppenschema
\newcommand*{\K}{\ensuremath{K}} % Körper
\newcommand*{\algK}{\ensuremath{\overline K}} % algebraischer
\newcommand*{\Rh}{\ensuremath{R^{\text{h}}}} % Henselisierung
\renewcommand*{\Rsh}{\ensuremath{R^{\text{sh}}}} % (strikte) Henselisierung
\newcommand*{\Ksh}{\ensuremath{\K^{\text{sh}}}} % (strikte) Henselisierung
\newcommand*{\p}{\ensuremath{\mathfrak{p}}} % Primideal
\renewcommand*{\P}{\ensuremath{\mathfrak{P}}} % Primideal
% Abschluss
\renewcommand*{\Im}{\text{im}} % Bild
\newcommand*{\proj}{\text{proj}} % Projektion
\newcommand*{\dra}{\dashrightarrow} % rationale Abbildung
\DeclareMathOperator{\Dom}{Dom} % Definitionsbereich einer rat. Abb.
\DeclareMathOperator{\Quot}{Quot} % Quotientenkörperoperator
\DeclareMathOperator{\Nil}{Nil} % Nilradikal
\DeclareMathOperator{\Spec}{Spec} % Spektrum
\newcommand*{\A}{\ensuremath{\mathds{A}}} % affiner Raum
\renewcommand*{\P}{\ensuremath{\mathds{P}}} % projektiver Raum
\renewcommand*{\m}{\mathfrak{m}} % Maximalideal
\renewcommand*{\k}{\kappa} % Restklassenkörperoperator
\newcommand*{\n}{\eta} % generischer Punkt
\renewcommand*{\O}{\mathcal{O}} % Garbe
\newcommand*{\I}{\mathcal{I}} % Idealgarbe
\newcommand*{\Om}[1]{\ensuremath{\Omega_{#1}^1}} % relative 1-Formen
\renewcommand*{\d}{\ensuremath{\operatorname{d}}} % (universelles) Differential
\DeclareMathOperator{\rk}{rk} % Rang (einer Matrix)
\newcommand*{\longto}{\longrightarrow}
\newcommand*{\longfrom}{\longleftarrow}
%\newcommand*{\Xn}{\underline{X}} % Variablenmenge
\newcommand*{\degs}{\operatorname{\deg}_s} % Separabilitätsgrad
\newcommand*{\degi}{\operatorname{\deg}_i} % Inseparabilitätsgrad
\DeclareMathOperator{\ord}{ord} % diskrete Bewertung
\DeclareMathOperator{\Div}{Div} % Divisorengruppe
\DeclareMathOperator{\Pic}{Pic} % Picard Gruppe
\renewcommand{\div}{\operatorname{div}}
\DeclareMathOperator{\Char}{char} % Charakteristik
\DeclareMathOperator{\Hom}{Hom} % Homomorphismenmenge
\DeclareMathOperator{\End}{End} % Endomorphismenmenge
\DeclareMathOperator{\Aut}{Aut} % Automorphismenmenge
\newcommand{\Id}{\mathrm{id}} % Identitätsabbildung
\DeclareMathOperator{\im}{im} % image
%\newcommand{\tlambda}{\tilde\lambda} % alternative Abbildung \lambda
%\newcommand{\F}{F} % endliche Untergruppe
\DeclareMathOperator{\genus}{genus} % Kurvengeschlecht
\newcommand*{\Groups}{\text{(Groups)}} % Kategorie der Gruppen
\newcommand*{\Sch}[1][S]{\text{(Sch/#1)}} % Kategorie der #1-Schemata


%%% Local Variables:
%%% mode: latex
%%% TeX-master: "neron_models"
%%% End:

\hyphenation{Né-ron=Ab-bil-dungs-ei-gen-schaft Grup-pen-sche-ma-ta}

% TITEL
\hypersetup{
  pdfauthor={Gesina Schwalbe},
  pdftitle={Neron Modelle}
}
\titlehead{Betreuer: Prof.\,Dr.\,Kerz\\ Universität Regensburg}
\subject{Bachelorarbeit}
\title{
  Néron-Modelle Elliptischer Kurven 
  % und Entwicklung solcher aus dem Weierstrass-Modell
}
\author{Gesina Schwalbe}

\begin{document}

\maketitle
\tableofcontents

\chapter{Motivation}
Wir werden nur elliptische Kurven über Körpern der Charakteristik
ungleich 2, 3 betrachten und die Restklassenkörper der Dedekindringe
seien perfekt.

Für Ringe $R$ wird $R$ stellenweise als Kürzel für das zugehörige
affine Schema $\Spec(R)$ verwendet. Die Bedeutung ist aus dem Kontext
zu erschließen (z.\,B. ist mit $X\times_R \K$ das Faserprodukt
$X\times_{\Spec(R)}\Spec(\K)$ gemeint).

\begin{Definition}[$S$-Modell]
  Sei $S$ ein zusammenhängendes Dedekindschema mit (eindeutigem)
  generischen Punkt $s$ und Funktionenkörper $\K=\k(s)$.
  Sei $X$ ein $\K$-Schema.
  Ein $S$-Schema $Y$ heißt $S$-Modell von $X$, falls
  $Y\times_S\Spec(\K)=Y_K\cong X$ für die generische Faser gilt.
\end{Definition}

% ----------

\chapter{Gruppenstrukturen auf Schemata}
\section{Gruppenschemata}
Im Folgenden sei $S$ ein Schema.
\begin{Definition}[$S$"=Gruppenschema]
  Ein \emph{Gruppenschema über $S$} ist ein
  $S$-Schema $\pi\colon G\to S$ zusammen mit $S$-Morphismen
  \begin{align*}
    \mu\colon G\times_S G &\longto G
    &&\text{(Gruppenverknüpfung)}\\
    \epsilon\colon S&\longto G 
    &&\text{(Neutrales Element (Schnitt von $p$))}\\
    i\colon G &\longto G    
    &&\text{(Inverses)}
  \end{align*}
  für die folgende Diagramme kommutieren
  \begin{enumerate}[label=(\alph*)]
  \item
    \begin{tikzcd}
      G\times_S G\times_S G 
      \arrow[r, "{(\mu,\Id)}"] \arrow[d, "{(\Id,\mu)}"]
      & G\times_S G \arrow[d, "{\mu}"] \\
      G\times_S G \arrow[r, "\mu"]
      & G
    \end{tikzcd}
    \hfill(Assoziativität)
  \item
    \begin{tikzcd}
      G \arrow[r, "{(\pi,\Id)}"] \arrow[d, "{\Id}"]
      & S\times_S G \arrow[d, "{\epsilon\times\Id}"] \\
      G \arrow[from=r, "\mu"]
      & G\times_S G
    \end{tikzcd}
    und
    \begin{tikzcd}
      G \arrow[r, "{(\Id,\pi)}"] \arrow[d, "\Id"]
      & G\times_S S \arrow[d, "{\epsilon\times\Id}"] \\
      G \arrow[from=r, "\mu"]
      & G\times_S G
    \end{tikzcd}
    \hfill(Neutrales Element)
  \item
    \begin{tikzcd}
      G \arrow[r, "{(i,\Id)}"] \arrow[d, "{\pi}"]
      & G\times_S G \arrow[d, "{\mu}"] \\
      S \arrow[r, "\epsilon"]
      & G
    \end{tikzcd}
    und
    \begin{tikzcd}
      G \arrow[r, "{(\Id,i)}"] \arrow[d, "{\pi}"]
      & G\times_S G \arrow[d, "{\mu}"] \\
      S \arrow[r, "\epsilon"]
      & G
    \end{tikzcd}
    \hfill(Inverses)
  \end{enumerate}
  \begin{Bemerkung}[Gruppe der $T$-wertigen Punkte]
    Diese Definition ist äquivalent dazu, dass die $T$-wertigen Punkte
    $G(T)=\Hom_S(T,G)$ eine Gruppenstruktur tragen, die funktoriell in
    $T$ ist für $S$-Schemata $T$.\optcite[Chapter (4.15)]{wedhorn}
    In anderen Worten wir erhalten den Funktor 
    \begin{align*}
      \Sch &\longto \Groups\\
      T &\longmapsto G(T)
    \end{align*}
    Man erhält eine solche Gruppenstruktur aus den oben angegebenen
    Morphismen durch
    \begin{align*}
      G(T)\times G(T) &\longto G(T)\\
      (\phi, \psi) &\longmapsto 
                     \left(
                     T\overset{(\phi,\psi)}\longto G\times_S G
                     \overset\mu\longto G
                     \right)
    \end{align*}
    wobei die Identität $T\longto S\overset{\epsilon}\longto G$ ist und
    man Inverse durch Verknüpfung mit $i$ erhält.
    \optcite[Proposition IV.3.2]{silverman2}
  \end{Bemerkung}

  \begin{Definition}[Morphismen von Gruppenschemata]
    \optcite[Definition 4.42]{wedhorn}
    Seien $G$ und $H$ $S$"=Gruppenschemata.
    Ein Morphismus $f\colon G\to H$ von $S$-Schemata ist ein Morphismus
    von $S$"=Gruppenschemata, falls die induzierten Abbildungen
    \begin{gather*}
      (\bullet\circ f)\colon G(T)\to H(T)
    \end{gather*}auf den $T$-wertigen Punkten
    Gruppenhomomorphismen sind für alle $S$-Schemata $T$.
    In anderen Worten
    \begin{gather*}
      f\circ\mu_G = \mu_H\circ(f\times f)
    \end{gather*}
  \end{Definition}

% IDENTITÄTSKOMPONENTE
\begin{Definition}[Untergruppenschema]
  \optcite[Definition 4.45]{wedhorn}
  Sei $G$ ein $S$"=Gruppenschema.
  Ein $S$"=Untergruppenschema von $G$ ist ein geschlossenes
  $S$-Unterschema von $G$, dessen geschlossene Immersion ein
  Homomorphismus von Gruppenschemata ist.
\end{Definition}
\begin{Definition}[Identitätskomponente]
  \optcite[Remark IV.6.1.2]{silverman2}
  Sei $G$ ein $S$"=Gruppenschema.
  Die Identitätskomponente einer Faser $G_s$ ($s\in S$) von $G$ ist
  die Zusammenhangskomponente von $G_s$, die $\epsilon(s)$ enthält.
  Die Identitäts-~oder Zusammenhangskomponente von $G$ ist
  die Vereinigung aller Identitätskomponenten der Fasern von $G$.
  Die Identitätskomponente ist ein Untergruppenschema.
\end{Definition}

% ABELSCHE SCHEMATA
\begin{Bemerkung}\label{gruppenschemaaequivalenzen}
    Für ein $\K$"=Gruppenschema endlichen Typs über einem Körper $\K$
    sind glatt und geometrisch reduziert äquivalent.
    \optcite[8.5, Excercise 11]{bosch}
    
    Genauso sind auch (geometrisch) irreduzibel und (geometrisch)
    zusammenhängend äquivalent, da aufgrund der Glattheit alle Halme
    lokal und regulär, also insb. integer, sind und $G$ lokal noethersch
    ist.
    \optcite[Exercise 3.16]{wedhorn}\optcite[Corollary 16.51]{wedhorn}
  \end{Bemerkung}
\end{Definition}
\begin{Definition}[Abelsche Varietät]\label{def:abelschevarietaet}
  Eine abelsche Varietät $X/\K$ ist ein eigentliches, glattes,
  irreduzibles $\K$"=Gruppenschema.
  \optcite[9.6, Definition 1]{bosch}\optcite[Definition 16.52]{wedhorn}

  % (Hier ist glatt äquivalent zu geometrisch reduziert und irreduzibel
  % äquivalent zu zusammenhängend, 
  % siehe \ref{gruppenschemaaequivalenzen}.)

  \begin{Bemerkung}\label{thm:abvarietaetprojektiv}
    Abelsche Varietäten sind projektiv nach
    \cite[9.6, Proposition 4]{bosch}.
  \end{Bemerkung}
\end{Definition}

\begin{Definition}[abelsches Schema]
  Ein abelsches $S$-Schema ist ein eigentliches, glattes $S$"=Gruppenschema,
  dessen Fasern abelsche Varietäten sind 
  (also zusätzlich irreduzibel bzw. zusammenhängend,
  s. \ref{def:abelschevarietaet} und \ref{gruppenschemaaequivalenzen},
  nachdem eigentlich und glatt stabil unter Basiswechsel sind).
\end{Definition}

% S-RATIONALE ABBILDUNG
% ??? Was von rationaler Abbildung wird benötigt?
\begin{Definition}[rationale Abbildung]\label{def:ratabb}
  \optcite[Chapter 2.5]{neron}
  Sei $S$ Schema, und seien $X$ und $Y$ glatte $S$-Schemata.

  Eine \emph{rationale Abbildung} $\phi\colon X\dra Y$ von
  $S$-Schemata ist eine Äquivalenzklasse der Menge
  \begin{gather*}
    \left\{
      \psi\colon U\to Y \text{$S$-Morphismus}
      \;\middle|\;
      U\subset X \text{dicht, offen in $X$}
    \right\}
  \end{gather*}
  wobei $\psi\colon U\to Y$ äquivalent zu $\psi'\colon U'\to Y$ ist,
  falls sie auf einer dichten, offenen Teilmenge von $U\cap U'$
  übereinstimmen.
  
  Ein offenes Unterschema $U\subset X$ heißt \emph{$S$-dicht} in $X$,
  falls für alle $s\in S$ die Faser $U_s$%=U\times_S\k(s)$
  topologisch dicht in der Faser $X_s$%=X\times_S\k(s)$
  liegt.
  Eine \emph{$S$-rationale Abbildung} $\phi\colon X\dra Y$ ist eine
  rationale Abbildung, wobei alle dichten offenen Mengen aus obiger
  Definition $S$-dicht sind.
  % Eine \emph{$S$-rationale Abbildung} $\phi\colon X\dra Y$ ist eine
  % Äquivalenzklasse der Menge
  % \begin{gather*}
  %   \left\{
  %     \psi\colon U\to Y \text{ $S$-Morphismus}
  %     \;\middle|\;
  %     U\subset X \text{ $S$-dicht, offen in $X$}
  %   \right\}
  % \end{gather*}
  % wobei $\psi\colon U\to Y$ äquivalent zu $\psi'\colon U'\to Y$ ist,
  % falls sie auf einer $S$-dichten, offenen Teilmenge von $U\cap U'$
  % übereinstimmen.

  Eine ($S$-)rationale Abbildung $\phi\colon X\dra Y$ heißt definiert
  in $x\in X$, falls $\phi$ von einem Morphismus $\psi\colon U\to Y$
  präsentiert werden kann mit $x\in U$.
  Der \emph{Definitionsbereich $\Dom(\phi)$ von $\phi$} sind die
  Punkte, in denen $\phi$ definiert ist. Er ist offene, ($S$-)dichte
  Teilmenge von $X$. Ist $X/R$ separiert gibt es einen Morphismus
  $\Dom(\phi)\to Y$, der $\phi$ präsentiert.
  
  $\phi\colon X\dra Y$ heißt ($S$-)birational, falls es von einem
  $S$-Morphismus $\psi\colon U\to Y$ präsentiert werden kann, der
  die offene Immersion einer ($S$-)dichten, offenen Teilmenge von $Y$
  ist.
\end{Definition}

Eigentliche Gruppenschemata verhalten sich gutartig bzgl. rationalen
Abbildungen: 
\begin{Lemma}\label{thm:rationalzumorphismus}
  \cite[Proposition IV.6.2]{silverman2}
  Sei $G$ ein $R$"=Gruppenschema, $Y$ ein glattes $R$-Schema und
  $\phi\colon Y\dra G$ eine $R$-rationale Abbildung.
  Dann gilt:
  \begin{enumerate}[label=(\roman*)]
  \item $Y\setminus \Dom(\phi)$ hat (reine) Kodimension 1.
  \item Ist $G$ eigentlich, ist $\phi$ Morphismus.
  \end{enumerate}
\end{Lemma}

% -----

\section{Normale Komposition auf Schemata}
Eine schwächere bzw. rationale Form einer Gruppenschemastruktur:
\begin{Definition}[normale Komposition]\label{def:normalekomposition}
  Sei $R$ Dedekindring, $V$ glattes $R$-Schema mit nichtleeren Fasern.
  Eine \emph{normale Komposition auf $V$} ist eine $R$-rationale
  Abbildung
  \begin{gather*}
    \mu\colon V\times_R V\dra V
  \end{gather*}
  die folgende Bedingungen erfüllt
  \begin{enumerate}[label=(\alph*)]
  \item (Assoziativität) Es gilt
    \begin{gather*}
      \mu(\mu(x,y),z)=\mu(x,\mu(y,z))
    \end{gather*}
    für alle $x,y,z$, so dass der Ausdruck beidseitig definiert ist.
  \item (Inverses)
    Die $R$-rationalen Abbildungen
    \begin{align*}
      \phi\colon V\times_R V
      &\dra V\times_R V
      &\psi\colon V\times_R V
      & \dra V\times_R V\\
      (x,y)
      &\longmapsto (x,\mu(x,y))
      &(x,y)
      &\longmapsto (y,\mu(x,y))       
    \end{align*}
    haben dichtes Bild in jeder Faser und die Einschränkung auf jede
    Faser ist birationaler Isomorphismus.
  \end{enumerate}
\end{Definition}

\begin{Satz}\label{thm:weil}
  \cite[Theorem VIII.1.12]{artin},
  \cite[Theorem IV.6.9]{silverman2}
  Sei $R$ Dedekindring, $V$ glattes $R$-Schema mit nichtleeren Fasern
  und normaler Komposition $\mu\colon V\times_R V\dra V$.
  $V$ sei zusätzlich endlichen Typs und $V(R)$ sei dicht in jeder
  Faser.
  % ??? V(R) dicht in jeder Faser?
  Dann gibt es ein $R$-Gruppenschema $(G,\mu_G)$ endlichen Typs und
  ein gemeinsames, offenes, $R$-dichtes Unterschemata $V\hookleftarrow
  U\hookrightarrow G$ von $V$ und $G$, so dass $\mu|_U=\mu_G|_U$.
  % \begin{proof}
  %   \begin{enumerate}
  %   \item Fall zusammenhängende Fasern:
  %     \begin{enumerate}
  %     \item $X=X_\phi\cap X_\psi\cap Y_\phi\cap Y_\psi$:
  %       $V$ hat normale Komposition $\mu$ mit zugehörigen rationalen
  %       Abbildungen $\phi$ und $\psi$.
  %       Wähle $R$-dichte, offene Teilmengen
  %       $X_\phi,X_\psi,Y_\phi,Y_\psi\subset V\times_R V$,
  %       so dass $\phi\colon X_\phi\overset\sim\to Y_\phi$,
  %       $\psi\colon X_\psi\overset\sim\to Y_\psi$ Isomorphismen sind.
  %       Definiere $X=X_\phi\cap X_\psi\cap Y_\phi\cap Y_\psi\subset
  %       V^2$.
  %       $X$ ist wieder $R$-dicht und $\phi|_X,\psi|_X\colon X\cong X$
  %       sind per Definition Isomorphismen auf $X$.
  %     \item Es gibt eine offene, $R$-dichte Teilmenge $U\subset V$, so
  %       dass die Einschränkung der Projektionen
  %       $\proj_i\colon V\times_R V\to V$ auf $X$ surjektiv sind.
  %     \item Für $V=U$ gilt:
  %       Sei $a\in V$, $x\in V$ in dichter, offener Teilmenge der
  %       Faser, die $a$ enthält.
  %       Dann sind $\phi^{\pm1}, \psi^{\pm1}$ in $(x,a)\in V\times_R V$
  %       definiert.
  %     \item Sei $\Gamma_\mu\subset V^3$ der Graf von $\mu$
  %       ($R$-Schema), $\Gamma\coloneqq \overline{\Gamma_\mu}\subset
  %       V^3$.
  %       Dann sind die Einschränkungen der Projektionen
  %       $\proj_{1,2},\proj_{2,3},\proj_{1,3}\colon \Gamma\to V^2$
  %       offene Immersionen und dicht in jeder Faser von $V^2$.
  %       % ??? Wie kann ein Morphismus dicht in einer Faser sein?
  %     \item Verkleben entlang von $W_s=V\times_R s\times_R V$ für
  %       $s\in V$:
  %       …
  %     \item …
  %     \end{enumerate}
  %   \item Stein-Faktorisierung + „dense set of local sections“?:
  %     % ??? Was ist ein „dense set of local sections“?
  %     Erhalte offenes Unterschema 
  %   \end{enumerate}
  % \end{proof}
\end{Satz}

% ----------


\chapter{Néron-Modelle}
Sei $R$ im Folgenden Dedekindring mit Quotientenkörper $\K$.
\section{Definition}
\begin{Definition}[Néron-Modell]
  \optcite[1.2, Definition 1]{neron}
  Sei $X_\K$ ein glattes, separiertes $\K$-Schema endlichen Typs.

  Ein \emph{Néron-Modell} von $X_\K$ ist ein glattes, separiertes
  $R$-Modell $X/R$ %endlichen Typs
  mit der sog. \emph{\NAbbEig}:
  \begin{quote}
    Für jedes weitere glatte $R$-Schema $Y$ wird jeder
    $\K$-Morphismus $Y_\K\to X_\K$ auf den generischen Fasern durch
    einen eindeutigen $R$-Morphismus $Y\to X$ erweitert.
  \end{quote}
\end{Definition}

% ERWEITERUNG DER GRUPPENSCHEMASTRUKTUR
Néron-Modelle sind besonders interessant für Gruppenschemata, da sich
die Gruppenstruktur auf das Néron-Modell erweitert.
\begin{Bemerkung}\label{thm:gruppenschemaerweiterung}
  \optcite[1.2, Proposition 6]{neron}
  Sei $X$ Néron-Modell über $\Spec(R)$ seiner generischen Faser
  $X_\K$. Ist $X_\K$ ein $\K$"=Gruppenschema, so wird diese durch die
  \NAbbEig eindeutig auf eine
  $S$"=Gruppenschemastruktur auf $X$ erweitert.
  \begin{proof}[Beweisskizze]
    Translationen, das Inverse und der Einheitsschnitt liften wegen
    der \NAbbEig eindeutig und erfüllen die gewünschten Eigenschaften.
  \end{proof}
\end{Bemerkung}

% NÉRON-MODELL K-GRUPPENSCHEMA
\begin{Bemerkung}\label{nerongruppenschemaglatt}
  \optcite[1.2, Remark 7]{neron} 
  Ist $X_\K$ ein $\K$"=Gruppenschema und $X$ ein
  glattes $R$-Modell und Gruppenschema von $X_\K$, das die
  \NAbbEig erfüllt, so ist $X$ ein Néron-Modell von $X_\K$.
  \begin{proof}
    $X_\K$ ist als Faser eines glatten Schemas glatt. Bleibt zu
    zeigen, dass $X_\K$ und $X$ separiert sind.
    
    Ein Gruppenschema $X$ ist genau dann separiert, wenn der
    Einheitsschnitt $\epsilon\colon\Spec(R)\to X$ eine geschlossene
    Immersion ist
    \cite[Lemma 38.6.1]{stacksproject} oder \cite[7.1, Lemma 2]{neron}.
    $\K$-Gruppenschemata sind entsprechend immer separiert
    \cite[Lemma 38.7.3]{stacksproject}. 

    Sei wegen \autoref{thm:neronmodelllokal} \OE{} $R$ diskreter
    Bewertungsring mit Restklassenkörper $k$.
      
    Wir werden zeigen, dass
    $\im(\epsilon)=\overline{\im(\epsilon_\K)}$,
    also dass $\overline{\im(\epsilon_\K)}$ aus genau dem Punkt
    $\epsilon(\Spec\K)$ in der generischen Faser und dem Punkt
    $\epsilon(\Spec k)$ in der speziellen Faser besteht.
    Dann ist $\im(\epsilon)$ geschlossen in $X$ und der Schnitt
    $\epsilon$ demnach geschlossene Immersion.
    
    Auf der generischen Faser gilt
    $\overline{\im(\epsilon_\K)}\cap X_\K=\im(\epsilon_\K)$,
    da $\epsilon_\K$ geschlossene Immersion ist.
    Außerdem stimmen hier $\im(\epsilon)=\{\epsilon_\K(\Spec\K)\}$ und
    $\im(\epsilon)\cap X_\K$ wegen der \NAbbEig punktweise
    überein.

    Auf der speziellen Faser $X_k$ betrachte einen Punkt
    $x\in\overline{\im(\epsilon_\K)}\cap X_k$. Sei $U$ eine
    offene, affine Umgebung von $x$ in $X$ und
    $\Spec(A)=U\cap\overline{\im(\epsilon_\K)}$.
    Dann induziert  ?
    % ??? $\Spec(\K)\overset{\epsilon_\K}\to \Spec(A) \overset{\pi}\to \Spec(R)$
    % TODO!!!
    die Inklusionen $R\subset A\subset\K$.
    Nachdem $R$ als diskreter Bewertungsring der größte Unterring
    von $\K$ ist, folglich $R=A$.
    Wir erhalten einen Schnitt
    $X(R)\ni e\colon\Spec(R)\to\Spec(A)\hookrightarrow X$,
    der $\epsilon_\K\colon\Spec(\K)\to X_\K$ erweitert.
    Dieser muss allerdings wegen der Eindeutigkeit aus der \NAbbEig
    gleich $\epsilon\in X(R)$ sein, also liegt $x$ in $\im(\epsilon)$.
    Nachdem $x\in \overline{\im(\epsilon_\K)}\cap X_k$ beliegig
    gewählt war, kann $\overline{\im(\epsilon_\K)}\cap X_k$ nur den
    Punkt $\epsilon(\Spec k)$ enthalten.

    Die Mengen $\overline{\im(\epsilon_\K)}=\im(\epsilon_\K)$ stimmen
    also überein, insbesondere ist $\im(\epsilon_\K)$ geschlossene
    Immersion.
  \end{proof}
\end{Bemerkung}

% Elliptische Kurven sind abelsche Varietäten und nach
% \autoref{thm:exneronmodellabvarietaet} hat somit jede elliptische
% Kurve ein Néron-Modell. Wir werden uns damit befassen, wie sich diese
% Modelle konkret angeben lassen mithilfe der Weierstraßgleichung der
% Kurve.

\begin{Bemerkung}[Spezialfall elliptische Kurven]
  \label{def:neronmodellellkurve}
  Sei $R$ ein Dedekindring (mit perfektem Restklassenkörper),
  $\K=\Quot(R)$, $E_\K$ eine elliptische Kurve, d.\,h. ein
  glattes, projektives (also separiertes), integeres, reguläres
  $\K$-Schema endlichen Typs vom Geschlecht~1.
  
  Dann ist ein Néron-Modell von $E_\K$ nach
  \ref{thm:gruppenschemaerweiterung} und \ref{nerongruppenschemaglatt}
  ein glattes $R$-Modell und Gruppenschema $E/R$ mit der
  \NAbbEig:
  \begin{quote}
    Für jedes weitere glatte $R$-Schema $Y$ wird jeder
    $\K$-Morphismus $\phi_\K\colon Y_\K\to E_\K$ auf den generischen
    Fasern durch einen eindeutigen $R$-Morphismus $Y\to E$ erweitert.
  \end{quote}
  \optcite[Chapter IV.5]{silverman2}
\end{Bemerkung}
\begin{Bemerkung}\label{thm:dichtefaser}
  Wir können $\phi_\K$ aus \autoref{def:neronmodellellkurve} auch als
  $\K$-rationale Abbildung wählen. Da $E_\K$ projektiv, also eigentlich,
  ist, erweitert sich $\phi_\K$ nach
  \autoref{thm:rationalzumorphismus} zu einem Morphismus.

  Im Fall eines diskreten Bewertungsrings $R$ ist der generische Punkt
  $(0)$ von $\Spec(R)$ offen, da das Komplement
  $\Spec(R)\setminus\{(0)\}$ aus nur endlich viele geschlossenen
  Punkten besteht. Entsprechend ist die generische Faser $Y_\K$ jedes
  $R$-Schemas $Y$ offen.
  Ist $Y$ glatt (also insb. flach und reduziert) über $R$, ist $Y_\K$
  dicht in $Y$ nach \autoref{thm:flachgenerischefaser}.
  Also präsentiert der Morphismus $\phi_\K\colon X_\K\to E_\K$ eine
  rationale Abbildung (vgl. \label{def:ratabb})
  \begin{gather*}
    \phi\colon X\dra E
  \end{gather*}
  Die \NAbbEig bedeutet, dass diese sich zu einem Morphismus $X\to E$
  erweitert.
\end{Bemerkung}


\section{Eigenschaften}
\subsection{Allgemeine Eigenschaften}
% EINDEUTIGKEIT, BASISWECHSEL
\begin{Satz}
  \optcite[1.2, Proposition 2]{neron}
  \optcite[Proposition IV.5.2]{silverman2}
  Sei $X/R$ ein Néron-Modell von $X_\K/\K$. Dann gilt
  \begin{enumerate}[label=(\roman*)]
  \item $X$ ist eindeutig bis auf kanonische Isomorphie.
  \item Für einen étalen Basiswechsel $\Spec(R')\to\Spec(R)$ mit $K'$
    Funktionenkörper von $R'$ ist $X\times_R R'$ Néron-Modell
    seiner generischen Faser $X_{\K'}=X_\K\times_{\K}\K'$,
    d.\,h. étaler Basiswechsel kommutiert mit der Bildung von
    Néron-Modellen.
  \end{enumerate}
  \begin{proof}
    \begin{enumerate}[label=(\roman*)]
    \item Sei $X'$ ein weiteres Néron-Modell von $X_\K$.
      Dann liefert die \NAbbEig zur Identität
      $\Id_{X_\K}\colon X_\K\to X_\K$ eindeutige Morphismen
      $\phi\colon X'\to X$ und $\psi\colon X\to X'$.
      Die Verknüpfungen $\phi\circ\psi\colon X\to X$ und
      $\psi\circ\phi\colon X'\to X'$ sind jeweils die Identität auf
      der generischen Faser $X_\K$, welche nach der Eindeutigkeit in der
      \NAbbEig eindeutig durch die Identitäten auf
      $X$ bzw. $X'$ erweitert wird. Also sind $\phi,\psi$ inverse
      Isomorphismen.
    \item
      Sei $X_{R'}\coloneqq X\times_R R'$ der Basiswechsel
      mit generischer Faser
      $X_{\K'}=X_{R'}\times_{R'}\K'=X_\K\times_\K\K'$.
      $X_{R'}$ ist glatt, separiert und von endlichem Typ 
      (d.\,h. lokal von endlichem Typ und quasikompakt), da alles
      stabil unter Basiswechsel ist. Es ist somit nur die
      \NAbbEig zu prüfen.
      Sei also $Y'$ glattes $R'$-Schema mit generischer Faser
      $Y_{\K'}'$ und $\phi_{\K'}\colon Y_{\K'}'\to X_{\K'}$ ein
      $\K'$-Morphismus. 
      \begin{description}
      \item[Existenz] 
        $Y'$ ist glattes $R$-Schema durch die Komposition
        \begin{gather*}
          Y'\overset{\text{glatt}}\longto\Spec(R')
          \overset{\text{étale}}\longto\Spec(R)
        \end{gather*}
        glatter Abbildungen.
        Außerdem lässt sich der $\K$-Morphismus
        \begin{gather*}
          \phi_{\K} \coloneqq \proj_{X_\K}\circ\phi_{\K'}
          \colon
          Y_{\K'}' \overset{\phi_{\K'}}\longto X_{\K'}
          \overset{\proj_{X_\K}}\longto X_{\K}
        \end{gather*}
        mit der \NAbbEig von $X$ zu einem eindeutigen
        $R$-Morphismus
        \begin{gather*}
          \phi_R\colon Y'\longto X
        \end{gather*}
        erweitern.
        So erhalten wir kanonisch einen eindeutigen $R$-Morphismus
        (bzw. einen $R'$-Morphismus)
        $\phi_{R'} = (\phi_R\;,\;\pi_{R'})$ auf dem Faserprodukt 
        \begin{center}
          \begin{tikzcd}
            Y'
            \arrow[drr, bend left, "\pi_{R'}"]
            \arrow[ddr, bend right, "{\phi_R}"]
            \arrow[dr, dashed, "{\exists!\phi_{R'}}"]\\
            & X\times_{R} R'
            \arrow[dr, phantom, "\ulcorner", very near start]
            \arrow[r, "\proj_{R'}"]\arrow[d, "\proj_X"]
            & \Spec(R')\arrow[d]\\
            & X \arrow[r]
            & \Spec(R)
          \end{tikzcd}
        \end{center}
        $\phi_{R'}$ erweitert $\phi_{\K'}$, denn
        \begin{align*}
          \phi_{R'}|_{Y_{\K'}'}
          &= \left(
            \phi_R|_{Y_{\K'}'}\;,\;
            \pi_{R'}|_{Y_{\K'}'}\colon
            Y_{\K'}'\hookrightarrow Y'\to\Spec(R')
            \right)\\
            % \pi|_{Y_{\K'}'})\colon Y'\times_{R'}\K'\to Y'\to\Spec(R')
            % \overset{\text{Faserprod.}}{=}
            % \pi_{\K'}\colon Y'\times_{R'}\K'\to \Spec(K')\to\Spec(R')
          &\overset{\mathllap{\text{Faserprod.}}}{=} \left(
            \phi_R|_{Y_{\K'}'}\;,\;
            \pi_{\K'}\colon
            Y_{\K'}'\to \Spec(K')\hookrightarrow\Spec(R')
            \right)\\
          &\overset{\mathllap{\text{Def. }\phi_R}}{=} \left(
            \phi_{\K}\;,\; \pi_{\K'}
            \right)\\
          &\overset{\mathllap{\text{Def. }\phi_\K}}{=} \left(
            \proj_{X_\K}\circ\phi_{\K'}\;,\;
            \proj_{\K'}\circ\phi_{\K'}
            \right)\\
          &= \phi_{\K'}
        \end{align*}
      \item[Eindeutigkeit]
        Für jeden weiteren Lift $\psi_{R'}$ von $\phi_{\K'}$ ist
        $\proj_{X}\circ\psi_{R'}$ ein Lift von $\phi_{K}$ und es gilt
        aufgrund der Eindeutigkeit in der \NAbbEig
        $\proj_{X}\circ\psi_{R'}=\phi_{R}$.
        Damit lässt $\psi$ ebenfalls das obige Diagramm kommutieren
        und ist nach der universellen Eigenschaft des Faserprodukts
        bereits gleich $\phi_{R'}$.
      \end{description}
    \end{enumerate}
  \end{proof}
\end{Satz}

% LOKALITÄT
Die Existenz von Néron-Modellen können auch lokal geprüft werden im
folgenden Sinne:
\begin{Satz}\label{thm:neronmodelllokal}
  \cite[1.2, Proposition 4]{neron}
  Für ein $R$-Schema $X$ endlichen Typs ist äquivalent
  \begin{enumerate}[label=(\roman*)]
  \item $X$ ist Néron-Modell seiner generischen Faser
  \item $X\times_R\Spec(\O_{\Spec(R),s})$ ist Néron-Modell
    seiner generischen Faser über $\O_{\Spec(R),s}$ für alle
    geschlossenen Punkte $s\in\Spec(R)$.
  \end{enumerate}
  D.\,h. man kann \OE{} davon ausgehen, dass $R$ lokal (also diskreter
  Bewertungsring) ist.
\end{Satz}

Wir werden im Folgenden nur die Implikation
\emph{(ii)}$\Rightarrow$\emph{(i)} benötigen, daher sei für die
andere Richtung auf \cite[1.2, Proposition 4]{neron} verwiesen.
Wir benötigen dazu folgendes Lemma und Korollar

% ERWEITERUNG V. MORPHISMEN AUF OFFENE TEILMENGEN
\begin{Lemma}\label{thm:morphismuserweiterung}
  % http://math.stackexchange.com/questions/1259876/extending-a-morphism-of-schemes
  \optcite[Exercise 3.2.4]{liu}
  Sei $S$ lokal noethersches Schema, $Y$, $X$ $S$-Schemata, $X$
  endlichen Typs, $y\in Y$, $f_y\colon \Spec\O_{Y,y}\to X$
  $S$-Morphismus.
  Dann gibt es eine offene Umgebung $U\subset Y$ von $y$, auf der
  $f_y$ von einem Morphismus $f\colon U\to X$ erweitert wird.
  \begin{proof}
    Wir können uns auf den affinen Fall $X=\Spec(B)$ beschränken.
    Denn $\O_{Y,y}$ ist lokal, also ist $y=\m_y\in\Spec\O_{Y,y}$ in jeder
    Zariski-geschlossenen, nichtleeren Menge von $\Spec\O_{Y,y}$
    enthalten bzw. die einzige offene Umgebung von $y$ in
    $\Spec\O_{Y,y}$ ist $\Spec\O_{Y,y}$ selbst.
    Somit ist das offene Urbild jeder offenen, affinen Umgebung
    $\Spec(B)\subset X$ von $f_y(y)$ ganz $\Spec\O_{Y,y}$.

    $X$ ist nach Voraussetzung von endlichem Typ.
    Wir können also eine genügend kleine, offene, affine Umgebung
    $U=\Spec(B)\subset X$ von $f_y(y)$ und $V=\Spec(R)\subset S$ auswählen,
    dass $R$ noethersch und $B$ eine $R$-Algebra endlichen Typs (wegen
    noethersch auch endlich präsentiert) ist. D.\,h.
    \begin{gather*}
      B=R[X_0,\dotsc, X_n]/(g_0,\dotsc, g_r)\qquad n,r\in\N
    \end{gather*}
    Wähle weiterhin eine genügend kleine, offene, affine Umgebung
    $\Spec(A)\subset Y$ von $y$ aus, dass $A$ ebenfalls $R$-Algebra ist.
    Es gilt $y=\m_y\in\Spec\O_{Y,y}=\Spec(A_y)\subset\Spec(A)$.
    
    $f_y\colon\Spec(O_{Y,y})=\Spec(A_y)\to\Spec(B)\subset X$
    korrespondiert also zu einem Ringhomomorphismus $f_y^\#\colon B\to
    A_y$.
    Insgesamt erhalten wir
    \begin{align*}
      \phi\colon
      R\left[X_0,\dotsc, X_n\right]
      &\twoheadrightarrow
        \dfrac{R\left[X_0,\dotsc, X_n\right]}{\left(g_0,\dotsc, g_r\right)}
        \overset{f_y^\#}{\longto}
        A_y\\
      X_i &\longmapsto \frac{a_i}{f_i}
            \qquad i=0,\dotsc, n\\
      g_j &\longmapsto \frac{a_j'}{h_j}
            \qquad j=0,\dotsc, r
    \end{align*}
    Dass $\frac{a_j'}{h_j}=f_y^\#(g_j)=0\in A_y$ gilt, ist äquivalent
    dazu, dass es $f_j'$ gibt, so dass $a_j' f_j'=0\in A$.
    Definiere $f\coloneqq \prod_{i=0}^n f_i \prod_{j=0}^r f_j'$.
    Dann gilt für den Morphismus
    \begin{align*}
      \phi_f\colon
      R\left[X_0,\dotsc, X_n\right]
      &\longto
        A_f\\
      X_i
      &\longmapsto
        \dfrac{a_i\cdot \prod_{k\neq i}^n f_k \prod_{j=0}^r f_j'}{f}
        \qquad i=0,\dotsc,n
    \end{align*}
    dass $(g_0, \dotsc, g_r)\subset\ker(\phi')$ und das folgende
    Diagramm kommutiert
    \begin{center}
      \begin{tikzcd}
        R\left[X_0,\dotsc, X_n\right] \arrow[r,"\phi_f"]
        \arrow[d, two heads]
        & A_f\arrow[d]\\
        B \arrow[r, "f_y^\#"{swap}]\arrow[ur,"\bar\phi_f"{near start}]
        & A_y
      \end{tikzcd}
    \end{center}
    Insbesondere faktorisiert $f_y^\#\colon B\to A_f\to A_y$ über
    $A_f$, was heißt, dass $f_y$ über einen Morphismus $f\colon
    D(f)\to\Spec(B)$ faktorisiert:
    \begin{gather*}
      f_y\colon
      \Spec\O_{Y,y}=\Spec(A_y)
      \hookrightarrow \Spec(A_f)=D(f)
      \overset{f}\longto \Spec(B)
    \end{gather*}
    $D(f)$ ist eine offene Umgebung von $y$, womit die Behauptung
    gezeigt ist.
  \end{proof}
\end{Lemma}

\begin{Korollar}\label{thm:allgmorphismuserweiterung}
  Sei $S$ lokal noethersches Schema, $Y$, $X$ $S$-Schemata, $X$
  endlichen Typs, $s\in S$,
  $f_s\colon Y\times_S\O_{S,s}\to X\times_S\O_{S,s}$ $S$-Morphismus.
  Dann gibt es eine offene Umgebung $U\subset Y$ von
  $Y\times_S\O_{S,s}$, auf der $f_s$ von einem Morphismus $f\colon
  U\to X$ erweitert wird.
  \begin{proof}
    Sei $y\in Y\times_S\O_{S,s'}\subset Y\times_S\O_{S,s}\subset Y$
    für ein $s'\in\Spec\O_{S,s}$. Der Morphismus $\pi\colon
    Y\to S$, der $Y$ zum $S$-Schema macht, induziert einen
    Ringhomomorphismus $\O_{S,s'}\to\O_{Y,y}$. Der zugehörige
    Morphismus von $R$-Schemata
    $\Spec\O_{Y,y}\to\Spec\O_{S,s'}\subset\Spec\O_{S,s}$ ist die
    Einschränkung von $\pi$ auf $\Spec\O_{Y,y}\subset Y$, insbesondere
    liegt $\Spec\O_{Y,y}$ im Urbild von $\Spec\O_{S,s}$.
    Das zeigt, dass $\Spec\O_{Y,y}\subset Y\times_S\O_{S,s}$ für jedes
    $y\in Y\times_S\O_{S,s}$, und man erhält $R$-Morphismen
    \begin{gather*}
      f_s(y)\colon \Spec\O_{Y,y}\to X\times_S\O_{S,s}\subset X
      \qquad \forall y \in Y\times_S\O_{S,s}
    \end{gather*}
    Diese erweitern sich jeweils nach
    \autoref{thm:morphismuserweiterung} zu Morphismen auf Umgebungen
    $U(y)\subset Y$ von $y$, welche nach Konstruktion auf den
    Überschneidungen übereinstimmen und sich einem Morphismus
    $f\colon U=\bigcup_{y\in Y\times_S\O_{S,s}} U(y)\to X$ verkleben
    lassen.
  \end{proof}
\end{Korollar}
    % \begin{Lemma}\label{thm:limesoffenemengen}
    %   \cite[1.2, Lemma 5]{neron}
    %   Für ein Basisschema $S$, einen Punkt $s\in S$ und endlich
    %   präsentierte $S$-Schemata $X$, $Y$ ist die kanonische
    %   Abbildung
    %   \begin{gather*}
    %     \varinjlim \Hom_{S'}(X\times_S S', Y\times_S S')
    %     \longto
    %     \Hom_{\O_{R,s}}(X\times_S \O_{R,s}, Y\times_S \O_{R,s})
    %   \end{gather*}
    %   bijektiv, wobei der direkte Limes über die offenen Umgebungen
    %   $S'\subset S$ von $s$ läuft.
    % \end{Lemma}

\begin{proof}[Beweis von \autoref{thm:neronmodelllokal}]
  Für den Beweis werden wir weiterhin die Kurzformen
  $\Spec(\O_{\Spec(R),s})=\O_{R,s}$ verwenden.
    
    Sei $\K=\Quot(R)$ und betrachte ein glattes $R$-Schema $Y$ sowie
    einen $\K$-Morphismus $\phi_\K\colon Y_\K\to X_\K$ der
    generischen Fasern.
    Sei $s\in\Spec(R)$ ein geschlossener Punkt.
    Nach Voraussetzung erweitert sich
    \begin{gather*}
      u_\K\colon
      Y\times_R\O_{R,s}\times_{\O_{R,s}}\K = Y_\K
      \longto
      X_\K = X\times_R\O_{R,s}\times_{\O_{R,s}}\K
    \end{gather*}
    eindeutig zu einem Morphismus
    \begin{gather*}
      u(s)\colon Y\times_R\O_{R,s}\to X\times_R\O_{R,s}\;,
    \end{gather*}
    da $X\times_R\O_{R,s}$ Néron-Modell seiner generischen Faser ist.
    % Nach \autoref{thm:limesoffenemengen} ist $u_s$ die Einschränkung
    % einer eindeutigen Äquivalenzklasse im direkten Limes
    % $\varinjlim \Hom_{S'}(X\times_S S', Y\times_S S')$ über die
    % offenen Umgebungen $S'\subset S$ von $s$.
    % Daher gibt es eine offene Umgebung $S'(s)$ von $s$, über der $u_s$
    % eindeutig auf einen $R$-Morphismus $u(s)\colon S'(s)\to X$
    % erweitert wird.
    % % ??? Warum kann ich eine offene Menge wählen?
    Nach \autoref{thm:allgmorphismuserweiterung} gibt es eine offene
    Umgebung $U(s)$ von $Y\times_R\O_{R,s}$, auf der $u_s$ wiederum
    auf einen Morphismus $u(s)\colon U(s)\to X$ erweitert wird.
    
    Die $u(s)$ stimmen per Konstruktion auf dem Schnitt ihrer
    affinen Definitionsbereiche überein,
    die ganz $Y$ überdecken ($\Spec(R)$ besteht genau aus seinen
    geschlossen Punkten und dem generischen Punkt, und letzterer ist
    in jeder offenen Teilmenge enthalten).
    Daher liefert Verkleben einen von den (eindeutigen) $u_s$ eindeutig
    bestimmten Morphismus $u\colon Y\to X$, der $u_\K$ erweitert.
    D.\,h. die \NAbbEig ist erfüllt.
  \end{proof}


\subsection{Néron-Modelle abelscher Varietäten}
% ALLG. EXISTENZ
% \begin{Satz}[Existenz von Néron-Modellen]\label{thm:existenz}
%   \cite[1.3, Theorem 1]{neron}
%   Sei $\Rsh$ die strikte Henselisierung von $R$ und $\Ksh$ der
%   Quotientenkörper. Ein glattes $\K$"=Gruppenschema $X_\K$ endlichen
%   Typs hat genau dann ein Néron-Modell über $R$, wenn $X_\K(\Ksh)$
%   beschränkt in $X_\K$ ist.
%   % eigentlich => beschränkt
% \end{Satz}
\begin{Korollar}\label{thm:exneronmodellabvarietaet}
  Abelsche Varietäten über $\K$ haben ein Néron-Modell über $R$.
  % \begin{proof}
  %   Abelsche Varietäten über $\K$ sind insbesondere glatte und
  %   eigentliche $\K$"=Gruppenschemata, siehe
  %   \autoref{def:abelschevarietaet}.
  %   Nachdem eigentlich beschränkt impliziert, ist die Bedingung aus
  %   \autoref{thm:existenz} erfüllt.
  % \end{proof}
\end{Korollar}

% Für die folgenden Aussagen ist beim Übergang zum allgemeinen Fall
% eines allgemeinen Dedekindschemas die Eigenschaft zusammenhängend
% nötig.

Abelsche Varietäten stellen einen Spezialfall dar. Hier ist bei guter
Reduktion die Existenz eines besonders schönen Néron-Modells gesichert.
\begin{Satz}\label{thm:abelscheneronmodelle}
  \cite[1.4, Proposition 2]{neron}
  Sei $A_\K$ abelsche Varietät über $\K$ mit guter Reduktion
  d.\,h. es gibt ein glattes, eigentliches $R$-Modell $A$ von $A_\K$.
  Dann ist $A$ Néron-Modell von $A_\K$ und die induzierte
  Gruppenschemastruktur macht $A$ zu einem abelschen Schema.
  \begin{proof}[Beweisskizze]
    Nach \autoref{thm:exneronmodellabvarietaet} existiert ein
    Néron-Modell $X$ von $A_\K$, das nach
    \autoref{thm:gruppenschemaerweiterung} eine Gruppenschemastruktur
    trägt.
    Es bleibt zu zeigen, dass $A\cong X$ ist
    % (d.\,h. $A$ Néron-Modell von $A_\K$)
    mit zusammenhängenden Fasern (dann ist $A$ ist abelsches Schema,
    da $A$ nach Voraussetzung bereits glatt und eigentlich ist).
    \begin{description}
    \item[$A$ isomorph zu $X$] $X$ ist glatt, also nach
      \autoref{thm:implikationenglatt} flaches und lokal endlich
      präsentiertes, $R$-Schema.
      Als solches ist es (universell) offen \cite[Theorem 14.33]{wedhorn}
      und somit zusammenhängend nach \cite[Proposition 3.24]{wedhorn},
      da die generische Faser $A_\K$ zusammenhängend und $\Spec(R)$
      ein integeres, insb. irreduzibles, Schema ist.
      Die \NAbbEig von $X$ liefert zur Identität $\Id_{A_\K}$ einen
      $R$-Morphismus $A\to X$, der bereits offene Immersion ist
      \cite[4.3/1 (ii) oder 4.4/1]{neron}.
      % ??? Wieso ist die Erweiterung von id hier offene Immersion?
      Nachdem $A$ eigentlich ist, ist das Bild dieses Morphismus
      geschlossen in $X$ und, da $X$ zusammenhängend ist, ist er ein
      Isomorphismus (injektiv als offene Immersion und surjektiv, da
      er surjektiv auf der einzigen Zusammenhangskomponente ist).
    \item[$A$ hat zusammenhängende Fasern]
      \OE{} sei $R$ diskreter Bewertungsring nach
      \autoref{thm:neronmodelllokal}.
      Dann ist nur noch die spezielle Faser über dem geschlossenen
      Punkt zu überprüfen, die nach \cite[5.5.1]{EGAIII-1} zusammenhängend
      ist.
      % ??? –> Zariski's Connectedness Thm/Stein-Faktoriesierung?
      % \cite[Exercise 12.29]{wedhorn}
    \end{description}
  \end{proof}
\end{Satz}

Mit diesem Ergebnis können wir für Néron-Modelle von
abelschen Varietäten eine Aussage dazu machen, wo die induzierte
Gruppenschemastruktur abelsch ist:
\begin{Satz}
  \cite[1.4, Theorem 3]{neron}
  Sei $A_\K$ abelsche Varietät über $\K$,
  $X$ ihr Néron-Modell über $R$ und
  $S\subset\Spec(R)$ die Menge der geschlossenen Punkte
  $s\in\Spec(R)$, über denen $A_\K$ gute Reduktion hat (d.\,h. $A_\K$
  hat glattes eigentliches $R_s$-Modell), zusammen mit dem generischen
  Punkt.
  $S$ ist entsprechend offen und dicht in $\Spec(R)$.
  Dann ist $X\times_R S$ abelsches Schema über $S$.
  \begin{proof}[Beweisskizze]
    Es gilt zu zeigen, dass $A_\K$ in der Umgebung jedes Punktes in
    $S$ gute Reduktion hat, also die generische Faser eines
    eigentlichen, glatten Schemas über dieser Umgebung ist.
    Diese Schemata sind dann nach \autoref{thm:abelscheneronmodelle}
    bereits die lokalen Néron-Modelle und abelsch.
    Wegen der \NAbbEig ist die Kozykelbedingung
    erfüllt und sie lassen sich zu einem abelschen $S$-Schema
    verkleben.

    In den geschlossenen Punkten von $S$ hat $A_\K$ nach Definition
    gute Reduktion, welche sich auf eine Umgebung erweitern
    lässt.
    % Gute Reduktion in s\in S <=> X\times_S\O_{S,s} eigentlich
    % (\autoref{thm:neronmodell}, \autoref{thm:abelscheneronmodelle}
    % Erweiterung von Morphismen auf Umgebung:
    % \autoref{thm:allgmorphismuserweiterung}
    % ??? TODO: Wieso erhält Morphismuserweiterung Eigentlichkeit?
    % ??? Wieso lässt sich gute Reduktion auf Umgebung erweitern?
    
    Für den generischen Punkt können wir den schematischen
    Abschluss $A$ von $A_\K\to\P_\K^n\to\P_R^n$ in $\P_R^n$
    betrachten
    (hier: der Abschluss des Bildes mit reduzierter
    Unterschemastruktur nach \cite[Remark 10.32]{wedhorn}, da $A_\K$
    reduziert ist),
    denn $A_\K$ ist als abelsche Varietät projektiv,
    vgl. \autoref{thm:abvarietaetprojektiv}.
    Dieser ist glatt über dem generischen Punkt von $\Spec(R)$
    % ??? Warum ist schematischer Abschl. glatt über generischem Pkt?
    und somit auch über einer offenen Umgebung $S'$ dessen.
    Wir erhalten ein glattes, projektives (also eigentliches)
    $S'$-Modell von $A_\K$ um den generischen Punkt, was gesucht war.
  \end{proof}
\end{Satz}

%\section{Beispiele}

% \begin{Satz}
%   \cite[1.2, Proposition 8]{neron}
%   Ein abelsches Schema über $\Spec(R)$ ist Néron-Modell seiner
%   generischen Faser.
% \end{Satz}

% \begin{Satz}
%   \cite[1.2, Criterion 9]{neron}
%   Ein glattes, separiertes $\Spec(R)$-Schema endlichen Typs ist genau
%   dann ein Néron-Modell seiner generischen Faser, wenn es die
%   Erweiterungseigenschaft für étale Punkte erfüllt.
% \end{Satz}

% ----------


\chapter{Weierstraßmodelle}
\section{Konstruktion}
\subsection{Weierstraßgleichungen}
Jede elliptische Kurve $E_\K$ über einem beliebigen Körper $\K$ kann
durch eine homogene Gleichung der Form
\begin{gather*}
  0 = Y^2 Z + \c_1 XYZ + \c_3 YZ^2 - X^3 - \c_2X^2 Z - \c_4 XZ^2 - \c_6 Z^3
  \eqqcolon F\in\K[X,Y,Z]
\end{gather*}
mit Koeffizienten $\c_1,\dotsc,\c_6\in\K$ und Nullpunkt $O=[0,1,0]$
dargestellt werden, d.\,h. sie ist isomorph zur projektiven Kurve
$V_+(F)\subset\P_\K^2$.
Umgekehrt ist jede reguläre Kurve, die durch eine
solche Gleichung definiert wird, eine elliptische Kurve mit genanntem
Nullpunkt nach \cite[Proposition III.3.1]{silverman}.
Gleichungen dieser Form werden \emph{Weierstraßgleichungen} genannt.

Für $\Char(\K)\neq 2,3$ lässt sich eine Weierstraßgleichung durch
homogene Koordinatentransformation vereinfachen zur Form
\begin{gather*}
  Y^2 Z = X^3 + \beta XZ^2 + \gamma Z^3
\quad\text{bzw. dehomogenisiert}\quad
  y^2 = x^3 + \beta x + \gamma
\end{gather*}
wobei $\beta,\gamma\in\K$.
Für Berechnungen wird im Folgenden von dieser vereinfachten Form
ausgegangen. Für $\Char(\K)=2,3$ sei verwiesen auf
\cite[Appendix: Elliptic Curves in Characteristics 2 and 3]{silverman}.


% DISKRIMINANTE
Zu einer Weierstraßgleichung ist die sog. \emph{Diskriminante}
assoziiert, welche für eine Gleichung in obiger Form definiert ist als
\begin{gather*}
  \Delta = -16\left(4\beta^3 + 27\gamma^2\right)
\end{gather*}
Eine Weierstraßgleichung ist genau dann regulär (definiert also eine
elliptische Kurve), wenn $\Delta\neq 0$ gilt,
und im singulären Fall gibt es genau einen singulären Punkt
nach \cite[Proposition III.1.4]{silverman}.


\subsection{Weierstraßmodelle}
% (MINIMALE) WEIERSTRASSGLEICHUNG ÜBER $R$
Sei im Folgenden Abschnitt $R$ Dedekindring,
$\K=\Quot(R)$ und $E_\K$ elliptische Kurve über $\K$.

Uns interessiert nun, ob sich ein $R$-Schema finden lässt, das die
Struktur der elliptischen Kurve, in erster Betrachtung die
Weierstraßgleichung, erhält.

Bleiben wir bei unserer Annahme $\Char(\K)\neq2,3$, sind die
Isomorphismen zwischen Weierstraßkurven zu $E_\K$ genau durch
Koordinatenwechsel der Form
\begin{gather*}
  [x,y,z] \longmapsto [a^{-2n}x, a^{-3n}y, z] \qquad a\in\K, n\in\Z
\end{gather*}
dem Koeffizientenwechsel
\begin{align*}
  \beta&\mapsto a^{4n}\beta
  &\gamma&\mapsto a^{6n}\gamma
  &\Delta&\mapsto a^{12n}\Delta
\end{align*}
gegeben
(\cite[1.5, Lemma 2]{neron} oder \cite[Chapter VII.1]{silverman}).
Sind $\beta=\frac{\beta_1}{\beta_2}$,
$\gamma=\frac{\gamma_1}{\gamma_2}$ mit $\beta_i,\gamma_i\in R$,
erzeugt ein solcher Koordinatenwechsel mit $a=\gamma_2\beta_2$ eine
Weierstraßgleichung mit Koeffizienten in $R$.
% Beschränken wir unsere Betrachtung auf elliptische Kurven über
% Bewertungskörpern $\K$ mit diskretem Bewertungsring $R$, können wir
% die Weierstraßgleichung durch homogenen Koordinatenwechsel bezüglich
% der Ringstruktur formulieren:
% Sei $\pi$ ein Uniformierer.
% Die Koeffizienten haben die Form $\beta=u_\beta\pi^n_\beta$, $\gamma=u_\gamma\pi^n_\gamma$ und
% entsprechend liegen beide nach Multiplikation mit $\phi^n$ in $R$ für
% genügend großes $n$. Zudem kommt der homogene Koordinatenwechsel
% \begin{gather*}
%   [x,y,z] \longmapsto [\pi^{-2n}x, \pi^{-3n}y, z]
% \end{gather*}
% dem Koeffizientenwechsel
% \begin{align*}
%   \beta&\mapsto \pi^{4n}\beta
%   &\gamma&\mapsto \pi^{6n}\gamma
%   &\Delta&\mapsto \pi^{12n}\Delta
% \end{align*}
% also der Multiplikation der Koeffizienten mit positiven Potenzen von
% $\pi$ gleich
% (\cite[1.5, Lemma 2]{neron} oder \cite[Chapter VII.1]{silverman}).
% Ein Anwenden dieses Koordinatenwechsels mit genügend hohem $n$ ergibt
% also eine Gleichung mit Koeffizienten in $R$.

\begin{Definition}[Weierstraßmodell]
  Wir nennen ein projektives $R$-Schema $W=V_+(F)\subset\P_R^2$, das
  von einer homogenen Weierstraßgleichung $0=F\in R[X,Y,Z]$ von $E_\K$
  mit Koeffizienten in $R\subset\K$ erzeugt wird, im Folgenden
  ein \emph{Weierstraßmodell} von $E_\K$ über $R$.
\end{Definition}

Ist $R$ ein diskreter Bewertungsring, können wir die Bewertung der
Diskriminante durch einen Koordinatenwechsel manipulieren. Daher macht
es Sinn, eine bzgl. der gegebenen diskreten Bewertung eine minimale
Form der Weierstraßgleichung in $R$ zu betrachten:
\begin{Definition}[minimale Weierstraßgleichung]
  \optcite[Chapter 1.5, S.\,22]{neron}
  \optcite[Chapter VII.1]{silverman}
  Sei $R$ ein diskreter Bewertungsring.
  Eine Weierstraßgleichung ($\Char(\K)\neq2,3$) von $E_\K$
  \begin{gather*}
    Y^2Z = X^3 + \beta XZ^2 + \gamma Z^3
  \end{gather*}
  heißt minimal, wenn für die Koeffizienten $\beta, \gamma\in R$ gilt
  und die Bewertung der Diskriminante $\ord_\K(\Delta)$ minimal
  bzgl. homogenem Koordinatenwechsel ist.

  Wir bezeichnen ein Weierstraßmodell von $E_\K$ über $R$ als minimal,
  wenn es von einer minimalen Weierstraßgleichung von $E_\K$ erzeugt
  ist.
\end{Definition}

% -----

\section{Gruppenschemastruktur}
\subsection{Allg. Eigenschaften}
Jetzt noch ein paar Eigenschaften von Weierstraßmodellen.

% INTEGER, PROJEKTIV, FLACH
\begin{Lemma}\label{thm:eigweierstrassmodelle}
  Ein Weierstraßmodell $W=V_+(F)$ zu einer Weierstraßgleichung $0=F$
  von $E_\K$ über einem Dedekindring $R$ ist
  integer,
  projektiv (insb. quasikompakt, noethersch),
  endlich präsentiert (insb. von endlichem Typ),
  flach.
  \begin{proof}
    \begin{description}
    \item[projektiv] $W$ ist per Konstruktion projektiv. Als
      geschlossenes Unterschema des quasikompakten, noetherschen Raums
      $\P_R^n$ ist es quasikompakt und noethersch.
    \item[integer]
      $W=V_+(F)$ ist genau dann irreduzibel, wenn $F\in R[X,Y,Z]$
      irreduzibel ist.
      Da nach Voraussetzung $E_\K\cong V_+(F)\subset\P_\K^2$
      elliptische Kurve, insb. irreduzibel ist, ist $F$ irreduzibel
      als Polynom in $\K[X,Y,Z]$. Nach Algebra impliziert das schon
      $F\in R[X,Y,Z]$ irreduzibel.
      % Dass $W$ irreduzibel ist, kann auf der affinen, offenen
      % Standardüberdeckung des $\P_R^2$ durch drei Kopien des $\A_R^2$
      % überprüft werden, da diese nicht-leeren Schnitt haben. Hier
      % erhalten wir die Kurven im $\A_R^2$, die durch die
      % dehomogenisierten Formen der Weierstraßgleichung $f(X,Y,Z)=0\in
      % R[X,Y,Z]$, d.\,h. $f_Z\in R[\frac{X}{Z},\frac{Y}{Z}]=R[x,y]$,
      % $f_Y\in R[x,z]$, $f_X\in R[y,z]$, erzeugt werden.
      % Diese stimmen nach Konstruktion mit den Erzeugern für die affinen
      % Kurven zu $E_\K$ überein und sind daher irreduzibel über $\K$,
      % also auch irreduzibel über $R$.
      % Entsprechend ist die gewählte affine, offene Überdeckung von $W$
      % irreduzibel und reduziert und somit $W$ integer.
    \item[endlich präsentiert]
      $W$ ist offen überdeckt von seinen drei dehomogenisierten,
      affinen Teilmengen $V(F_i)=\Spec(R[x,y]/(F_i))$ ($F_X,F_Y,F_Z$
      Dehomogenisierungen von $F$ nach $X,Y,Z$). Diese sind jeweils
      endlich präsentiert, also ist $W$ endlich präsentiert.
      % Die zuvor beschriebene Überdeckung ist eine (endliche)
      % Überdeckung mit Spektren von endlich präsentierten $R$-Algebren,
      % also ist $W$ endlich präsentiert.
      % % ??? Stimmt das?
      Entsprechend folgt, dass $W$ von endlichem Typ ist.
    \item[flach]
      Wir werden zeigen, dass aus integer und nichtleerer generischer
      Faser $E_\K$ folgt, dass der eindeutige generische Punkt $\eta$
      von $W$ auf den generischen Punkt von $R$ geschickt wird.
      Das zeigt, dass $W$ flach ist
      nach \cite[Proposition III.9.7]{hartshorne}
      (hierbei wird zusätzlich benutzt, dass $\Spec(R)$ integer,
      regulär und von Dimension 1 ist).
      
      Nachdem $W$ integer ist, hat es einen eindeutigen generischen
      Punkt $\eta$.
      Angenommen, dieser wird durch die Abbildung, die $W$ zum
      $R$-Schema macht, auf einen geschlossenen Punkt $\m$ von $\Spec(R)$
      geschickt.
      Dann liegt $\eta$ im geschlossenen Urbild von $\m$ –
      aber der Abschluss $\overline{\{\eta\}}=X$ von $\eta$ ist darin
      enthalten. Entsprechend wäre die generische Faser $E_\K$
      leer, was ein Widerspruch ist.
      Also muss der generische Punkt auf den generischen Punkt geschickt
      werden.
    \end{description}
  \end{proof}
\end{Lemma}
% Für Regularität siehe auch \cite[Lemma IV.9.5]{silverman2}

% ÄQ GLATTHEIT
% \begin{Lemma}\label{thm:glattüberdvr}
%   Sei $R$ diskreter Bewertungsring,
%   $\K=\Quot(R)$,
%   $X$ irreduzibles,
%   lokal endlich präsentiertes
%   (für lokal noethersch äquivalent:
%   lokal endlichen Typs \cite[Remark 10.36]{wedhorn})
%   $R$-Schema mit nichtleerer, generischer Faser $X_\K$.
%   Dann ist $X$ genau dann glattes $R$-Schema, wenn die Fasern glatt
%   sind.
%   \begin{proof}
%     % siehe auch \cite[Proposition IV.2.9]{silverman2}
%     Ist $X$ glatt, so auch jede Faser, da Glattheit stabil unter
%     Basiswechsel ist.

%     Seien also die Fasern von $X$ glatt, insb. reduziert.
%     Dann ist $X$ ebenfalls reduziert, da Reduziertheit auf den Halmen
%     geprüft werden kann.
%     Außerdem ist $X$ nach Voraussetzung irreduzibel, daher integer,
%     und (lokal) endlich präsentiert.
%     Wir wissen, dass $X$ glatt ist, wenn es lokal endlich präsentiert
%     und flach über $R$ ist und seine Fasern glatt sind
%     \cite[8.5, Proposition 17]{bosch}.
%     Es ist also Flachheit zu zeigen.
    
%     Wir werden zeigen, dass aus integer und nichtleerer generischer
%     Faser $X_\K$ folgt, dass der eindeutige generische Punkt $\eta$
%     von $X$ auf den generischen Punkt von $R$ geschickt wird.
%     Das zeigt, dass $X$ flach ist
%     \cite[Proposition III.9.7]{hartshorne}
%     (hierbei wird zusätzlich benutzt, dass $\Spec(R)$ integer,
%     regulär und von Dimension 1 ist).

%     Nachdem $X$ integer ist, hat es einen eindeutigen generischen
%     Punkt $\eta$.
%     Angenommen, dieser wird durch die Abbildung, die $X$ zum
%     $R$-Schema macht, auf einen geschlossenen Punkt $\m$ von $\Spec(R)$
%     geschickt.
%     Dann liegt $\eta$ im geschlossenen Urbild von $\m$ –
%     aber der Abschluss $\overline{\{\eta\}}=X$ von $\eta$ ist darin
%     enthalten. Entsprechend wäre die generische Faser $X_\K$
%     leer, was ein Widerspruch ist.
%     Also muss der generische Punkt auf den generischen Punkt geschickt
%     werden.
%   \end{proof}
% \end{Lemma}

\begin{Lemma}\label{thm:weierstrassglatt}
  \optcite[Remark IV.5.4.1]{silverman2}
  \optcite[Chapter VII.5 für die Definition von guter Reduktion]{silverman}
  % Gute Reduktion auf elliptischen Kurven heißt, dass die spezielle
  % Faser von $W$ regulär (also glatt also ebenfalls elliptische
  % Kurve) ist
  % \optcite[Chapter VII.5]{silverman}.
  Sei $(R,\m)$ diskreter Bewertungsring,
  $\K=\Quot(R)$,
  der Restklassenkörper $\k(\m)=R/\m$ perfekt,
  $E_\K$ elliptische Kurve über $\K$,
  $W$ ein Weierstraßmodell von $E_\K$ über $R$
  mit spezieller Faser $W_\m$ über $\k(\m)$.
  Dann gilt
  \begin{enumerate}[label=(\alph*)]
  \item $W$ ist genau dann glatt, wenn die spezielle Faser von $W$ glatt
    ist.
  \item Ist $W$ nicht glatt, so ist der eindeutige, geschlossene,
    singuläre Punkt $\gamma$ der speziellen Faser der eindeutige
    Punkt, in dem $W$ nicht glatt ist, d.\,h.
    \begin{gather*}
      W^0\coloneqq W\setminus \{\gamma\}
    \end{gather*}
    ist das offene Unterschema der glatten Punkte von $W$.
    $W^0$ ist glattes $R$-Modell von $E_\K$.
  \end{enumerate}
  \begin{proof}
    \begin{enumerate}[label=(\alph*)]
    \item $W$ ist flaches, lokal endlich präsentiertes
      $R$-Schema (s. \autoref{thm:eigweierstrassmodelle}).
      Nach \cite[8.5, Proposition 17]{bosch} ist es also genau dann
      glatt, wenn seine Fasern glatt sind.
      Nachdem $W_\K=E_\K$ nach Voraussetzung glatt ist, ist $W$ genau
      dann glatt, wenn $W_\m$ glatt ist.
    \item Für Schemata lokal endlichen Typs über einem Körper sind
      glatt und geometrisch regulär äquivalent
      \cite[Corollary 6.32]{wedhorn}.
      Da $\k(\m)$ perfekt ist, ist dies für $\k(\m)$-Schemata sogar
      äquivalent zu regulär
      \cite[Remark 6.33]{wedhorn}.
      Schemata lokal endlichen Typs über einem Körper sind genau dann
      regulär, wenn sie in allen geschlossenen Punkten regulär sind
      \cite[Remark 6.25 (3)]{wedhorn}. Insbesondere sind alle
      singulären Punkte geschlossen.

      $W_\m$ ist als Weierstraßkurve über $\k(\m)$ ein $\k(\m)$-Schema
      lokal endlichen Typs, also treffen obige Aussagen zu.
      Nach \cite[Proposition III.1.4]{silverman} ist unter den
      geschlossenen Punkten von $W_\m$ maximal ein singulärer Punkt
      $\gamma$. Demnach kann $W_\m$ maximal in einem geschlossenen
      Punkt $\gamma\in W_\m$ singulär (bzw. nicht glatt) sein.

      Angenommen, $W_\m$ habe den singulären Punkt $\gamma$.
      $W_\m=W\setminus E_\K$ ist als Urbild des geschlossenen Punktes
      $\m$ von $\Spec(R)$ geschlossenes Unterschema von $W$.
      Also ist der Punkt $\gamma$, der geschlossen in $W_\m$ ist, auch
      geschlossener Punkt von $W$.
      
      Das Unterschema $W^0=W\setminus\gamma$ ist dementsprechend offen.
      $W^0$ ist irreduzibel, da der Abschluss $\overline{W^0}=W$ in
      $W$ irreduzibel ist nach \autoref{thm:eigweierstrassmodelle}.
      Es ist als Unterschema von $W$ ebenfalls reduziert, endlich
      präsentiert und flach (s. \autoref{thm:eigweierstrassmodelle}).
      Es hat wie $W$ die (nichtleere) glatte, generische Faser $E_\K$ und
      per Konstruktion die spezielle Faser $W_{\m}\setminus\{\gamma\}$,
      welche nach Voraussetzung beide regulär (also glatt) sind.
      Damit ist $W^0$ wieder nach \cite[8.5, Proposition 17]{bosch}
      glatt über $R$ bzw. glattes $R$-Modell von $E_\K$ und das offene
      Unterschema der glatten Punkte von $W$.
    \end{enumerate}
  \end{proof}
\end{Lemma}

\begin{Bemerkung}\label{thm:weierstraßmodellglatt}
  Für den Fall eines diskreten Bewertungsringes $(R,\m=(\pi))$ liefert
  \autoref{thm:weierstrassglatt} ein sehr einfaches Kriterium für die
  Glattheit eines Weierstraßmodell $W$ von $E_\K$.
  $W$ ist genau dann glatt, wenn seine spezielle Faser glatt
  (bzw. äquivalent dazu regulär) ist.
  Dies lässt sich recht leicht an der Weierstraßgleichung ablesen.
  
  Die spezielle Faser $W_{\m}=W\times_R R/\m$ ist definiert
  durch die Reduktion der Weierstraßgleichung von $W$ modulo dem
  Maximalideal $\m$ von $R$, also ebenfalls eine Weierstraßkurve.
  % ??? Ungenau/unschön formuliert
  Entsprechend ist $W_{\m}$ (und damit $W$) genau dann glatt, wenn
  eine der äquivalenten Bedingungen erfüllt ist
  ($\Delta_{\m}$ Diskriminante von $W_{\m}$,
  $\Delta$ Diskriminante von $W$)
  \cite[Proposition VII.5.1 (a)]{silverman}:
  \begin{enumerate}[label=(\roman*)]
  \item $0\neq\Delta_{\m}=\bar\Delta\in R/\m$.
  \item $\Delta\not\in\m$ bzw. $\pi\nmid\Delta$.
  \item $\ord_R(\Delta)=0$ bzw. $\Delta\in R^\times$
  \end{enumerate}
  Letzteres ist einfach nachzuprüfen.
\end{Bemerkung}

\subsection{Gruppenstruktur}
\begin{Lemma}\label{thm:erweiterunggruppenstruktur}
  Sei $(R,\m)$ hier diskreter Bewertungsring,
  $\k(\m)=R/\m$ perfekt,
  $\K=\Quot(R)$.
  Sei $W$ ein Weierstraßmodell von $E_\K$ über $R$ und
  $W^0$ das offene $R$-Unterschema der glatten Punkte von $W$ 
  (ebenfalls $R$-Modell von $E_\K$ nach \autoref{thm:weierstrassglatt}).
  $W^0$ ist ein glattes $R$"=Gruppenschema
  mit Addition, Inversenbildung und neutralem Schnitt
  \begin{gather*}
    \mu_0\colon W^0\times_R W^0 \longto W^0
    \qquad
    i_0\colon W^0\longto W^0
    \qquad
    \epsilon_0\colon R\longto W^0
  \end{gather*}
  welche die Gruppenstruktur (Addition $\mu$, Inversenbildung $i$ und
  neutraler Schnitt $\epsilon$)
  der abelschen Varietät $E_\K$ erweitern.
  \optcite[Theorem IV.5.3 (c)]{silverman2}

  % TODO: ??? Referenzen
  $W^0(R)\to W(R) \cong E_\K(\K)$ ist ein Gruppenisomorphismus,
  falls $W$ regulär ist.
  \optcite[Theorem IV.5.3 (b)]{silverman2}
  
  \begin{proof}[Beweisskizze] Sei wieder $\Char(\K)\neq 2,3$.
    \autoref{thm:erweiterunggruppenstruktur} folgt als Spezialfall von
    \autoref{thm:exneronmodelle}.
    Für das Nachweisen der Gruppenstruktur ist zu zeigen, dass die von
    $E_\K$ induzierten, rationalen Abbildungen $\mu\colon W\times_R
    W\dra W$ und $i\colon W\dra W$ auf $W^0$ definiert sind.

    $i$ ist in projektiven Koordinaten (also als graduierter
    Ringmorphismus der zugrundeliegenden graduierten Ringe)
    dargestellt als
    \begin{gather*}
      i\colon [X,Y,Z]\mapsto [X,-Y,Z]
    \end{gather*}
    Dies definiert einen Isomorphismus auf ganz $\P_R^2$, ist also
    insbesondere auf $W^0$ definiert.

    Für $\mu$ kann die zugehörige Darstellung als Morphismus
    graduierter Polynomringe über $\K$ auf einen Morphismus von
    Polynomringen über $R$ erweitert werden \cite[vgl. Formeln für
    Addition in][Group Law Algorithm III.2.3]{silverman}.
    Berechnungen wie in \cite[Proposition 2.5]{silverman} zeigen, dass
    $\mu$ auf $W\times_R W$ höchstens im singulären Punkt nicht
    definiert ist, falls dieser existiert.
    Dies kann auf der affinen Überdeckung von $W$ nachgerechnet werden.
    
  %   \begin{enumerate}
  %   \item $W^0=W\setminus\{\gamma\}$: \autoref{thm:weierstrassglatt}
  %     % \cite[Remark IV.5.4.2]{silverman2}
  %     % Die nicht-glatten Punkte von $W$ sind die singulären Punkte
  %     % der Fasern.
  %     % Die generische Faser ist regulär (ell. Kurve), die spezielle
  %     % Faser ist durch eine Weierstraßgleichung über $\k(\m)$
  %     % definiert. Sie hat also maximal einen singulären Punkt $\gamma$.
  %     % \cite[Proposition III.1.4 (a)]{silverman}
  %   \item $W^0$ ist $R$-Modell von $E_\K$: \autoref{thm:weierstrassglatt}
  %     % $W$ ist $R$-Modell von $E_\K$ per Definition,
  %     % $W^0=W\setminus\{\gamma\}$ hat dieselbe generische Faser
  %   \item Es treffen alle Voraussetzungen für
  %     \autoref{thm:ratpkteregulaeremodelle} auf $W$ zu
  %     ($W$ ist projektiv, also eigentlich, von Dimension 2 mit
  %     projektiven Kurven als Fasern, und nach
  %     \autoref{thm:weierstrassglatt} genau dann glatt in einem Punkt,
  %     wenn die Fasern dort regulär sind).
  %     Also gilt $E_\K(\K)=W(R)=W^0(R)$.
  %     % \cite[Corollary IV.4.4]{silverman2}
  %     % $E_\K(\K)=W(R)=W^0(R)$
  %     % (s. \autoref{thm:ratpkteregulaeremodelle}):
  %     % \begin{enumerate}
  %     % \item $E_\K(\K)=W(R)$: $W$ eigentlich
  %     % \item $W(R)\subset W^0(R)$:
  %     %   nach \cite[Proposition IV.4.3 (b)]{silverman2}
  %     % \item $W(R)=W^0(R)$: Alle Bilder von Schnitten $W(R)$ sind
  %     %   glatt
  %     % \end{enumerate}
  %   \item rationale Abbildungen
  %     $\mu\colon W\times_R W\dra W$,
  %     $i\colon W\dra W$:
  %     von generischer Faser $E_\K$ induziert, erfüllen Gruppenaxiome
  %     auf (größter, offener) Definitionsmenge (wo es Morphismus ist)
  %   \item $\mu$ definiert auf $W^0$:
  %     \begin{enumerate}
  %     \item Aufteilen in affine Überdeckung
  %       ($W_Z=W\cap U_Z=\{[x,y,z]\in W| z\neq0\}$,
  %       $W_Y=W\cap U_Y=\{[x,y,z]\in W| y\neq0\}$)
  %       $W_Z\times W_Z$, $W_Z\times W_Y$, $W_Y\times W_Z$,
  %       $W_Y\times W_Y$
  %     \item $\mu$ auf $W_Z\times W_Z\setminus(\gamma,\gamma)$
  %       definiert:
  %       Schreibe rationale Abbildung von $\mu$ auf und finde
  %       Gleichungen für Unterschema, an denen der Nullpunkt getroffen
  %       würde
  %     \end{enumerate}
  %   \item $i$ definiert auf $W^0$:
  %     $i\colon [X,Y,Z]\mapsto [X,-Y,Z]$ ist Morphismus in $\P_R^2$,
  %     also definiert auf ganz $W$
  %   \end{enumerate}
  \end{proof}
\end{Lemma}

% IDENTITÄTSKOMPONENTE
\begin{Bemerkung}
  \cite[Chapter 1.5, S.\,23]{neron},
  \cite[Corollary IV.9.1]{silverman2},
  \cite[S. 46]{tate}
  Für ein minimales Weierstraßmodell $W$ von $E_\K$ über $R$
  ist $W^0$ isomorph zur Identitätskomponente des Néron-Modells von
  $E_\K$ über $R$.
\end{Bemerkung}

% Im Folgenden sei $E_\K$ immer eine elliptische Kurve über dem
% Quotientenkörper $\K$ des Dedekindringes $R$.

In bestimmten Fällen erhält man das Néron-Modell bereits aus einem
Weierstraßmodell von $E_\K$ über $R$.
\begin{Korollar}\label{thm:neronmausweierstrassgl}
  \optcite[Corollary IV.6.3]{silverman2}
  Sei $R$ Dedekindring mit perfekten Restklassenkörpern, $\K=\Quot(R)$,
  $E_\K$ elliptische Kurve über $\K$, $W$ Weierstraßmodell von $E_\K$
  über $R$.
  Ist $W$ glatt, so ist $W$ bereits das Néron-Modell von $E_\K$ über
  $R$.
  \begin{proof}
    Für die Gruppenstruktur betrachte vorerst die Unterschemata
    $W\times_R R_s$ von $W$ über den Lokalisierungen
    $R_s$ ($s\in\Spec(R)$, $\Spec(R_s)\hookrightarrow\Spec(R)$) von $R$.
    Diese tragen nach \autoref{thm:erweiterunggruppenstruktur} eine
    Gruppenschemastruktur, die $E_\K$ erweitert, da die $R_s$ diskrete
    Bewertungsringe sind.
    Diese Strukturen verkleben nach Definition zu einer
    Gruppenschemastruktur auf ganz $W$, da sie durch die gleiche
    Abbildung zwischen graduierten Ringen definiert werden.

    Für die \NAbbEig sei $X$ ein weiteres glattes $R$-Schema mit
    generischer Faser $X_\K$ und $\phi_\K\colon X_\K\to E_\K$ ein
    Morphismus.
    Wir werden zeigen, dass $\phi_\K$ eine $R$-rationale Abbildung
    $\phi\colon X\dra W$ liefert.
    Nachdem $W$ glattes, d.\,h. insb. eigentliches, $R$-Gruppenschema
    ist, ist $\phi$ nach \autoref{thm:rationalzumorphismus} ein
    Morphismus, der $\phi_\K$ erweitert, und die \NAbbEig ist erfüllt.
    
    $\Spec(R)$ und folglich $X$ als glattes $R$-Schema sind lokal
    noethersch, außerdem ist $\pi\colon X\to R$ flach über $R$. Daher
    gilt für ein $x\in X$ \optcite{?}
    \begin{gather*}
      \dim\O_{X,x} = \dim R_{\pi(x)} + \dim\O_{X_{\phi(x)}, x}
    \end{gather*}
    Für einen generischen Punkt $\eta_{s}$ einer irreduziblen
    Komponente $Y_{s}$ der Faser $X_s$, $s\in\Spec(R)$,
    von $X\to R$ gilt $\dim\O_{X_s, \eta_{s}}=0$, also
    $\dim\O_{X,\eta_{s}}=\dim R_s=1$.
    Da $X$ als glattes $R$-Schema auch regulär über $R$ ist, ist
    $\O_{X,\eta_{s}}$ diskreter Bewertungsring.
    Der Quotientenkörper von $\O_{X,\eta_{s}}$ ist der lokale Ring
    eines generischen Punkts $\eta$ von $X$. Da $X$ flach und
    reduziert ist, liegt dieser nach
    \cite[Proposition III.9.7]{hartshorne} in der generischen Faser.
    Wir können also das Bewertungskriterium für Eigentlichkeit für $W$
    anwenden:
      \begin{center}
        \begin{tikzcd}
          \eta\arrow[r, equal]
          &\Spec(\k(\eta_j))
          \arrow[r, "{\phi_\K|_{\eta}}"] \arrow[d]
          & E_\K \arrow[r, hook]
          & W\arrow[d]\\
          \eta\cup\eta_{s}\arrow[r, equal]
          &\Spec(\O_{X,\eta_{s}}) \arrow[rr]
          \arrow[urr, dashed, "{\exists!}", out=20, in=215]
          && \Spec(R)
        \end{tikzcd}
      \end{center}
    Damit erweitert sich $\phi_\K$ zu einem Morphismus
    $\Spec(\O_{X,\eta_{s}})\to W$, der sich nach
    \autoref{thm:morphismuserweiterung} auf einen Morphismus in einer
    Umgebung von $\eta_{s}$ erweitert.
    
    Die so gewonnenen Morphismen in Umgebungen der generischen Punkte
    der Fasern können wiederum verklebt werden und
    wir erhalten eine rationale Abbildung $\phi\colon U\to W$ auf
    einem offenen Unterschema $U\subset X$.
    $U$ enthält alle generischen Punkte der Fasern von $X\to R$ und
    ist somit $R$-dicht.
    $\phi$ ist demnach $R$-rationale Erweiterung von $\phi_\K$.
    
    % Denn $\phi_\K$ lässt sich eindeutig auf eine offene, $R$-dichte
    % Umgebung $U$ von $X_\K$ erweitern, wie sich lokal zeigen lässt.
    % \begin{description}
    % \item[Erweiterung auf $U$]
    %   % ??? Wie erweitert sich die rationale Abbildung?
    %   % Sowohl $X$ als auch $W$ sind glatt, also endlich
    %   % präsentiert.

    %   % Da $R$ Dedekindring ist, lässt faktorisiert jedes Element in ein
    %   % endliches Produkt aus Primzahlen (s. \autoref{def:dedekindring}). 
      
    %   % Sei $x\in X_\K$ und $x\in\Spec(B)\subset X$ genügend kleine,
    %   % offene, affine Umgebung von $x$, dass das Bild $\im(\Spec(B))$ in
    %   % einer affinen Teilmenge $\Spec(A)\subset W$ liegt und $A$, $B$
    %   % endlich präsentierte $R$-Algebren sind.
    %   % Es gilt $A\subset A_{\phi_\K(x)}=\O_{W,\phi_\K(x)}$,
    %   % $B\subset B_x=\O_{X,x}$.
    % \item[$U$ $R$-dicht]
    %   % $X\to R$ schickt als glattes, d.\,h. reduziertes und flaches
    %   % $R$-Schema, alle generischen Punkte seiner irreduziblen
    %   % Komponenten auf den generischen Punkt von
    %   % $\Spec(R)$
    %   % \cite[Proposition III.9.7]{hartshorne}.
    %   % Entsprechend liegen alle generischen Punkte in der
    %   % generischen Faser und sie ist dicht in $X$.
    %   $X_\K$ ist dicht in $X$ nach \autoref{thm:dichtefaser}, da $X$
    %   glatt ist.
    %   Daher auch $U$, da es über $X_\K$ liegt.  
    % \end{description}
  \end{proof}
\end{Korollar}
\begin{Bemerkung}
  Im Fall diskreter Bewertungsringe kann die Glattheit anhand der Fasern
  überprüft werden:
  Wir wissen aus \autoref{thm:weierstraßmodellglatt},
  dass ein Weierstraßmodell $W$ von $E_\K$ über $R$ genau dann glatt
  ist, wenn $\ord_R(\Delta)=0$ für die Diskriminante der
  Weierstraßgleichung gilt.
\end{Bemerkung}



\chapter{Arithmetische Flächen}
\section{Definition}
\begin{Definition}[arithmetische Fläche]
  Sei $R$ Dedekindring mit Quotientenkörper $\K$.
  % \optcite[IV.4]{silverman2}
  % Eine arithmetische Fläche $X$ über $R$ ist ein
  % integeres, normales, exzellentes Schema, flach und von endlichem
  % Typ über $R$,
  % dessen generische Faser eine reguläre, zusammenhängende,
  % projektive Kurve $X_\K/\K$ ist
  % und dessen spezielle Fasern Vereinigungen von Kurven über den
  % entsprechenden Restklassenkörpern sind.

  % % exzellent-Bediungung:
  % Hat $R$ Quotientenkörper der Charakteristik Null, so ist jedes
  % $R$-Schema lokal endlichen Typs bereits excellent.
  % \optcite[Theorem 12.51]{wedhorn}
  
  \optcite[Definition 8.3.1, 8.3.14]{liu}
  Eine arithmetische Fläche ist ein projektives, integeres, flaches,
  reguläres $R$-Schema von Dimension 2 (also Dimension 1 über $R$).
  Die generische Faser $X_\K$ ist integere Kurve über $\K$ und jede
  spezielle Faser $X_s$ ($s\in\Spec(R)$) ist projektive Kurve über
  $\k(s)$ nach \cite[Lemma 8.3.3]{liu}.
  
  $X$ ist also ein $X_\K$-Modell.
\end{Definition}
% -----
\section{Eigenschaften}
\begin{Satz}\label{thm:ratpkteregulaeremodelle}
  \optcite[Corollary IV.4.4]{silverman2}
  Sei $R$ Dedekindring
  mit perfekten Restklassenkörpern,
  $\K=\Quot(R)$,
  $X$ ein $R$-Schema,
  $X^0$ offenes Unterschema der glatten Punkte,
  $X_\K/\K$ generische Faser.
  Dann gilt
  \begin{enumerate}[label=(\roman*)]
  \item Ist $X$ eigentlich über $R$, so gilt $X(R)=X_\K(\K)$.
  \item Ist $X$ regulär über $R$ mit Dimension 2 und mit
    Fasern der Dimension 1, und ist $X$ genau dann glatt in einem
    Punkt $x\in X$, wenn die ihn enthaltene Faser regulär in $x$ ist,
    so gilt $X(R)=X^0(R)$.
  \end{enumerate}

  Es gilt also für ein reguläres, eigentliches $R$-Modell von $X_K$,
  das genau dann glatt in $x\in X$ ist, wenn die ihn enthaltene
  Faser in $x$ regulär ist,
  \begin{gather*}
    X_\K(\K)=X^0(R)=X(R)
  \end{gather*}
  \begin{proof}
    % ??? benötigte Eigenschaften von X: eigentlich, regulär, flach
    \begin{enumerate}[label=(\roman*)]
    \item
      Die Einschränkung von $\Hom_R(\Spec(R),X)=X(R)$ auf 
      $\Spec(\K)\subset\Spec(R)$ liefert eine kanonische Abbildung
      \begin{center}
        \begin{tikzcd}
          \Hom_R(\Spec(R),X) \arrow[d, equal]
          \arrow[r, "{\text{Res.}}"]
          &\Hom_R(\Spec(K),X)
          \arrow[r, equal, "{\text{$R$-lin.}}"]
          &\Hom_\K(\Spec(\K),X_\K) \arrow[d, equal]\\
          X(R)&& X_\K(\K)
        \end{tikzcd}
      \end{center}
      Nach Voraussetzung ist $X$ eigentlich über $R$, d.\,h. nach
      dem Bewertungskriterium für $X$ gibt es für ein $\phi_\K\in
      X_\K(\K)$ genau einen Lift in $\Hom_R(R,X)$,
      bzw. ein $\phi\colon\Spec(R)\to X$, so dass das folgende
      Diagramm kommutiert
      \cite[Theorem II.4.7]{hartshorne}:
      \begin{center}
        \begin{tikzcd}
          \Spec(\K)\arrow[d, hook, "{\text{incl}}"]
          \arrow[r,"\phi_\K"]
          &X_\K\arrow[r, hook, "{\text{incl}}"]
          &X\arrow[d,"{\text{eig.}}"]
          \\
          \Spec(R)\arrow[rr,"{\Id}"]
          \arrow[to=urr, dashed, "{\exists!\phi}"{near start}]
          &&\Spec(R)
        \end{tikzcd}
      \end{center}
      Dadurch sind Injektivität und Surjektivität der
      Einschränkungsabbildung gegeben und es gilt
      $X(R)=\Hom_R(R,X)\cong\Hom_\K(\K,X_\K)=X_\K(\K)$.
    \item Da $X^0\subset X$, gibt es eine natürliche Inklusion
      $X^0(R)\overset{\circ \text{incl}}{\hookrightarrow} X(R)$.
      Jetzt gilt es zu zeigen, dass im regulären Fall das Bild jedes
      Schnitts $\phi\in X(R)$ in $X^0$ liegt (also nur glatte Punkte
      enthält). Dann ist die Inklusion ein Isomorphismus.

      % $X$ ist nach Voraussetzung flach über $R$, also ist glatt in
      % einem Punkt $x\in X$ äquivalent dazu, dass die Faser von $X$, in
      % der $x$ liegt, glatt in $x$ ist
      % \cite[8.5, Proposition 17]{bosch}.
      % Die Fasern von $X$ sind nach Voraussetzung Vereinigungen
      % projektiver Kurven, also Schemata lokal endlichen Typs über
      % einem der perfekten Restklassenkörper von $R$.
      % Nach \cite[Corollary 6.32]{wedhorn} und
      % \cite[Remark 6.33]{wedhorn} sind also glatt und regulär auf den
      % Fasern äquivalent.
      
      Nach \autoref{thm:arithflschnittbilderglatt} enthält das Bild
      $\phi(\Spec(R))$ eines Schnitts $\phi\in X(R)$ nur reguläre
      Punkte der Fasern, also nach Voraussetzung glatte Punkte von $X$.
      Damit ist die Behauptung gezeigt.
    \end{enumerate}
  \end{proof}
\end{Satz}

\begin{Bemerkung}
  Die obigen Bedingungen treffen z.\,B. auf ein eigentliches,
  reguläres, flaches $R$-Schema zu, also insbesondere arithmetische
  Flächen (diese sind integer, regulär, flach, projektiv).
\end{Bemerkung}

\begin{Lemma}\label{thm:arithflschnittbilderglatt}
  \cite[Proposition IV.4.3]{silverman2}
  Sei $R$ Dedekindring (mit perfekten Restklassenkörpern),
  $\p\in\Spec(R)$,
  $\pi\colon X\to\Spec(R)$ ein reguläres $R$-Schema der Dimension 2
  mit Fasern der Dimension 1,
  $\phi\in X(R)$.
  Dann ist $X_\p$ regulär in $\phi(\p)$.
  \begin{proof}
    …  
  \end{proof}
\end{Lemma}

% -----

\section{Minimale reguläre Modelle}
% exzellent nötig!!! evtl. nur Konstruktion?
\begin{Satz}
  Sei $R$ Dedekindring (mit perfekten Restklassenkörpern),
  $\K=\Quot(R)$,
  $C/\K$ reguläre, projektive Kurve vom Geschlecht $g$.
  \begin{enumerate}[label=(\roman*)]
  \item\cite[Proposition IV.4.5(a)]{silverman2}
    Dann existiert ein eigentliches, reguläres $R$-Modell $X$ von $C$,
    das eine arithmetische Fläche darstellt
    (kurz: eigentliches, reguläres Modell für $C/\K$).
  \item\cite[Proposition IV.4.5(b)]{silverman2} Ist $g\geq1$ kann
    dieses minimal gewählt werden, d.\,h.
    für jedes andere eigentliche, reguläre Modell für $C/\K$ und jeden
    Isomorphismus der generischen Fasern ist die induzierte birationale
    Abbildung
    % ??? Welche induzierte rat. Abbildung?
    ein $R$-Isomorphismus.
  \item\cite[Proposition IV.4.6]{silverman2}
    Für ein minimales, eigentliches, reguläres Modell $X$ für $C/\K$
    gilt, dass jeder $\K$"=Automorphismus von $C$ sich zu einem
    $R$-Morphismus von $X$ erweitert, der glatte Punkte auf glatte
    Punkte abbildet.
  \end{enumerate}
\end{Satz}

% ----------


\chapter{Existenz von Néron-Modellen auf elliptischen Kurven}
% EXISTENZSATZ ELL. KURVEN
\begin{Satz}\label{thm:exneronmodelle}
  % Der diskrete Bewertungsring $R$ sei strikt henselsch mit algebraisch
  % abgeschlossenem Quotientenkörper $\K$ und
  % $W$ ein minimales reguläres, eigentliches, flaches $R$-Modell der
  % elliptischen Kurve $E_\K$.
  % Dann ist das $R$-Unterschema der glatten Punkte von $W$ ein
  % Néron-Modell von $E_\K$.
  % \cite[1.5, Proposition 1, S.\,21]{neron}
  % \begin{proof}
  %   …
  % \end{proof}

  % Fehlt: \cite[Theorem IV.6.1]{silverman2}
  % (ähnlich, ohne Bed. an den Ring, dafür mehr an das Modell)
  Sei $R$ Dedekindring mit perfekten Quotientenkörpern,
  $\K=\Quot(R)$, $E_\K$ elliptische Kurve über $\K$ und $C$ minimales
  eigentliches, reguläres $R$-Modell von $E_\K$
  (d.\,h. eigentliches, reguläres, integeres, projektives,
  flaches $R$-Schema mit Minimalitätseigenschaft).
  Sei $E$ das größte glatte $R$-Unterschema von $C$.
  Dann ist $E$ Néron-Modell von $E_\K$.

  Beachte, dass minimale eigentliche reguläre Modelle elliptischer
  Kurven existieren nach \autoref{thm:exneronmodellabvarietaet}.
  \begin{proof}
    \begin{enumerate}
    \item \autoref{thm:rationalzumorphismus}:
      Eine rationale Abbildung $X\dra G$ von einem glatten
      $R$-Schema in ein eigentliches $R$-Gruppenschema ist Morphismus.
    \item \autoref{def:henselscheringe}: (strikt) Henselsche Ringe
    \item Eigenschaften henselscher Ringe auf Schnitteinschränkung:
      $(R,\m)$ diskreter Bewertungsring mit Restklassenkörper $k$, $X$
      glattes $R$-Schema, $X_k$ spezielle Faser.
      Dann
      \begin{enumerate}[label=(\roman*)]
      \item Falls $R$ henselsch, ist $X(R)\to X_k(k)$ surjektiv.
      \item Falls $R$ strikt henselsch, ist $X(R)\to X_k(k)$ surjektiv
        und hat dichtes Bild in $X_k$.
        (Beachte, dass $X_k$ glatt, also von endlichem Typ ist und $k$
        nach Voraussetzung algebraisch abgeschlossen ist. Daher
        gilt $X_k(k)\cong\{x\in X_k\text{ geschl.}\}\subset X_k$
        \cite[Example 4.1]{wedhorn}.)
      \end{enumerate}
      \cite[Proposition IV.6.4]{silverman2},
      \cite[2.3, Proposition 5]{neron}
    \item Henselisierung: \autoref{thm:exhenselisierung}
    \item Fall $R$ strikt henselscher, diskreter BWR:
      \begin{enumerate}
      \item Ist $E$ Gruppenschema, so ist es Néron-Modell
        \autoref{thm:fallstriktehenselisierung}
      \item Def. Normale Komposition \autoref{def:normalekomposition}
      \item Für $R$ strikter henselscher, diskreter BWR
        definiert $E_\K$ eine normale Komposition auf $E$.
      \item (Theorem von Weil) Ein glattes $R$-Schema mit normaler
        Komposition ist birational zu einem $R$-Gruppenschema endlichen
        Typs: \autoref{thm:weil}
      \item Also ist $E$ birational zu einem Gruppenschema $G$.
        Die rationalen Abbildungen sind Morphismen.
        \autoref{thm:egruppenschema}
      \end{enumerate}
    \item Fall diskreter BWR:
      \begin{enumerate}
      \item Minimale eigentliche, reguläre Modelle sind stabil unter
        flachem, unverzweigtem Basiswechsel.
      \item $E^\text{sh}=E\times_R\Rsh$ ist glatter Teil von
        $C^\text{sh}$, also Néron-Modell von $E_\K^\text{sh}$.
      \item $E$ ist Néron-Modell wegen „faithfully flat descent“
      \end{enumerate}
    \item Allg. Dedekindring: lokal (\autoref{thm:neronmodelllokal})
    \end{enumerate}
  \end{proof}
\end{Satz}
Beweisidee: Wegen \autoref{thm:neronmodelllokal} können wir uns auf
diskrete Bewertungsringe $R$ beschränken.
…

\section{Äquivalenz zur Gruppenstruktur}

% ÄQ GRUPPENSTRUKTURERW.
% strikt henselsch nicht nötig!!
\begin{Lemma}\label{thm:fallstriktehenselisierung}
  Sei $R$ hier strikt henselscher, diskreter Bewertungsring mit
  perfektem Restklassenkörper $k$
  und $\K$, $E_\K$, $C$, $E$ wie in \autoref{thm:exneronmodelle}.
  Erweitert sich die $\K$-Gruppenschemastruktur von $E_\K$ auf
  $E=C^0$, dann ist $E$ Néron-Modell von $E_\K$.
  \begin{proof}
    Da $E$ bereits glattes $R$-Gruppenschema ist nach Voraussetzung,
    ist noch die \NAbbEig zu prüfen.
    Sei also $Y$ ein glattes $R$-Schema mit einem Morphismus
    $\phi_\K\colon Y_\K\to E_\K$, bzw. nach \autoref{thm:dichtefaser}
    einer rationalen Abbildung $\phi\colon Y\dra E$, die von $\phi_\K$
    präsentiert wird.

    Wir werden den Beweis durch Widerspruch führen.
    Angenommen, $\phi$ sei kein Morphismus.
    \begin{enumerate}
    \item Da $E$ $R$-Gruppenschema ist und $\phi$ auf der dichten
      generischen Faser definiert ist, hat $Y\setminus\Dom(\phi)$
      Kodimension 1 (s. \autoref{thm:rationalzumorphismus})
      Daher gibt es ein irreduzibles, geschlossenes Unterschema
      $Z\subset Y$, auf dem $\phi$ nicht definiert ist.
      Dies ist äquivalent dazu, dass $\phi$ nicht im generischen Punkt
      $\eta_Z$ von $Z$ definiert ist, da jede offene Teilmenge $U$ von
      $Z$ Umgebung von $\eta_Z$ ist.
    \item $\O_{Y,\eta_Z}$ ist diskreter Bewertungsring, da er von
      Dimension 1 ist ($Z$ hat Dimension 1), regulär ($Y$ glatt, also
      regulär) und lokal ist.
      Sein Quotientenkörper ist der Quotienten-~bzw. Restklassenkörper
      eines generischen Punkts $\eta$ der irreduziblen Komponente von
      $X$, in der $\eta_Z$ liegt. Wir erhalten somit ein kommutatives
      Diagramm
      \begin{center}
        \begin{tikzcd}
          \eta\arrow[r, equal]
          &\Spec(\k(\eta))
          \arrow[r, "{\phi_\K|_{\eta}}"] \arrow[d]
          & E_\K \arrow[r, hook]
          & C\arrow[d]\\
          \eta\cup\eta_Z\arrow[r, equal]
          &\Spec(\O_{Y,\eta_Z}) \arrow[rr]
          \arrow[urr, dashed, "{\exists!}", out=20, in=215]
          && \Spec(R)
        \end{tikzcd}
      \end{center}
      % ??? stimmen die Gleichheiten?
      $C$ ist nach Voraussetzung eigentlich, daher erweitert sich
      $\phi_\K|_{\eta}$ nach dem Bewertungskriterium für
      Eigentlichkeit auf einen Morphismus
      $\Spec(\O_{Y,\eta_Z})\to C$.
    \item \autoref{thm:morphismuserweiterung} besagt, dass sich
      dieser auf eine Umgebung von $\eta_Z$ erweitert, was bedeutet,
      dass $\phi\colon Y\dra E\hookrightarrow C$ in $\eta_Z$ definiert
      ist.
    \item Also kann $\phi(\eta_Z)$ nicht in $E$ liegen. Ansonsten
      wäre $\phi\colon Y\dra E$ auf dem Urbild einer genügend kleinen
      Umgebung von $\eta_Z$ in $E$ Morphismus und entgegen der Annahme in
      $\eta_Z$ definiert.
      % C integer, urbilder dichter Mengen dicht
    \item[?] $\phi(\eta_Z)\not\in E$ ist äquivalent dazu, dass für
      jeden geschlossenen (also $k$-wertigen) Punkt $x\in Z\setminus
      X_\K$, in dem $\phi$ definiert ist, $\phi(x)\not\in E$ gilt.
    \item Da $R$ strikt henselsch ist, hat die Einschränkung $Y(R)\to
      Y_k(k)$ der $R$-wertigen Punkte auf die spezielle Faser ein
      dichtes Bild in $Y_k$ nach \autoref{thm:eigstrikthenselsch}.
    \item[?] Es gibt daher einen $R$-wertigen Punkt $x\in Y(R)$,
      dessen Einschränkung $x_0\in Y_k(k)$ einem geschlossen Punkt in
      $Z$ entspricht und in dem $\phi\colon Y\to C$ definiert ist.
    \item Wir erhalten das kommutative Diagramm
      \begin{center}
        \begin{tikzcd}
          \Spec(\K) \arrow[r, "{x|_{\Spec(\K)}}"] \arrow[d]
          & Y_\K \arrow[r, "{\phi_\K}"]
          & E_\K \arrow[r, hook]
          & C \arrow[d]\\
          \Spec(R) \arrow[rrr, "\Id"{swap}]
          \arrow[urrr, dashed, "{\exists!}", in=210, out=15]
          &&&\Spec(R)
        \end{tikzcd}
      \end{center}
      und nach dem Bewertungskriterium für die Eigentlichkeit von $C$
      erweitert sich $\phi\circ x|_{\Spec(\K)}$ zu einem Morphismus
      $\Spec(R)\to C$, d.\,h.
      \begin{gather*}
        C(R)\ni\phi\circ x\colon
        \Spec(R)
        \overset{x}{\longto}
        \left\{ x|_{\Spec(\K)}, x_0 \right\}
        \overset{\phi}{\longto} C
      \end{gather*}
      % ??? Stimmt die Gleichheit so?
      ist Morphismus.
    \item Es muss aber gelten, dass $\phi\circ x\not\in E(R)$, da
      ansonsten entgegen unserer Annahme $\phi$ in $x_0$ definiert
      wäre.
    \item Dies ist ein Widerspruch, da nach
      \autoref{thm:ratpkteregulaeremodelle} die Gleichheit
      $x\not\in E(R)=C(R)\ni x$ gilt.
    \end{enumerate}
  \end{proof}
\end{Lemma}

\section{Fall strikt henselscher Ringe}
% Nur Erweiterung der Gruppenstruktur!
\subsection{Strikt henselsche Ringe}
% HENSELSCHE RINGE
\begin{Definition}[henselsche Ringe]\label{def:henselscheringe}
  \optcite[2.3, Proposition 4]{neron}
  \optcite[Chapter IV.6]{silverman2}
  Ein lokaler Ring $(R,\m)$ heißt \emph{henselsch}, wenn
  er eine der folgenden Bedingungen erfüllt
  \begin{enumerate}[label=(\roman*)]
  \item henselsches Lemma:
    Für jedes normierte Polynom $f\in R[X]$ und jede einfache Nullstelle
    $\bar a\in R/\m$ seiner Restklasse $\bar f\in (R/\m)[X]$
    gibt es eine eindeutige Nullstelle $a\in R$ von $f$, die ein Lift
    von $\bar a$ ist.
  \item Jede endliche $R$-Algebra ist ein Produkt lokaler Ringe.
  \end{enumerate}
  
  Ein henselscher Ring heißt \emph{strikt henselsch}, falls sein
  Restklassenkörper $R/\m$ separabel abgeschlossen ist
  (im Fall eines perfekten Restklassenkörpers ist dies äquivalent zu
  algebraisch abgeschlossen).
\end{Definition}

\begin{Definition}[Henselisierung]\label{def:henselisierung}
  \optcite[2.3, Definition 6]{neron}
  Die Henselisierung eines lokalen Rings $R$ ist ein henselscher
  lokaler Ring $\Rh$ mit einem lokalen Homomorphismus
  $i^\text{h}\colon R\to\Rh$, s.\,d. folgende universelle Eigenschaft
  erfüllt ist:
  
  Für jeden weiteren lokalen Homomorphismus $u\colon R\to A$ in einen
  henselschen lokalen Ring $A$ gibt es einen eindeutigen lokalen
  Homomorphismus $u^\text{h}\colon\Rh\to A$, so dass das folgende Diagramm
  kommutiert
  \begin{center}
    \begin{tikzcd}
      R \arrow[r, "{u}"]\arrow[d, "{i^\text{h}}"]
      & A \\
      \Rh\arrow[ur, dashed, "{\exists!u^\text{h}}"{swap}]
    \end{tikzcd}
  \end{center}

  Die Henselisierung ist nach Definition eindeutig bis auf
  Isomorphismus und hat denselben Restklassenkörper wie $R$.
  \optcite[Chapter 2.3, S. 47]{neron}
\end{Definition}

\begin{Definition}[strikte Henselisierung]\label{def:striktehenselisierung}
  Die strikte Henselisierung eines lokalen Rings $(R,\m)$ mit
  Restklassenkörper $k=R/\m$ ist ein strikt henselscher lokaler Ring
  $\Rsh$, dessen Restklassenkörper der separable Abschluss $k_s$ von
  $k$ ist, zusammen mit einem lokalen Homomorphismus
  $i^\text{sh}\colon R\to\Rsh$, s.\,d. folgende universelle
  Eigenschaft erfüllt ist:
  
  Für jeden weiteren lokalen Homomorphismus $u\colon R\to A$ in einen
  strikt henselschen lokalen Ring $A$ mit Restklassenkörper $k_A$ und
  für jeden $k$-Homomorphismus $\alpha\colon k_s\to k_A$ gibt es einen
  eindeutigen lokalen Homomorphismus $u^\text{sh}\colon\Rsh\to A$, so
  dass das folgende Diagramm kommutiert und so dass $\alpha$ die
  Projektion von $u^\text{sh}$ auf die Restklassenkörper ist
  \begin{center}
    \begin{tikzcd}
      R \arrow[r, "{u}"]\arrow[d, "{i^\text{sh}}"]
      & A \\
      \Rsh\arrow[ur, dashed, "{\exists!u^\text{sh}}"{swap}]
    \end{tikzcd}
  \end{center}

  Die strikte Henselisierung ist nach Definition eindeutig bis auf
  Isomorphismus und hat $k_s$ als Restklassenkörper.
\end{Definition}

\begin{Satz}\label{thm:exhenselisierung}
  \cite[Proposition IV.6.5]{silverman2}
  Jeder diskrete Bewertungsring $(R,\m)$ hat eine Henselisierung $\Rh$
  und eine strikte Henselisierung $\Rsh$.
  \begin{proof}
    \cite[Proposition IV.6.5]{silverman2},
    \cite[Remark IV.6.6.2]{silverman2}
    \cite[Chapter 2.3, S. 48]{neron}
  \end{proof}
\end{Satz}

\begin{Satz}\label{thm:eigstrikthenselsch}
  \cite[Proposition IV.6.4]{silverman2}
  Sei $(R,\m)$ ein diskreter Bewertungsring mit (perfektem)
  Restklassenkörper $k=R/\m$ und $X$ glattes $R$-Schema mit spezieller
  Faser $X_k$.
  Die Einschränkung
  \begin{gather*}
    X(R)\longto X_k(k)
  \end{gather*}
  ist
  \begin{enumerate}[label=(\roman*)]
  \item surjektiv, falls $R$ henselsch ist.
  \item surjektiv mit dichtem Bild, falls $R$ strikt henselsch ist.
  \end{enumerate}
\end{Satz}

\subsection{Nachweis der Gruppenstruktur für strikt henselsche Ringe}
% ERWEITERUNG DER GRUPPENSTRUKTUR
\begin{Satz}\label{thm:egruppenschema}
  $E$ ist Gruppenschema für $R$ strikt henselsch.
\end{Satz}

\section{Allgemeine Existenz}
\begin{proof}[Beweis von \autoref{thm:exneronmodelle}]
  treuflacher Abstieg
\end{proof}



\chapter{Existenz minimaler eigentlicher regulärer Modelle für
  elliptische Kurven}

\chapter{Ausblick}
Tate's Algorithmus zur Berechnung der speziellen Faser bzw. des
Néron-Modells und des minimalen eigentlichen, regulären Modells einer
elliptischen Kurve
\cite[IV.9]{silverman2}

\appendix
\chapter{Verwendete Definitionen}
\section{Dedekindringe}
% DEDEKINDRINGE
\begin{Definition}[Dedekindring]\label{def:dedekindring}
  \optcite[S.\,40]{hartshorne}\optcite[Definition B.84]{wedhorn}
  Ein Integritätsring $R$ heißt Dedekindring, wenn er eine der
  äquivalenten Bedingungen erfüllt
  \begin{enumerate}[label=(\roman*)]
  \item regulär und $\dim(A)\leq 1$
  \item noethersch, normal und $\dim(A)\leq 1$
  \item $A$ Körper oder
    jede Lokalisierung nach einem Maximalideal ist ein diskreter
    Bewertungsring
    (d.\,h. lokal, noethersch, normal von Dimension 1)
  \item jedes echte Ideal ist endliches Produkt von Primidealen
    (diese ist dann eindeutig bis auf Reihenfolge) 
  \end{enumerate}

  Faktorielle Dedekindringe sind genau die Hauptidealringe.
  \optcite[Proposition B.85]{wedhorn}

  \begin{Definition}[Dedekindschema]
    Ein Dedekindschema ist ein noethersches, integeres Schema $X$, das
    eine der äquivalenten Eigenschaften erfüllt
    \begin{enumerate}[label=(\roman*)]
    \item die offenen affinen Unterschemata sind Spektren von Dedekindringen
    \item $\dim(X)\leq 1$ und $X$ regulär
      (in Dimension 1 äquivalent zu normal nach
      \cite[Corollary 6.39, Proposition 6.40]{wedhorn})
    \end{enumerate}
    Ein integeres Schema mit endlicher Überdeckung durch Dedekindschemata
    ist selbst Dedekindschema.
    \optcite[Chapter (7.13)]{wedhorn}
  \end{Definition}

  \begin{Bemerkung}
    Die Bedingung, dass $S$ integer sein muss, liefert einen eindeutigen
    generischen Punkt $\eta$ und einen Funktionenkörper
    $\K=\k(\eta)=\O_{S,\eta}$, so dass jeder Schnitt und jeder Halm ein
    Unterring von $\K$ ist, d.\,h. es gibt für jeden Punkt $x\in X$ einen
    Morphismus $\Spec(\K)\to\Spec(\k(x))$ und für jede offene Teilmenge
    $U\subset X$ einen Morphismus $\Spec(\K)\to\Spec(\Gamma(U,\O_X))$.

    Verzichtet man auf diese Bedingung, erhält man eine Familie
    generischer Punkte $\eta_i$ der endlich vielen irreduziblen
    Komponenten (beachte: $S$ noethersch). Dieser Fall kann analog
    behandelt werden, indem man $\K=\bigoplus_i \k(\eta_i)$ setzt.
  \end{Bemerkung}

  \begin{Bemerkung}[generische und spezielle Fasern]
    Ist $S=\Spec(R)$ mit $R$ Dedekindring, so hat ein $S$-Schema die
    \emph{generische Faser} über dem generische Punkt $\eta=(0)$ und die
    \emph{speziellen Fasern} über den verbleibenden Maximalidealen
    (geschlossenen Punkten).
  \end{Bemerkung}
\end{Definition}

\section{Eigenschaften von Schemata}
% REDUZIERT
\begin{Definition}[reduziert, geometrisch reduziert]
  Ein Schema heißt reduziert, falls alle Halme reduzierte Ringe sind
  (d.\,h. $\Nil(0)=(0)$).

  Ein $\K$-Schema heißt geometrisch reduziert, wenn für jeden
  Basiswechsel zu einer Körpererweiterung von $\K$ die generische
  Faser reduziert ist. 
\end{Definition}

% LOKAL VON ENDLICHEM TYP
\begin{Definition}[lokal von endlichem Typ]
  Ein Morphismus $f\colon X\to Y$ von Schemata heißt lokal von endlichem
  Typ, wenn lokal auf affinen offenen Teilmengen $V\subset Y$ die
  induzierte Abbildung $f_U^\#: f_\ast\O_X(V)\to \O_Y(V)$
  (wobei $U=f^{-1}(V)$) $f_\ast\O_X(V)=\O_X(U)$ zu einer \emph{endlich erzeugten}
  $\O_Y(V)$-Algebra macht.

  Er heißt von endlichem Typ, wenn er lokal von endlichem Typ und
  quasikompakt (Urbilder quasikompakter Mengen sind quasikompakt) ist.
\end{Definition}

% REGULÄR
\begin{Definition}[Regulär]
  Ein Schema $X$ heißt regulär im Punk $x\in X$, wenn sein Halm
  $\O_{X,x}$ ein regulärer (lokaler) Ring ist
  (d.\,h. $\dim_{\O_{X,x}/\m_x}(\m_x/\m_x^2)=\dim(\O_{X,x})$).

  Regularität kann auf lokal noetherschen Schemata auf den geschlossenen
  Punkten geprüft werden:
  Ist $X$ lokal noethersch und regulär in allen geschlossenen Punkten,
  so ist $X$ regulär nach \cite[Remark 6.25 (3)]{wedhorn}.
\end{Definition}

\section{Eigenschaften von Morphismen von Schemata}
% FLACH
\begin{Definition}[flacher Morphismus]
  \optcite{bosch, goodreduction}
  Ein Morphismus $f\colon X\to Y$ von Schemata ist flach, wenn alle
  induzierten Ringhomomorphismen $f_x^\#\colon \O_{Y,f(x)}\to\O_{X,x}$  auf
  den Halmen flach sind.
\end{Definition}
\begin{Bemerkung}\label{thm:flachgenerischefaser}
  Für einen Dedekindring $R$ und ein reduziertes $R$-Schema $f\colon
  X\to R$ ist flach äquivalent dazu, dass alle generischen Punkte von 
  $X$ auf den generischen Punkt von $R$ geschickt werden
  nach \cite[Proposition III.9.7]{hartshorne}.
  Insbesondere ist dann die generische Faser nichtleer und dicht in $X$.
\end{Bemerkung}

% UNVERZWEIGT
% ??? Unverzweigt <=> Glatt für diskr. BWR?
\begin{Definition}[unverzweigter Morphismus]
  Ein Morphismus $f\colon X\to S$ von Schemata heißt unverzweigt im
  Punkt $x\in X$, falls es eine offene Umgebung
  $U\subset X$ von $x$ gibt und eine geschlossene $S$-Immersion
  $j\colon U\cong\O_W/\I\to W\subset\A_S^n$ in ein offenes Unterschema
  des affinen $n$-Raums mit zugehöriger Idealgarbe $\I\subset\O_W$,
  für die gilt
  \begin{enumerate}[label=(\alph*)]
  \item $\I$ ist in einer Umgebung von $z\coloneqq j(x)$ endlich erzeugt.
  \item $\Om{\A_S^n/S}$ ist erzeugt von den $\d g\in\Om{\A_S^n/S}$
    mit $g\in\I$.
  \end{enumerate}
  $f$ heißt unverzweigt, falls es in jedem Punkt $x\in X$
  unverzweigt ist.
  \begin{Bemerkung}
    \optcite[8.4, Theorem 3]{bosch}
    Ist $X$ lokal endlich präsentiert, sind äquivalent
    \begin{enumerate}[label=(\roman*)]
    \item $X$ unverzweigt in $x$,
    \item $\Om{X/S,x}=0$,
    \item $X_{f(x)}\to \Spec(\k(f(x)))$ unverzweigt.
    \end{enumerate}
  \end{Bemerkung}
\end{Definition}

% GLATT
\begin{Definition}[glatter Morphismus]
  \optcite[8.5, Definition 1]{bosch}
  Ein Morphismus $f\colon X\to S$ von Schemata heißt glatt im Punkt
  $x\in X$ von relativer Dimension $r$, wenn es eine offene Umgebung
  $U\subset X$ von $x$ gibt und eine geschlossene Immersion $j\colon
  U\cong\O_W/\I\to W\subset\A_S^n$ in ein offenes Unterschema des affinen
  $n$-Raums mit zugehöriger Idealgarbe $\I\subset\O_W$, für die
  gilt
  \begin{enumerate}[label=(\alph*)]
  \item $\I$ wird in einer Umgebung von $z\coloneqq j(x)$ von $n-r$ Schnitten
    $g_{r+1},\dotsc,g_{n}$ erzeugt.
  \item Die $\d g_{r+1}(z),\dotsc,\d g_{n}(z)\in 
    \Om{\A_S^n/S,z} \otimes_{\O_{\A_S^n,z}} \k(z)
    \cong \Om{\A_S^n/S,z}/\m_z\Om{\A_S^n/S,z}$ 
    sind linear unabhängig über $\k(z)$.
    Dies ist äquivalent zu
    \begin{gather*}\tag{Jacobi Bedingung}
      \rk\left(
        \frac{\partial g_j}{\partial t_i}\d t_i 
      \right)_{\substack{j=r+1,\dotsc,n\\i=1,\dotsc,n\phantom{+1}}}
      = n-r
    \end{gather*}
  \end{enumerate}
  Ist $f$ glatt in jedem Punkt heißt $f$ glatt bzw. $X$ ist glattes
  $S$-Schema.  

  Für $X$ lokal endlich präsentiert ist $f$ glatt in $x$ äquivalent zu
  $f$ flach in $x$ und $X_{f(x)}$ glatt über $\k(f(x))$ nach
  \cite[8.5, Proposition 17]{bosch}.

  Ist $X$ ein $\K$-Schema lokal endlichen Typs sind glatt und
  geometrisch regulär äquivalent nach \cite[Corollary 6.32]{wedhorn}.
  Ist $\K$ perfekter Körper (z.\,B. $\Char(\K)=0$), so ist dies sogar
  äquivalent zu regulär nach \cite[Remark 6.33]{wedhorn}.

  \begin{Bemerkung}\label{thm:implikationenglatt}
    Glatt impliziert reduziert, lokal von endlichem Typ und flach
    \cite[für flach s.][Theorem 14.22]{wedhorn}.
  \end{Bemerkung}
\end{Definition}

% ÉTALE
\begin{Definition}[étaler Morphismus]% Äquivalent zu lokal umkehrbar
  \optcite[8.5, Definition 1 und Corollary 12]{bosch}  
  Ein Morphismus $f\colon X\to Y$ von $S$-Schemata heißt étale in
  einem Punkt $x\in X$, falls er eine der äquivalenten Bedingungen
  erfüllt 
  \begin{enumerate}[label=(\roman*)]
  \item $f$ glatt von relativer Dimension 0 in $x$
  \item $X$ glatt in $x$, $Y$ glatt in $f(x)$ und 
    $(f^*\Om{Y/S})_x \overset{\sim}{\longto} \Om{X/S,x}$
  \end{enumerate}
  Er heißt étale, falls er étale in jedem Punkt von $X$ ist.
  
  $f$ ist genau dann étale, wenn $f$ glatt und unverzweigt ist.
  \optcite[8.5, Proposition 6]{bosch}

  Offene Immersionen sind genau die étalen Immersionen nach
  \cite[8.5, Lemma 7]{bosch}.
\end{Definition}

% EIGENTLICH
\begin{Definition}[eigentlicher Morphismus]
  % Verallgemeinerung von projektiv; Äquivalent zu kompakt
  Ein Morphismus $f\colon X\to Y$ heißt eigentlich, wenn er eine der
  folgenden äquivalenten Bedingungen erfüllt
  \begin{enumerate}[label=(\roman*)]
  \item $f$ ist separiert, von endlichem Typ und universell
    geschlossen (d.\,h. jeder Basiswechsel ist geschlossen)
  \item $f$ ist quasi-separiert, von endlichem Typ und für jeden
    Bewertungsring $R$ gilt
    \begin{gather*}
      X(R)=\Hom_Y(R, X) 
      \overset{\sim}{\underset{\circ\Spec(\operatorname{incl})}
        {\longrightarrow}}
      \Hom_Y(\Quot(R), X)=X(\Quot(R))\;.
    \end{gather*}
    In anderen Worten, es findet sich für jedes kommutative Diagramm
    wie folgt ein entsprechender, eindeutiger Lift
    \begin{center}
      \begin{tikzcd}
        \Spec(\Quot(R)) \arrow[r]\arrow[d]
        & X \arrow[d, "f"] \\
        \Spec(R) \arrow[r]\arrow[to=ur, dashed, "\exists!"]
        & Y
      \end{tikzcd}
    \end{center}
  \end{enumerate}
  Endliche sowie projektive Morphismen sind eigentlich
  \cite[9.5, Remark 5 und Theorem 9]{bosch}.

  \begin{Bemerkung}[Stein Faktorisierung]
    \cite[9.5, Theorem 12]{bosch}
    Jeder eigentliche Morphismus $f\colon X\to Y$ faktorisiert über
    einen eigentlichen, surjektiven Morphismus $g\colon X\to
    \Spec(f_*\O_X)$ mit zusammenhängenden Fasern und einen endlichen
    Morphismus $h\colon \Spec(f_*\O_X)\to Y$.
  \end{Bemerkung}
\end{Definition}


\section{$\K$-Gruppenschemata}
% GUTE REDUKTION
\begin{Definition}[Gute Reduktion]\label{def:gutereduktion}
  % $X/S$ hat Gute Reduktion nach dem Primideal $P\in S$ heißt, dass die
  % Faser $X_P$ Glattheit erhält.
  Sei $S$ ein Dedekindring, $\K=\Quot(R)$ und $X/\K$
  ein eigentliches glattes $\K$-Schema.
  $X$ hat \emph{gute Reduktion im geschlossenen Punkt $s\in\Spec(R)$},
  falls es ein eigentliches glattes $R_s$-Schema $Y_s$ gibt, so dass
  $X=Y_s\times_{R_s}\Spec(\K)$ ist. D.\,h. folgendes Diagramm ist
  kartesisch
  \begin{center}
    \begin{tikzcd}
      X \arrow[r]\arrow[d]
      \arrow[dr, phantom, "\ulcorner", very near start]
      & Y_s \arrow[d] \\
      \Spec(\K) \arrow[r]
      & \Spec(R_s)
    \end{tikzcd}
  \end{center}
  \cite[vgl.][Chapter 1.4]{neron}

  Wir sagen $X$ hat \emph{gute Reduktion}, falls es ein eigentliches
  glattes $R$-schema $Y$ gibt, so dass $X$ isomorph zur generischen
  Faser $Y_{\K}$ von $Y$ ist. D.\,h. das folgende Diagramm ist
  kartesisch
  \begin{center}
    \begin{tikzcd}
      X \arrow[r]\arrow[d]
      \arrow[dr, phantom, "\ulcorner", very near start]
      & Y \arrow[d] \\
      \Spec(\K) \arrow[r]
      & \Spec(R)
    \end{tikzcd}
  \end{center}
  \cite[Definition 3.1]{goodreduction}, \cite[vgl.][Chapter 1.4]{neron}

  Alternative Definition für elliptische Kurven: \cite[Chapter VII.5]{silverman}
\end{Definition}

% VARIETÄT, ELLIPTISCHE KURVE
\begin{Definition}[Varietät]
  \cite[Theorem 3.37]{wedhorn}
  Eine affine Varietät über einem Körper $\K$ ist eine Unterprägarbe der
  Prägarbe $U\mapsto \Hom(U,\K)$ über den geschlossenen Punkten eines
  geschlossenen, irreduziblen Unterschemas des $\A_\K^n$.
  Eine affine Prävarietät ist eine quasikompakte, zusammenhängende
  Prägarbe, die lokal affine Varietät ist.

  Für integere $\K$-Schemata endlichen Typs sind die geschlossenen
  Punkte isomorph zu den $\K$-Schnitten 
  $\Hom_{\Spec(\K)}(\Spec(\K),X)$.
  Dadurch gibt es eine Kategorienäquivalenz
  \begin{align*}
    \left\{\text{Integere $\K$-Schemata endlichen Typs}\right\}
    &\longto
      \left\{\text{Prävarietäten über $\K$}\right\}\\
    (X,\O_X) 
    &\longmapsto
      \left( X(\K), \O_{X(\K)}\subset \Hom(\bullet, \K) \right)
  \end{align*}
  wobei $f\in\O_{X(\K)}(U)$ der Abbildung 
  \begin{align*}
    U\cap X(\K) &\longto \K\\
    x  &\longmapsto f(x)
         \coloneqq \bar{f_x}\in\O_{X,x}/\m_x\cong \K
  \end{align*}
  entspricht (Analogon zum Einsetzhomomorphismus für Polynomringe).
  Der inverse Funktor ist gegeben durch
  \begin{align*}
    (P,\O_P) \longmapsto (t(P), t_\ast\O_P) 
  \end{align*}
  wobei $t\colon P\to t(P)$, $x\mapsto \{x\}$. Auf affinen Varietäten
  $V$ wird dies zu
  \begin{gather*}
    (V, \O_V) \mapsto (\Spec(\O_V(V)), \O_{\Spec(\O_V(V))})
  \end{gather*}

  Daher werden wir im folgenden mit Prävarietät ein integeres
  $\K$-Schema endlichen Typs meinen und mit affiner Varietät ein affines,
  integeres $\K$-Schema endlichen Typs.

  Eine algebraische Varietät sei im folgenden ein separiertes, integeres
  Schema über einem algebraisch abgeschlossen Körper $\K$
  (alternativ: über einem Körper $\K$, wobei die Basiserweiterung auf
  den algebraischen Abschluss weiterhin integer ist).
\end{Definition}
\begin{Definition}[Kurve, elliptische Kurve]
  Eine Kurve über einem Körper $\K$ ist eine projektive Varietät der
  Dimension 1,
  also ein projektives, integeres $\K$-Schema endlichen Typs.

  Eine elliptische Kurve über $\K$ ist eine reguläre Kurve $E$ vom
  Genus 1 zusammen mit einem $\K$-Schnitt $O\in E(\K)$.
  Elliptische Kurven sind glatte \cite[Proposition III.3.1]{silverman}
  $\K$"=Gruppenschemata mit Nullelement $O$ \cite{silverman}.
\end{Definition}


\nocite{*}
\printbibliography
\end{document}
