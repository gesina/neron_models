\chapter{Beispiele und Ausblick}\label{chap:ausblick}
Inzwischen ist aus \ref{bsp:regweierstraßmodell} und
\ref{bem:konstruktionminmodell} bekannt, dass ein reguläres
Weierstraßmodell über einem Dedekindring bereits minimales Modell
seiner generischen Faser ist.
In einigen Fällen kann die Regularität an den
Koeffizienten der Weierstraßgleichung abgelesen werden.
Sei im Folgenden $(R,(\pi))$ diskreter Bewertungsring mit perfektem
Restklassenkörper, $\K=\Quot(R)$, $E_\K$ elliptische Kurve über $\K$
und $W$ ein minimales Weierstraßmodell von $E_\K$ zur Gleichung
($\Char(\K)\neq2,3$):
\begin{gather}\label{eq:allgweierstraßgleichung}
  f\colon y^2=x^3 + \beta x + \gamma
\end{gather}
% \begin{Beispiel}[$b_n$-Reduktion: $\ord_\K(\Delta)=1$]
%   \optcite[Lemma IV.9.5]{silverman2}
%   Ist $\ord_\K(\Delta)=1$, sind nach
%   \cite[Chapter 1.5, Lemma 3 and Lemma 4]{neron} die Koeffizienten
%   $\beta$ und $\gamma$ der Gleichung Einheiten in $R$. Eine Rechnung
%   wie in \cite[Lemma IV.9.5]{silverman2} zeigt, dass $W$ dann regulär
%   im singulären Punkt $(x,y,\pi)\subset R[x,y]/(f)$ ist.
% \end{Beispiel}

\begin{Korollar}\label{thm:c1red}
  % $\ord_\K(j)\geq0$]
  \label{bem:bedregweierstraßmodell}
  Eine hinreichende Bedingung dafür, dass $W$ regulär ist, ist
  \begin{gather*}
    \ord_\K(\beta)\geq 1
    \qquad\text{und}\qquad
    \ord_\K(\gamma)=1
  \end{gather*}
  das heißt $\gamma=u\pi$, $\beta=v\pi^n$ für Einheiten $u,v\in R^\times$
  und $n\in\N_{\geq1}$.

  \begin{proof}
    Wie aus \ref{thm:weierstrassglatt} bekannt, ist nur die
    Regularität des singulären Punkts der speziellen Faser zu
    überprüfen.
    Sei $\m_\pi\coloneqq(\pi)$ und $k=(R_{\m_\pi}/\m_\pi)$ der
    Restklassenkörper von $R$.
    Die spezielle Faser $W_k$ ist auf dem betrachteten Teil seiner
    affinen Überdeckung das Spektrum von 
    \begin{gather*}
      \dfrac{k[\bar x,\bar y]}{(\bar y^2-\bar x^3+\beta \bar x+\gamma)}
      =\dfrac{k[\bar x,\bar y]}{(\bar y^2-\bar x^3+v\pi^n \bar x+u\pi x)}
      =\dfrac{k[\bar x,\bar y]}{(\bar y^2-\bar x^3)}
    \end{gather*}
    welches im abgeschlossenen Punkt $(0,0)\in\A_k^2$ bzw.
    $(\bar x,\bar y)\in W_k$ singulär ist. Da der singuläre Punkt von
    $W_k$ eindeutig ist, muss nur dieser überprüft werden.
    Er entspricht in $W$ dem Maximalideal
    $\p\coloneqq(\pi,x,y)\subset R[x,y]/(f)$.
    Für Regularität gilt zu zeigen, dass $\p$ in seinem Halm
    $(R[x,y]/(f))_\p=\O_{W,\p}$ nur von zwei Elementen erzeugt wird.
    Durch Umstellen von \eqref{eq:allgweierstraßgleichung} ist aber
    ersichtlich, dass
    \begin{gather*}
      \pi = u^{-1}\left( y^2-x^3-\beta x \right)\in\p
    \end{gather*}
    weshalb $\p=(x,y)$ tatsächlich nur zwei Erzeuger hat.
  \end{proof}
\end{Korollar}

\begin{Bemerkung}
Nach \cite[Chapter 1.5, Lemma 3 and Lemma 4]{neron} ist die
Bedingung $\ord_\K(\beta)\geq 1$ und $\ord_\K(\gamma)=1$ für
minimale Weierstraßgleichungen äquivalent dazu,
dass $\ord_\K(\Delta)=2$ und $\ord_\K(j)\geq 1$,
wobei $j=2^8\cdot 3^3\cdot\beta^3$ die $j$-Invariante der
Weierstraßgleichung ist.
\end{Bemerkung}

Wie in \cite[Chapter 1.5]{neron} beschrieben, ist die
Gruppenstruktur der speziellen Faser im Fall von \ref{thm:c1red} die
des additiven Gruppenschemas $\G_{a,k}$ über $k$. Dieses ist definiert
als $\G_{a}\times_\Z k$, wobei $\G_{a}\coloneqq\Spec(\Z[T])$ ist mit
Gruppenstruktur
\begin{center}
  \begin{tikzcd}[row sep=5pt]
    \G_{a}\times_\Z \G_{a} \arrow[r,"\mu"] \arrow[d, equal]
    & \G_{a}\arrow[ddd, equal]\\
    \left(\Spec\Z[T_1]\right) \times_\Z
    \left(\Spec\Z[T_2]\right) \arrow[d, equal] \\
    \Spec\left(\Z[T_1]\otimes_\Z\Z[T_2]\right) \arrow[d,equal]\\
    \Spec\Z[T_1,T_2] \arrow[r, "\mu"]
    &\Spec(\Z[T])\\
    T_1+T_2\arrow[from=r, mapsto]
    & T
  \end{tikzcd}
\end{center}
    
\begin{Beispiel}
  Sei $R=\Z_{(5)}$ und betrachte die Weierstraßgleichungen
  \begin{align*}
    y^2&=x^3+5x\pm 10
    & \Delta&=-16\cdot3200=-16\cdot2^7\cdot5^2\\
    y^2&=x^3\pm 5
    & \Delta&=-16\cdot675=-16\cdot3^3\cdot5^2\\
    y^2&=x^3\pm 10
    & \Delta&=-16\cdot2700=-16\cdot2^2\cdot3^3\cdot5^2
  \end{align*}
  Für alle ist offensichtlich die Bedingung aus
  \ref{bem:bedregweierstraßmodell} erfüllt.
  Somit sind die projektiven Abschlüsse der elliptischen Kurven über
  $\Z_{(5)}$ alle bereits minimale Modelle und ihr glatter Teil,
  das heißt der Punkt $(x,y,5)$ ausgenommen, ist das jeweilige
  Néron-Modell.
\end{Beispiel}

\begin{Beispiel}
  Ein Anwendungsbeispiel für die Aussage aus
  \ref{thm:teilwabelscheneronmodelle} ist jetzt besonders ersichtlich:
  Bei einem regulären Weierstraßmodell $W$ ist die induzierte
  Gruppenstruktur auf dem glatten Teil in einer Umgebung der glatten
  Fasern von $W$ abelsch.
  Dies liegt daran, dass die glatten Fasern von $W$ bereits
  elliptische Kurven sind.
\end{Beispiel}

% TODO: Tate's alg
Wie zuvor bereits angedeutet, ist es möglich, die minimalen Modelle
elliptischer Kurven anhand ihrer Weierstraßgleichung nach der Form
ihrer speziellen Faser in zehn verschiedenen Kategorien zu
klassifizieren \cite[vgl.][Theorem IV.8.2]{silverman2}.
Zur Klassifizierung kann ein Algorithmus von Tate verwendet werden,
der von einer Weierstraßgleichung der elliptischen Kurve ausgeht, diese
minimalisiert und dann einer der Kategorien in
\autoref{tab:faserklassifizierung} zuordnet.
\begin{table}[tbhp]
  \begin{center}
    \begin{tabularx}{\linewidth}{@{}l>{\hsize=.7\hsize}X>{\hsize=1.3\hsize}X@{}}
      \toprule
       & Bedingungen an \newline die Koeffizienten
      & ~\newline Beschreibung der speziellen Faser\\
      \midrule[\heavyrulewidth]
      $a$& $\ord_\K(j)\geq0$, $\ord_\K(\Delta)=0$
      & elliptische Kurve\\\midrule
      $b_n$& ${\ord_\K(j)<0}$, \newline
      ${\ord_\K(\beta)=0}$, ${\ord_\K(\gamma)=0}$
      & $n=1$: Kurve mit Knoten\newline
        $n>1$: $n$ reguläre Kurven, die sich schneiden \\\midrule
      $c1$& $\ord_\K(j)\geq0$, $\ord_\K(\Delta)=2$
      & singuläre Kurve mit Scheitelpunkt\\\midrule
      $c2$& $\ord_\K(j)\geq0$, $\ord_\K(\Delta)=3$
      & 2 reguläre Kurven mit 1 tangentialem Schnittpunkt \\\midrule
      $c3$& $\ord_\K(j)\geq0$, $\ord_\K(\Delta)=4$
      & 3 reguläre Kurven mit 1 gem. Schnittpunkt \\\midrule
      $c4$& $\ord_\K(j)\geq0$, $\ord_\K(\Delta)=6$
      & 1 reguläre Kurve mit Multiplizität 2, die 4 weitere reguläre
      Kurven schneidet\\\midrule
      $c5_n$& ${\ord_\K(j)<0}$, \newline
      ${\ord_\K(\beta)=2}$, ${\ord_\K(\gamma)=3}$
      & $n$ reguläre Kurven mit Multiplizität 2 bilden eine Kette, die
        an den 2 Enden jeweils eine Kurve der Multiplizität 1 schneidet\\\midrule
      $c6$& $\ord_\K(j)\geq0$, $\ord_\K(\Delta)=8$
      & 1 reguläre Kurve der Multiplizität 3 schneidet 3 Kurven mit
        Multiplizität 2, die jeweils 1 mit Multiplizität 1 schneiden\\\midrule
      $c7$& $\ord_\K(j)\geq0$, $\ord_\K(\Delta)=9$
      & ähnlich $c6$ mit 8 Kurven und bis Multiplizität 4 \\\midrule
      $c8$& $\ord_\K(j)\geq0$, $\ord_\K(\Delta)=10$
      & ähnlich $c6$ mit 9 Kurven und bis Multiplizität 6\\\bottomrule
    \end{tabularx}
  \end{center}
  \caption{\label{tab:faserklassifizierung} Kategorien für die
    Klassifizierung der speziellen Faser des minimalen Modells einer
    elliptischen Kurve über einem diskreten Bewertungsring mit
    perfektem Restklassenkörper,
    nach Néron,
    vergleiche \cite[Chapter 1.5]{neron} und \cite[Figure IV.4.4]{silverman2}} 
\end{table}
Für genaueres zu Tates Algorithmus siehe
\cite[Chapter IV.9]{silverman2},
\cite[Chapter 1.5]{neron} und \cite{tate}.
Ein analoger Algorithmus, der auch den Fall von Dedekindringen mit
nicht perfekten Restklassenkörpern abdeckt, ist in \cite{nonperfect}
zu finden.


%%% Local Variables:
%%% mode: latex
%%% TeX-master: "neron_models"
%%% End:
