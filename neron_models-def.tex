\KOMAoptions{parskip=half}

% PACKAGES
% \usepackage{mathpazo}
\usepackage{fontspec}
%\setmainfont{Latin Modern Roman}
\setmainfont{Charis SIL}
\usepackage{babel}
\usepackage[autostyle=try]{csquotes}

\usepackage[style=alphabetic, backend=biber]{biblatex}
\bibliography{neron_models.bib}

\usepackage{mathtools}
\usepackage{amssymb}
\usepackage{amsfonts}
\usepackage{amsthm}
\usepackage{dsfont}
\usepackage{tikz-cd}
\usetikzlibrary{babel}
\hyphenation{wo-raus}

\usepackage{enumitem}

\usepackage[colorlinks=true, linkcolor=blue]{hyperref}


% SONSTIGE DEFINITIONEN
\newcommand*{\NAbbEig}{Néron-Abbildungseigenschaft }

% MATHE DEFINITIONEN
\newtheorem{Satz}{Satz}[chapter]
\newtheorem{Theorem}[Satz]{Theorem}
\newtheorem{Proposition}[Satz]{Proposition}
\newtheorem{Lemma}[Satz]{Lemma}
\newtheorem{Korollar}[Satz]{Korollar}
\theoremstyle{definition}
\newtheorem{Definition}[Satz]{Definition}
\newtheorem{LemmaDefinition}[Satz]{Lemma/Definition}
\newtheorem{Beispiel}[Satz]{Beispiel}
%\theoremstyle{remark}
\newtheorem{Bemerkung}[Satz]{Bemerkung}

\newcommand*{\N}{\mathds{N}}
\newcommand*{\Z}{\mathds{Z}}
\newcommand*{\K}{\ensuremath{K}} % Körper
\newcommand*{\algK}{\ensuremath{\overline K}} % algebraischer
\renewcommand*{\Rsh}{\ensuremath{R^{\text{sh}}}} % (strikte) Henselisierung
\newcommand*{\Ksh}{\ensuremath{\K^{\text{sh}}}} % (strikte) Henselisierung
% Abschluss
\renewcommand*{\Im}{\text{im}} % Bild
\newcommand*{\proj}{\text{proj}} % Projektion
\newcommand*{\dra}{\dashrightarrow} % rationale Abbildung
\DeclareMathOperator{\Dom}{Dom} % Definitionsbereich einer rat. Abb.
\DeclareMathOperator{\Quot}{Quot} % Quotientenkörperoperator
\DeclareMathOperator{\Nil}{Nil} % Nilradikal
\DeclareMathOperator{\Spec}{Spec} % Spektrum
\newcommand*{\A}{\ensuremath{\mathds{A}}} % affiner Raum
\renewcommand*{\P}{\ensuremath{\mathds{P}}} % projektiver Raum
\renewcommand*{\m}{\mathfrak{m}} % Maximalideal
\renewcommand*{\k}{\kappa} % Restklassenkörperoperator
\newcommand*{\n}{\eta} % generischer Punkt
\renewcommand*{\O}{\mathcal{O}} % Garbe
\newcommand*{\I}{\mathcal{I}} % Idealgarbe
\newcommand*{\Om}[1]{\ensuremath{\Omega_{#1}^1}} % relative 1-Formen
\renewcommand*{\d}{\ensuremath{\operatorname{d}}} % (universelles) Differential
\DeclareMathOperator{\rk}{rk} % Rang (einer Matrix)
\newcommand*{\longto}{\longrightarrow}
\newcommand*{\longfrom}{\longleftarrow}
%\newcommand*{\Xn}{\underline{X}} % Variablenmenge
\newcommand*{\degs}{\operatorname{\deg}_s} % Separabilitätsgrad
\newcommand*{\degi}{\operatorname{\deg}_i} % Inseparabilitätsgrad
\DeclareMathOperator{\ord}{ord} % diskrete Bewertung
\DeclareMathOperator{\Div}{Div} % Divisorengruppe
\DeclareMathOperator{\Pic}{Pic} % Picard Gruppe
\renewcommand{\div}{\operatorname{div}}
\DeclareMathOperator{\Char}{char} % Charakteristik
\DeclareMathOperator{\Hom}{Hom} % Homomorphismenmenge
\DeclareMathOperator{\End}{End} % Endomorphismenmenge
\DeclareMathOperator{\Aut}{Aut} % Automorphismenmenge
\newcommand{\Id}{\mathrm{id}} % Identitätsabbildung
\DeclareMathOperator{\im}{im} % image
%\newcommand{\tlambda}{\tilde\lambda} % alternative Abbildung \lambda
%\newcommand{\F}{F} % endliche Untergruppe
\DeclareMathOperator{\genus}{genus} % Kurvengeschlecht
\newcommand*{\Groups}{\text{(Groups)}} % Kategorie der Gruppen
\newcommand*{\Sch}[1][S]{\text{(Sch/#1)}} % Kategorie der #1-Schemata


%%% Local Variables:
%%% mode: latex
%%% TeX-master: "neron_models"
%%% End:
