\appendix
\chapter{Verwendete Definitionen}
\section{Dedekindringe}
% DEDEKINDRINGE
\begin{Definition}[Dedekindring]\label{def:dedekindring}
  \optcite[S.\,40]{hartshorne}\optcite[Definition B.84]{wedhorn}
  Ein Integritätsring $R$ heißt Dedekindring, wenn er eine der
  äquivalenten Bedingungen erfüllt
  \begin{enumerate}[label=(\roman*)]
  \item regulär und $\dim(A)\leq 1$
  \item noethersch, normal und $\dim(A)\leq 1$
  \item $A$ Körper oder
    jede Lokalisierung nach einem Maximalideal ist ein diskreter
    Bewertungsring
    (d.\,h. lokal, noethersch, normal von Dimension 1)
  \item jedes echte Ideal ist endliches Produkt von Primidealen
    (diese ist dann eindeutig bis auf Reihenfolge) 
  \end{enumerate}
\end{Definition}
Dedekindringe sind genau die Hauptidealringe.
  \optcite[Proposition B.85]{wedhorn}
  
  \begin{Definition}[Dedekindschema]
    Ein Dedekindschema ist ein noethersches, integeres Schema $X$, das
    eine der äquivalenten Eigenschaften erfüllt
    \begin{enumerate}[label=(\roman*)]
    \item die offenen affinen Unterschemata sind Spektren von Dedekindringen
    \item $\dim(X)\leq 1$ und $X$ regulär
      (in Dimension 1 äquivalent zu normal nach
      \cite[Corollary 6.39, Proposition 6.40]{wedhorn})
    \end{enumerate}
    Ein integeres Schema mit endlicher Überdeckung durch Dedekindschemata
    ist selbst Dedekindschema.
    \optcite[Chapter (7.13)]{wedhorn}
  \end{Definition}

\begin{Bemerkung}
  Die Bedingung, dass $S$ integer sein muss, liefert einen eindeutigen
  generischen Punkt $\eta$ und einen Funktionenkörper
  $\K=\k(\eta)=\O_{S,\eta}$, so dass jeder Schnitt und jeder Halm ein
  Unterring von $\K$ ist, d.\,h. es gibt für jeden Punkt $x\in X$ einen
  Morphismus $\Spec(\K)\to\Spec(\k(x))$ und für jede offene Teilmenge
  $U\subset X$ einen Morphismus $\Spec(\K)\to\Spec(\Gamma(U,\O_X))$.
  
  Verzichtet man auf diese Bedingung, erhält man eine Familie
  generischer Punkte $\eta_i$ der endlich vielen irreduziblen
  Komponenten (beachte: $S$ noethersch). Dieser Fall kann analog
  behandelt werden, indem man $\K=\bigoplus_i \k(\eta_i)$ setzt.
\end{Bemerkung}

\begin{Bemerkung}[generische und spezielle Fasern]
  Ist $S=\Spec(R)$ mit $R$ Dedekindring, so hat ein $S$-Schema die
  \emph{generische Faser} über dem generische Punkt $\eta=(0)$ und die
  \emph{speziellen Fasern} über den verbleibenden Maximalidealen
  (geschlossenen Punkten).
\end{Bemerkung}


\section{Eigenschaften von Schemata}
% REDUZIERT
\begin{Definition}[reduziert, geometrisch reduziert]
  Ein Schema heißt reduziert, falls alle Halme reduzierte Ringe sind
  (d.\,h. $\Nil(0)=(0)$).

  Ein $\K$-Schema heißt geometrisch reduziert, wenn für jeden
  Basiswechsel zu einer Körpererweiterung von $\K$ die generische
  Faser reduziert ist. 
\end{Definition}

% LOKAL VON ENDLICHEM TYP
\begin{Definition}[lokal von endlichem Typ]
  Ein Morphismus $f\colon X\to Y$ von Schemata heißt lokal von endlichem
  Typ, wenn lokal auf affinen offenen Teilmengen $V\subset Y$ die
  induzierte Abbildung $f_U^\#: f_\ast\O_X(V)\to \O_Y(V)$
  (wobei $U=f^{-1}(V)$) $f_\ast\O_X(V)=\O_X(U)$ zu einer
  \emph{endlich erzeugten} $\O_Y(V)$-Algebra macht.

  Er heißt von endlichem Typ, wenn er lokal von endlichem Typ und
  quasikompakt (Urbilder quasikompakter Mengen sind quasikompakt) ist.
\end{Definition}

% REGULÄR
\begin{Definition}[Regulär]
  Ein Schema $X$ heißt regulär im Punk $x\in X$, wenn sein Halm
  $\O_{X,x}$ ein regulärer (lokaler) Ring ist
  (d.\,h. $\dim_{\O_{X,x}/\m_x}(\m_x/\m_x^2)=\dim(\O_{X,x})$).

  Regularität kann auf lokal noetherschen Schemata auf den geschlossenen
  Punkten geprüft werden:
  Ist $X$ lokal noethersch und regulär in allen geschlossenen Punkten,
  so ist $X$ regulär nach \cite[Remark 6.25 (3)]{wedhorn}.
\end{Definition}

\section{Eigenschaften von Morphismen von Schemata}
% FLACH
\begin{Definition}[flacher Morphismus]
  \optcite{bosch, goodreduction}
  Ein Morphismus $f\colon X\to Y$ von Schemata ist flach, wenn alle
  induzierten Ringhomomorphismen $f_x^\#\colon \O_{Y,f(x)}\to\O_{X,x}$  auf
  den Halmen flach sind.
\end{Definition}
\begin{Bemerkung}\label{thm:flachgenerischefaser}
  Für einen Dedekindring $R$ und ein reduziertes $R$-Schema $f\colon
  X\to R$ ist flach äquivalent dazu, dass alle generischen Punkte von 
  $X$ auf den generischen Punkt von $R$ geschickt werden
  nach \cite[Proposition III.9.7]{hartshorne}.
  Insbesondere ist dann die generische Faser nichtleer und dicht in $X$.
\end{Bemerkung}

% UNVERZWEIGT
% ??? Unverzweigt <=> Glatt für diskr. BWR?
\begin{Definition}[unverzweigter Morphismus]
  Ein Morphismus $f\colon X\to S$ von Schemata heißt unverzweigt im
  Punkt $x\in X$, falls es eine offene Umgebung
  $U\subset X$ von $x$ gibt und eine geschlossene $S$-Immersion
  $j\colon U\cong\O_W/\I\to W\subset\A_S^n$ in ein offenes Unterschema
  des affinen $n$-Raums mit zugehöriger Idealgarbe $\I\subset\O_W$,
  für die gilt
  \begin{enumerate}[label=(\alph*)]
  \item $\I$ ist in einer Umgebung von $z\coloneqq j(x)$ endlich erzeugt.
  \item $\Om{\A_S^n/S}$ ist erzeugt von den $\d g\in\Om{\A_S^n/S}$
    mit $g\in\I$.
  \end{enumerate}
  $f$ heißt unverzweigt, falls es in jedem Punkt $x\in X$
  unverzweigt ist.
  \begin{Bemerkung}
    \optcite[8.4, Theorem 3]{bosch}
    Ist $X$ lokal endlich präsentiert, sind äquivalent
    \begin{enumerate}[label=(\roman*)]
    \item $X$ unverzweigt in $x$,
    \item $\Om{X/S,x}=0$,
    \item $X_{f(x)}\to \Spec(\k(f(x)))$ unverzweigt.
    \end{enumerate}
  \end{Bemerkung}
\end{Definition}

% GLATT
\begin{Definition}[glatter Morphismus]
  \optcite[8.5, Definition 1]{bosch}
  Ein Morphismus $f\colon X\to S$ von Schemata heißt glatt im Punkt
  $x\in X$ von relativer Dimension $r$, wenn es eine offene Umgebung
  $U\subset X$ von $x$ gibt und eine geschlossene Immersion $j\colon
  U\cong\O_W/\I\to W\subset\A_S^n$ in ein offenes Unterschema des affinen
  $n$-Raums mit zugehöriger Idealgarbe $\I\subset\O_W$, für die
  gilt
  \begin{enumerate}[label=(\alph*)]
  \item $\I$ wird in einer Umgebung von $z\coloneqq j(x)$ von $n-r$ Schnitten
    $g_{r+1},\dotsc,g_{n}$ erzeugt.
  \item Die $\d g_{r+1}(z),\dotsc,\d g_{n}(z)\in 
    \Om{\A_S^n/S,z} \otimes_{\O_{\A_S^n,z}} \k(z)
    \cong \Om{\A_S^n/S,z}/\m_z\Om{\A_S^n/S,z}$ 
    sind linear unabhängig über $\k(z)$.
    Dies ist äquivalent zu
    \begin{gather*}\tag{Jacobi Bedingung}
      \rk\left(
        \frac{\partial g_j}{\partial t_i}\d t_i 
      \right)_{\substack{j=r+1,\dotsc,n\\i=1,\dotsc,n\phantom{+1}}}
      = n-r
    \end{gather*}
  \end{enumerate}
  Ist $f$ glatt in jedem Punkt heißt $f$ glatt bzw. $X$ ist glattes
  $S$-Schema.  
\end{Definition}
\begin{Bemerkung}\label{thm:eigglatt}
  Für $X$ lokal endlich präsentiert ist $f$ glatt in $x$ äquivalent zu
  $f$ flach in $x$ und $X_{f(x)}$ glatt über $\k(f(x))$ nach
  \cite[8.5, Proposition 17]{bosch}.
  
  Ist $X$ ein $\K$-Schema lokal endlichen Typs sind glatt und
  geometrisch regulär äquivalent nach \cite[Corollary 6.32]{wedhorn}.
  Ist $\K$ perfekter Körper (z.\,B. $\Char(\K)=0$), so ist dies sogar
  äquivalent zu regulär nach \cite[Remark 6.33]{wedhorn}.
    
  Glatt impliziert reduziert, lokal von endlichem Typ und flach
  \cite[für flach s.][Theorem 14.22]{wedhorn}.
\end{Bemerkung}


% ÉTALE
\begin{Definition}[étaler Morphismus]% Äquivalent zu lokal umkehrbar
  \optcite[8.5, Definition 1 und Corollary 12]{bosch}  
  Ein Morphismus $f\colon X\to Y$ von $S$-Schemata heißt étale in
  einem Punkt $x\in X$, falls er eine der äquivalenten Bedingungen
  erfüllt 
  \begin{enumerate}[label=(\roman*)]
  \item $f$ glatt von relativer Dimension 0 in $x$
  \item $X$ glatt in $x$, $Y$ glatt in $f(x)$ und 
    $(f^*\Om{Y/S})_x \overset{\sim}{\longto} \Om{X/S,x}$
  \end{enumerate}
  Er heißt étale, falls er étale in jedem Punkt von $X$ ist.
  
  $f$ ist genau dann étale, wenn $f$ glatt und unverzweigt ist.
  \optcite[8.5, Proposition 6]{bosch}

  Offene Immersionen sind genau die étalen Immersionen nach
  \cite[8.5, Lemma 7]{bosch}.
\end{Definition}

SEPARIERT
\begin{Definition}[separierter Morphismus]
  \cite[Theorem 15.8]{wedhorn}
  
  Unterschemata separierter Schemata sind wieder separiert:
  Sei $S$ Schema, $X$ separiertes $S$-Schema,
  $W\overset{\iota}\hookrightarrow X$ Unterschema.
  Dann ist
  \begin{center}
    \begin{tikzcd}
      W\arrow[r, hook]\arrow[d, "{\Delta}"]
      & X\arrow[d, "{\Delta}"]\\
      W\times_R W \arrow[r, hook] X\times_R X
    \end{tikzcd}
  \end{center}
  kartesisch, denn es kommutiert und für jedes weitere $V$, so dass
  \begin{center}
    \begin{tikzcd}
      V\arrow[r, "\phi"]\arrow[d, "{(\phi_1,\phi_2)}"]
      & X\arrow[d, "{\Delta}"]\\
      W\times_R W \arrow[r, hook] X\times_R X
    \end{tikzcd}
  \end{center}
  ebenfalls kommutiert bzw. $\iota\circ\phi_1=\phi=\iota\circ\phi_2$,
  muss wegen der Injektivität von $\iota$ gelten $\phi_1=\phi_2$ und
  $\phi_1$ lässt 
  
  http://math.stackexchange.com/questions/1088879/closed-immersion-of-separated-scheme-is-separated
  \end{Definition}

% EIGENTLICH
\begin{Definition}[eigentlicher Morphismus]
  % Verallgemeinerung von projektiv; Äquivalent zu kompakt
  Ein Morphismus $f\colon X\to Y$ heißt eigentlich, wenn er eine der
  folgenden äquivalenten Bedingungen erfüllt
  \begin{enumerate}[label=(\roman*)]
  \item $f$ ist separiert, von endlichem Typ und universell
    geschlossen (d.\,h. jeder Basiswechsel ist geschlossen)
  \item $f$ ist quasi-separiert, von endlichem Typ und für jeden
    Bewertungsring $R$ gilt
    \begin{gather*}
      X(R)=\Hom_Y(R, X) 
      \overset{\sim}{\underset{\circ\Spec(\operatorname{incl})}
        {\longrightarrow}}
      \Hom_Y(\Quot(R), X)=X(\Quot(R))\;.
    \end{gather*}
    In anderen Worten, es findet sich für jedes kommutative Diagramm
    wie folgt ein entsprechender, eindeutiger Lift
    \begin{center}
      \begin{tikzcd}
        \Spec(\Quot(R)) \arrow[r]\arrow[d]
        & X \arrow[d, "f"] \\
        \Spec(R) \arrow[r]\arrow[to=ur, dashed, "\exists!"]
        & Y
      \end{tikzcd}
    \end{center}
  \end{enumerate}
\end{Definition}
\begin{Bemerkung}
  Endliche sowie projektive Morphismen sind eigentlich
  \cite[9.5, Remark 5 und Theorem 9]{bosch}.
\end{Bemerkung}
\begin{Bemerkung}[Stein Faktorisierung]
  \cite[9.5, Theorem 12]{bosch}
  Jeder eigentliche Morphismus $f\colon X\to Y$ faktorisiert über
  einen eigentlichen, surjektiven Morphismus $g\colon X\to
  \Spec(f_*\O_X)$ mit zusammenhängenden Fasern und einen endlichen
  Morphismus $h\colon \Spec(f_*\O_X)\to Y$.
\end{Bemerkung}


\section{$\K$-Gruppenschemata}
% GUTE REDUKTION
\begin{Definition}[Gute Reduktion]\label{def:gutereduktion}
  % $X/S$ hat Gute Reduktion nach dem Primideal $P\in S$ heißt, dass die
  % Faser $X_P$ Glattheit erhält.
  Sei $S$ ein Dedekindring, $\K=\Quot(R)$ und $X/\K$
  ein eigentliches glattes $\K$-Schema.
  $X$ hat \emph{gute Reduktion im geschlossenen Punkt $s\in\Spec(R)$},
  falls es ein eigentliches glattes $R_s$-Schema $Y_s$ gibt, so dass
  $X=Y_s\times_{R_s}\Spec(\K)$ ist. D.\,h. folgendes Diagramm ist
  kartesisch
  \begin{center}
    \begin{tikzcd}
      X \arrow[r]\arrow[d]
      \arrow[dr, phantom, "\ulcorner", very near start]
      & Y_s \arrow[d] \\
      \Spec(\K) \arrow[r]
      & \Spec(R_s)
    \end{tikzcd}
  \end{center}
  \cite[vgl.][Chapter 1.4]{neron}

  Wir sagen $X$ hat \emph{gute Reduktion}, falls es ein eigentliches
  glattes $R$-schema $Y$ gibt, so dass $X$ isomorph zur generischen
  Faser $Y_{\K}$ von $Y$ ist. D.\,h. das folgende Diagramm ist
  kartesisch
  \begin{center}
    \begin{tikzcd}
      X \arrow[r]\arrow[d]
      \arrow[dr, phantom, "\ulcorner", very near start]
      & Y \arrow[d] \\
      \Spec(\K) \arrow[r]
      & \Spec(R)
    \end{tikzcd}
  \end{center}
  \cite[Definition 3.1]{goodreduction}, \cite[vgl.][Chapter 1.4]{neron}

  Alternative Definition für elliptische Kurven: \cite[Chapter VII.5]{silverman}
\end{Definition}

VARIETÄT, ELLIPTISCHE KURVE
\begin{Definition}[Varietät]
 \cite[Theorem 3.37]{wedhorn}
  Eine affine Varietät über einem Körper $\K$ ist eine Unterprägarbe der
  Prägarbe $U\mapsto \Hom(U,\K)$ über den geschlossenen Punkten eines
  geschlossenen, irreduziblen Unterschemas des $\A_\K^n$.
  Eine affine Prävarietät ist eine quasikompakte, zusammenhängende
  Prägarbe, die lokal affine Varietät ist.

  Für integere $\K$-Schemata endlichen Typs sind die geschlossenen
  Punkte isomorph zu den $\K$-Schnitten 
  $\Hom_{\Spec(\K)}(\Spec(\K),X)$.
  Dadurch gibt es eine Kategorienäquivalenz
  \begin{align*}
    \left\{\text{Integere $\K$-Schemata endlichen Typs}\right\}
    &\longto
      \left\{\text{Prävarietäten über $\K$}\right\}\\
    (X,\O_X) 
    &\longmapsto
      \left( X(\K), \O_{X(\K)}\subset \Hom(\bullet, \K) \right)
  \end{align*}
  wobei $f\in\O_{X(\K)}(U)$ der Abbildung 
  \begin{align*}
    U\cap X(\K) &\longto \K\\
    x  &\longmapsto f(x)
         \coloneqq \bar{f_x}\in\O_{X,x}/\m_x\cong \K
  \end{align*}
  entspricht (Analogon zum Einsetzhomomorphismus für Polynomringe).
  Der inverse Funktor ist gegeben durch
  \begin{align*}
    (P,\O_P) \longmapsto (t(P), t_\ast\O_P) 
  \end{align*}
  wobei $t\colon P\to t(P)$, $x\mapsto \{x\}$. Auf affinen Varietäten
  $V$ wird dies zu
  \begin{gather*}
    (V, \O_V) \mapsto (\Spec(\O_V(V)), \O_{\Spec(\O_V(V))})
  \end{gather*}

  Wir werden im folgenden mit Prävarietät ein integeres
  $\K$-Schema endlichen Typs meinen und mit affiner Varietät ein
  affines, integeres $\K$-Schema endlichen Typs.

  Eine algebraische Varietät sei im folgenden ein separiertes, integeres
  Schema über einem algebraisch abgeschlossen Körper $\K$
  (alternativ: über einem Körper $\K$, wobei die Basiserweiterung auf
  den algebraischen Abschluss weiterhin integer ist).
\end{Definition}
\begin{Definition}[Kurve, elliptische Kurve]
  Eine Kurve über einem Körper $\K$ ist eine projektive Varietät der
  Dimension 1,
  also ein projektives, integeres $\K$-Schema endlichen Typs.

  Eine elliptische Kurve über $\K$ ist eine reguläre Kurve $E$ vom
  Genus 1 zusammen mit einem $\K$-Schnitt $O\in E(\K)$.
  Elliptische Kurven sind glatte \cite[Proposition III.3.1]{silverman}
  $\K$"=Gruppenschemata mit Nullelement $O$ \cite{silverman}.
\end{Definition}

%%% Local Variables:
%%% mode: latex
%%% TeX-master: "neron_models"
%%% End:
