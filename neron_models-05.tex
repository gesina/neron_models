\chapter{Arithmetische Flächen}\label{chap:arithmetischeflächen}
Arithmetische Flächen sind besonders gutartige Kurven über
einem Dedekindring $R$, das heißt $R$"~Schemata der Dimension~2.
Besonders wichtig ist, dass ihre Fasern alle projektive Kurven sind,
weswegen sie leicht zu beschreiben sind.

In diesem Kapitel beginnt mit der Definition arithmetischer Flächen,
woraufhin sich zeigt, dass es für elliptische Kurven immer eine
$R$"~Modell dieser Form gibt. Für die Konstruktion wird das Weierstraßmodell
aus \autoref{chap:weierstraßmodelle} nötig sein.
Weiterhin werden einige Eigenschaften arithmetischer Flächen
untersucht. Unter anderem wird gezeigt, dass alle $R$"~Schnitte einer
arithmetischen Fläche bereits $R$"~Schnitte des glatten Teils
sind. Dies ist für die Konstruktion des Néron-Modells später von
Nutzen, welches sich als der glatte Teil einer minimal gewählten
arithmetischen Fläche entpuppen wird.
Diese Form der Minimalität wird in \autoref{chap:minmodelle}
eingeführt, wo auch die Existenz solcher minimalen Modelle für
elliptische Kurven gezeigt wird.

\section{Definition}
\begin{Definition}[Arithmetische Fläche]
  Sei $R$ ein Dedekindring mit Quotientenkörper $\K$.
  % \optcite[IV.4]{silverman2}
  % Eine arithmetische Fläche $X$ über $R$ ist ein
  % integeres, normales, exzellentes Schema, flach und von endlichem
  % Typ über $R$,
  % dessen generische Faser eine reguläre, zusammenhängende,
  % projektive Kurve $X_\K/\K$ ist
  % und dessen spezielle Fasern Vereinigungen von Kurven über den
  % entsprechenden Restklassenkörpern sind.
  % % exzellent-Bediungung:
  % Hat $R$ Quotientenkörper der Charakteristik Null, so ist jedes
  % $R$"~Schema lokal endlichen Typs bereits excellent.
  % \optcite[Theorem 12.51]{wedhorn}
  \optcite[Definition 8.3.1, 8.3.14]{liu}
  Eine arithmetische Fläche ist ein projektives, integres, flaches,
  reguläres $R$"~Schema von Dimension~2, also relativer Dimension~1 über
  $R$, falls $R$ kein Körper ist.
  Die generische Faser $X_\K$ ist eine integre, projektive Kurve über
  $\K$ und jede spezielle Faser $X_s$ über einem abgeschlossenen Punkt
  $s\in\Spec(R)$ ist eine projektive Kurve über $\k(s)$ nach
  \cite[Lemma~8.3.3]{liu}.
\end{Definition}

\begin{Beispiel}\label{bsp:regweierstraßmodell}
  Ein reguläres Weierstraßmodell einer elliptischen Kurve ist eine
  arithmetische Fläche.
  Beachte, dass $W$ regulär sein kann, aber dennoch nicht glatt, also
  singulär in endlich vielen Fasern.
\end{Beispiel}

\begin{Satz}\label{thm:exarithfl}
  Für eine elliptische Kurve $E_\K$ über $\K$ existiert eine
  arithmetische Fläche über $R$, deren generische Faser $E_\K$ ist
  (kurz: eigentliches Modell von $E_\K$ über $R$).
  \begin{proof}
    Aus \autoref{chap:weierstraßmodelle} ist bekannt, dass zu $E_\K$
    ein Weierstraßmodell $W$ über $R$ existiert, das heißt
    ein projektives, integres, flaches $R$"~Schema, dessen generische
    Faser $E_\K$ ist. Zur arithmetischen Fläche fehlt $W$ nur noch die
    Regularität.
    \cite[Corollary~8.3.51]{liu} besagt nun, dass $W$ eine starke,
    projektive Desingularisierung besitzt. Das bedeutet, es gibt
    einen projektives, reguläres $W$"~Schema ${\pi\colon Z\to W}$,
    welches birational zu $W$ und ein Isomorphismus auf den regulären
    Punkten von $W$ ist.
    Birationale Abbildungen erhalten per Definition
    Irreduzibilitätskomponenten, Reduziertheit und Dimension.
    Daher ist $Z$ insbesondere wie $W$ integer
    und von Dimension~2.

    Weiterhin schicken birationale Morphismen generische Punkte auf
    generische Punkte, weshalb nach
    \cite[Proposition~III.9.7]{hartshorne} das reduzierte Schema $Z$
    flach über $R$ ist , weil $W$ flach über $R$ ist.

    Außerdem hat $Z$ dieselbe generische Faser $E_\K$, da
    $W$ in $E_\K$ glatt und insbesondere regulär ist.
    Insgesamt ist $Z$ arithmetischen Fläche über $R$ mit
    generischer Faser $E_\K$.
  \end{proof}
\end{Satz}

\begin{Bemerkung}\label{bem:konstruktioneigmodell}
  Die Desingularisierung von $W$ kann wie folgt induktiv durch
  eine Abfolge von abwechselnden Normalisierungen und Blow-Ups
  konstruiert werden:
  \begin{gather*}
    Z=Z_{n}\to Z_{n-1}'\to Z_{n-1}\to\dotsb
    \to Z_{2}\to Z_{1}'\to Z_{1}\to Z_{0}'=W
  \end{gather*}
  Die $Z_i\to Z_{i-1}'$ sind die Normalisierung der $Z_{i-1}'$,
  das heißt birationale, integre Morphismen mit normalem $Z_{i}$ 
  und wegen Birationalität auch wieder integer. Sie existieren nach
  \cite[Proposition~4.1.22]{liu}.
  
  Die $Z_{i}'\to Z_{i}$ sind Blow-ups entlang des singulären Teils von
  $Z_{i}$, der in diesem Fall jeweils abgeschlossen ist.
  Zu Blow-Ups siehe auch
  \cite[Chapter~8.1]{liu} oder \cite[Chapter~13.19]{wedhorn}.
  Wichtig hier ist, dass Blow-Ups außerhalb des aufgeblähten Bereichs
  Isomorphismen sind. Es bleibt also insbesondere die generische Faser
  und die Eigenschaft integer erhalten.

  \cite[Corollary~8.3.51]{liu} besagt, dass diese Folge in endlich
  vielen Schritten ein reguläres $Z_{n}$ liefert.
\end{Bemerkung}

% -----
\section{Eigenschaften}
Eine recht praktische Eigenschaft von arithmetischen Flächen über
Dedekindringen mit perfekten Restklassenkörpern ist, dass
sie in den Bildern all ihrer Schnitte glatt sind.
Dieser Abschnitt befasst sich mit dieser Aussage, die später für
den Existenzbeweis von Néron-Modellen elliptischer Kurven nötig
ist.

\begin{Lemma}\label{thm:arithflschnittbilderglatt}
  \optcite[Proposition IV.4.3]{silverman2}
  Sei $R$ ein Dedekindring,
  $\pi\colon X\to\Spec(R)$ eine arithmetische Fläche über $R$,
  ${\phi\in X(R)}$ ein $R$"~Schnitt und $\p\in\Spec(R)$.
  Dann ist $X_\p$ regulär in $\phi(\p)$.
  \begin{proof}
    Nachdem die generische Faser einer arithmetischen Fläche bereits
    glatt, also regulär ist, sei $\p\subset\Spec(R)$ ein abgeschlossener
    Punkt.
    Sei $x=\phi(\p)\in X_\p\subset X$ und
    ${\m_{X_\p,x}\subset\O_{X_\p,x}}$ das Maximalideal des Halmes von
    $x$ in $X_\p$, analog ${\m_{X,x}\subset\O_{X,x}}$.
    Der Punkt $x$ ist als Bildpunkt des Schnitts
    $\phi|_{\Spec(\k(\p))}$ abgeschlossen in $X_\p$.
    Daher gilt wegen \cite[Corollary~2.5.24]{liu}
    \begin{gather*}
      \dim\O_{X_\p,x}=\dim X_\p=1\;,
    \end{gather*}
    denn $X_p$ ist equidimensional nach
    \cite[Proposition~4.4.16]{liu} und projektive Kurve.
    Für die Regularität von $X_\p$ in $x$ ist demnach zu zeigen,
    dass $\m_{X_\p,x}/\m_{X_\p,x}^2$ als $(\O_{X_p,x}/\m_{X_p,x})$"~Vektorraum
    von nur einem Element $f_1$ erzeugt wird.
    
    $\O_{X,x}$ ist eine $R$"~Algebra über den induzierten lokalen
    Homomorphismus ${\pi^\#\colon R_\p\to\O_{X,x}}$.
    Sei ${\phi^\#\colon\O_{X,x}\to R_\p}$ der induzierte lokale
    Ringhomomorphismus zu $\phi$ auf den Halmen. Für diesen muss
    ${\phi^\#(\m_{X,x})=\p}$ gelten, denn aus
    \begin{gather*}
      {\p=(\pi\circ\phi)^\#(p)=\phi^\#\circ\pi^\#(\p)}
    \end{gather*}
    folgt ${\p\subset\phi^\#(\m_{X,x})}$ und aus der Lokalität folgt
    ${\p\supset\phi^\#(\m_{X,x})}$.
    
    Es ist zu zeigen, dass es ein Element
    ${t\in\pi^\#(\p)\setminus\m_{X,x}^2}$ gibt, woraus die
    die gewünschte Dimension von ${\m_{X_\p,x}/\m_{X_\p,x}^2}$
    beziehungsweise die Regularität von $X_\p$ in $x$ geschlossen
    werden kann.
    Angenommen, es gelte ${\pi^\#(\p)\subset\m_{X,x}^2}$.
    Mit ${\pi\circ\phi=\Id_{\Spec(R)}}$ folgt die Abschätzung
    \begin{gather*}
      \p = (\pi\circ\phi)^\#(\p)
      = \phi^\#\circ\pi^\#(\p)
      \overset{\text{Ann.}}{\subset} \phi^\#(\m_{X,x}^2)
      = \phi^\#(\m_{X,x})^2
      = \p^2
    \end{gather*}
    Dies ist ein Widerspruch, da ${\p\subset R_\p}$ das Maximalideal
    eines diskreten Bewertungsringes ist und deshalb wegen Regularität
    von $R_\p$ gelten muss, dass ${\p/\p^2\neq 0}$.
    Also ist ${\pi^\#(\p)\not\subset\m_{X,x}^2}$ gezeigt.

    Nach \autoref{eq:stalkdimension} gilt
    ${\dim\O_{X,x}=\dim R_\p+\dim\O_{X_\p,x}=2}$ und $X$ ist nach
    Voraussetzung regulär, weshalb $\O_{X,x}$ regulär von Dimension~2
    ist. Das bedeutet, dass es zwei Elemente $f_1, {f_2\in\m_{X,x}}$
    gibt, so dass
    \begin{gather}\label{eq:maximalideal}
      \m_{X,x} = f_1\O_{X,x} + f_2\O_{X,x} + \m_{X,x}^2
    \end{gather}
    Wegen ${\pi^\#(\p)\not\subset\m_{X,x}^2}$ gibt es ein
    ${t\in\pi^\#(\p)\setminus\m_{X,x}^2\subset\m_{X,x}}$.
    \autoref{eq:maximalideal} bedeutet für $t$, dass es $\alpha_1,
    {\alpha_2\in\O_{X,x}}$ gibt, so dass
    \begin{gather*}
      0 \neq t = \alpha_1 f_1 + \alpha_2 f_2 \mod \m_{X,x}^2
    \end{gather*}
    Da $t\not\in\m_{X,x}^2$, muss mindestens einer der Koeffizienten
    $\alpha_1,\alpha_2$ in $\O_{X,x}\setminus\m_{X,x}=\O_{X,x}^\times$
    liegen, also eine Einheit sein. Ohne Einschränkung sei
    $\alpha_2\in\O_{X,x}^\times$. Dann nach $f_2$ wie folgt
    aufgelöst werden
    \begin{gather*}
      f_2 = {\alpha_2}^{-1}t + {\alpha_1}{\alpha_2}^{-1}f_1 
    \end{gather*}
    und aus \eqref{eq:maximalideal} wird die neue Gleichung
    \begin{gather}\label{eq:maximalidealneu}
      \m_{X,x} = t\O_{X,x} + f_1\O_{X,x} + \m_{X,x}^2\;.
    \end{gather}
    Für den lokalen Ring $\O_{X_\p,x}$ von $x$ in der generischen
    Faser ${X_\p = X\times_R (R/\p)}$ gilt
    \begin{align*}
      \O_{X_\p,x} &= \O_{X,x}/\pi^\#(\p)\;,\\
      \m_{X_\p,x} &= \m_{X,x}/\pi^\#(\p)
                    \overset{\eqref{eq:maximalidealneu}}{=}
                    \bar t\O_{X_\p,x} + \bar f_1\O_{X_\p,x} +
                    \m_{X_\p,x}^2
                    \overset{t\in\pi^\#(\p)}{=}
                    \bar f_1\O_{X_\p,x} + \m_{X_\p,x}^2 \;.
    \end{align*}
    Es ist zu sehen, dass $\m_{X_\p,x}$ von $\m_{X_\p,x}^2$ und einem Element
    erzeugt ist, was $\O_{X_\p,x}$ zu einem regulären Ring der Dimension~1
    macht.
  \end{proof}
\end{Lemma}

\begin{Satz}\label{thm:ratpkteregulaeremodelle}
  \optcite[Corollary IV.4.4]{silverman2}
  Sei $R$ ein Dedekindring
  mit perfekten Restklassenkörpern,
  $\K=\Quot(R)$,
  sei $X$ eine arithmetische Fläche,
  $X^0$ das offenes Unterschema der glatten Punkte und
  $X_\K$ die generische Faser von $X$.
  Dann gilt
  \begin{gather*}
    X_\K(\K)=X(R)=X^0(R)\;.
  \end{gather*}
  \begin{proof}
    Die erste Gleichheit ist allein eine Anwendung des
    Bewertungskriteriums für Eigentlichkeit auf das projektive, also
    eigentliche Schema $X$.
    Wegen $X^0\subset X$ gibt es eine natürliche Inklusion
    $\circ \text{incl}\colon X^0(R)\hookrightarrow X(R)$.
    Jetzt gilt es zu zeigen, dass im regulären Fall das Bild jedes
    Schnitts $\phi\in X(R)$ in $X^0$ liegt, also nur glatte Punkte
    enthält. Dann ist die Inklusion ein Isomorphismus.
    
    Nach \ref{thm:arithflschnittbilderglatt} enthält das Bild
    $\phi(\Spec(R))$ eines Schnitts $\phi\in X(R)$ nur reguläre
    Punkte der Fasern. Diese sind glatte Punkte von $X$. Denn
    $X$ ist flach und lokal endlich präsentiert, daher genau dann
    glatt, wenn die Fasern glatt sind nach
    \cite[Proposition~8.5/17]{bosch}.
    Die generische Faser ist aber bereits glatt und die speziellen
    Fasern sind als Schemata endlichen Typs über einem perfektem
    Körper genau dann glatt, wenn sie regulär sind.
    Damit ist die Behauptung gezeigt.
  \end{proof}
\end{Satz}

% -----

\section{Minimale Modelle}\label{chap:minmodelle}
In diesem Abschnitt sei $R$ immer ein Dedekindring,
% mit perfekten Restklassenkörpern,
$\K$ sein Quotientenkörper und $C_\K$ eine
reguläre, projektive Kurve über $\K$.

Eigentliche Modelle können für elliptische Kurven minimal
bezüglich birationaler Morphismen gewählt werden, wie in diesem
Abschnitt gezeigen wird. Aus der Konstruktion ist auch gleich ersichtlich,
dass der Algorithmus aus \ref{bem:konstruktioneigmodell} bereits ein
minimales Modell liefert.
Ein solches minimales Modell ist die Grundlage für den späteren
Existenzbeweis des Néron-Modells, welches genau dessen glatter Teil
ist.

\begin{Definition}[Minimales Modell]
  Ein minimales eigentliches Modell $X_{\text{min}}$ für $C_\K$ über
  $R$ (kurz: minimales Modell von $C_\K$ über $R$) ist ein
  eigentliches $R$"~Modell von $C_\K$, das folgende universelle
  Minimalitätseigenschaft erfüllt
  \begin{quote}
    Für jedes weitere eigentliche Modell $X$ von
    $C_\K$ über $R$ und für jeden $\K$"~Isomorphismus $\phi_\K$ der
    generischen Fasern ist die nach \ref{thm:erwrrational}
    induzierte $R$"~birationale Abbildung $\phi\colon X\dra
    X_{\text{min}}$ ein $R$"~Morphismus.
  \end{quote}
\end{Definition}
% % exzellent oder glatt nötig!!! evtl. nur Konstruktion?
% \begin{Definition}
%   Sei $R$ Dedekindring mit perfekten Restklassenkörpern,
%   $\K=\Quot(R)$,
%   $C/\K$ reguläre, projektive Kurve vom Geschlecht $g$.
%   \begin{enumerate}[label=(\roman*)]
%   \item\cite[Proposition IV.4.5(a)]{silverman2}
%     Dann existiert ein eigentliches $R$"~Modell $X$ von $C$,
%     das eine arithmetische Fläche darstellt.
%   \item\cite[Proposition IV.4.5(b)]{silverman2} Ist $g\geq1$ kann
%     ein minimales eigentliches Modell $X_{\text{min}}$ für $C$ über
%     $R$ gewählt werden (kurz: minimales Modell von $C$ über $R$).
%     D.\,h. für jedes andere eigentliche Modell $X$ von
%     $C/\K$ über $R$ und für jeden $\K$"~Isomorphismus $\phi_\K$ der
%     generischen Fasern ist die nach \ref{thm:erwrrational}
%     induzierte $R$"~birationale Abbildung $\phi\colon X\dra
%     X_{\text{min}}$ ein $R$"~Morphismus.
%   \end{enumerate}
% \end{Definition}

\begin{Satz}\label{thm:eigminmodell}
  Ein minimales Modell $X_{\text{min}}$ von $C_\K$ über $R$ hat
  folgende Eigenschaften
  \begin{enumerate}[label=(\alph*)]
  \item $X_{\text{min}}$ ist eindeutig bis auf eindeutige Isomorphie.
  \item\optcite[Proposition IV.4.6]{silverman2}
    Jeder $\K$"~Automorphismus von $C_\K$ erweitert sich eindeutig zu
    einem $R$"~Automorphismus von $X_{\text{min}}$, der glatte Punkte
    auf glatte Punkte abbildet.
  \item\optcite[Beweis von Proposition IV.6.10]{silverman2}
    Minimale Modelle sind wie folgt stabil unter Basiswechsel:
    Sei $Y$ ein glattes $R$"~Schema, ${x\in Y}$ und der Halm
    ${R'=\O_{Y,x}}$ in $x$  ein diskreter Bewertungsring mit
    ${\K'\coloneqq\Quot(R')}$.
    Dann ist das minimale Modell von $C_\K\times_\K\K'$ genau
    $X_{\text{min}}\times_R R'$.
    Ist $E\subset X_{\text{min}}$ das Unterschema der glatten Punkte
    von $X_{\text{min}}$, ist
    $E\times_R R'\subset X_{\text{min}}\times_R R'$ das Unterschema
    der glatten Punkte von $X_{\text{min}}\times_R R'$.
  \end{enumerate}
  \begin{proof}
    Es werden wie in \ref{thm:eigneronmodell} die universelle
    Eigenschaft sowie Stabilität unter Basiswechsel für mehrere
    Eigenschaften benötigt.
    \begin{enumerate}[resume*,start=1]
    \item Für zwei minimale Modelle $X_1$, $X_2$ von $C_\K$ über $R$
      liefert die Minimalitätseigenschaft zwei $R$"~Morphismen
      $X_1\to X_2$, $X_2\to X_1$, deren Komposition jeweils die
      Identität auf $C_\K$ erweitert. Wegen der Eindeutigkeit aus
      \ref{thm:erweindeutig} müssen die Kompositionen jeweils die
      Identität sein und die Morphismen sind zueinander inverse
      $R$"~Isomorphismen.
    \item Die Minimalitätseigenschaft liefert zu einem
      $\K$"~Automorphismus $\tau\colon C_\K\to C_\K$ die Erweiterung auf
      einen $R$"~Automorphismus
      $\tau\colon X_{\text{min}}\to X_{\text{min}}$, der wegen
      Eigentlichkeit eindeutig ist.
      Als bijektive, also offene Abbildungen, schicken $\tau$ und
      sein Inverses offene glatte Umgebungen auf offene glatte
      Umgebungen, und somit glatte Punkte auf glatte
      Punkte. Einschränkung liefert also einen $R$"~Automorphismus auf
      dem Unterschema der glatten Punkte.
    \item Die Eigenschaften flach, projektiv und integer sind stabil
      unter Basiswechsel. Regulär ist stabil unter glattem
      Basiswechsel \cite[Chapter~2.3, S.\,49]{neron}, weshalb
      $C_\K\times_R Y$ regulär ist.
      Nachdem $C_\K\times_R R'$ dieselben Halme hat, ist es ebenfalls
      regulär.
      Die Minimalitätseigenschaft von ${X_{\text{min}}\times_R R'}$
      folgt aus der von $X_{\text{min}}$ und der universellen
      Eigenschaft des Faserprodukts.
      
      $E\times_R R'$ ist glattes Unterschema von
      ${X_{\text{min}}\times_R R'}$, da glatt stabil unter Basiswechsel
      ist. Nachdem $Y$ glatt ist, ist ein Punkt 
      ${X_{\text{min}}\times_R Y}$ und insbesondere im Unterschema
      ${X_{\text{min}}\times_R R'\subset X_{\text{min}}\times_R Y}$
      genau dann glatt, wenn seine Projektion auf $X_{\text{min}}$
      glatt ist, also wenn sie in $E$ liegt. Damit ist ${E\times_R R'}$
      das Unterschema aller glatten Punkte in
      ${X_{\text{min}}\times_R R'}$.
      \qedhere
    \end{enumerate}
  \end{proof}
\end{Satz}

\begin{Satz}\label{thm:exminmodell}
  Minimale $R$"~Modelle elliptischer Kurven existieren und sind nach
  \ref{thm:eigminmodell} eindeutig bis auf eindeutige Isomorphie.
  \begin{proof}[Beweisskizze]
    Für den vollständigen Beweis siehe \cite[Theorem~9.3.21]{liu}.
    Aus \ref{thm:exarithfl} ist bekannt, dass zu einer elliptischen
    Kurve $E_\K$ über $\K$ zumindest eine arithmetische Fläche
    gefunden werden kann.
    \cite[Theorem~9.3.21]{liu} besagt, dass allgemein für
    arithmetische Flächen mit generischer Faser vom Geschlecht
    größer~0 eine minimale existiert.
    Genutzt wird hierbei, dass die Existenz eines minimalen Modells
    äquivalent dazu ist, dass folgende Bedingungen erfüllt sind:
    \begin{enumerate}[label=(\alph*)]
    \item Es existiert ein \emph{relativ minimales Modell}, das heißt
      ein $R$"~Modell mit der abgeschwächten Minimalitätseigenschaft:
      Jeder birationale Morphismus in ein anderes $R$"~Modell ist
      Isomorphismus.
    \item Alle relativ minimalen Modelle sind isomorph, es gibt also
      ein \emph{eindeutiges} relativ minimales Modell.
    \end{enumerate}
    Relative Minimalität ist nach \cite[Theorem~9.2.2]{liu} äquivalent
    dazu, dass das Modell keine außerordentlichen Divisoren besitzt,
    also keine integren, zusammenziehbaren Kurven als Teilmengen
    hat.
    Außerordentliche Divisoren können unter anderem mit Castelnuovos
    Kriterium \cite[Theorem~9.3.8]{liu} identifiziert werden und es
    gibt nur eine endliche Anzahl an Fasern, die solche enthalten nach
    \cite[Lemma~9.3.17]{liu}.

    Wie in \cite[Proposition~9.3.19]{liu} gezeigt wird, ist es in
    endlich vielen Schritten möglich, die zusammenziehbaren Divisoren
    nacheinander zu einem relativ minimalen Modell zusammenzuziehen,
    womit Bedingung (a) erfüllt ist.

    Bedingung (b) mit der Eindeutigkeit dieses Modells kann wie in
    \cite[Theorem~9.3.21]{liu} beschrieben durch Widerspruchsbeweis
    mithilfe der Bedingung an das Geschlecht der generischen Faser
    gezeigt werden.
    Das Zusammenziehen hat also bereits ein minimales Modell geliefert.
  \end{proof}
\end{Satz}

\begin{Bemerkung}\label{bem:konstruktionminmodell}
  Ist in \ref{thm:exarithfl} das $n$ minimal gewählt, ist das
  resultierende eigentliche Modell bereits ein relativ minimales, also
  minimales. Denn bei jedem Zusammenziehen eines Divisors würde per
  Konstruktion die Regularität des Modells verloren gehen.
\end{Bemerkung}

%%% Local Variables:
%%% mode: latex
%%% TeX-master: "neron_models"
%%% End:
