\chapter{Gruppenstrukturen auf Schemata}
\label{chap:gruppenstrukturenaufschemata}
\section{Gruppenschemata}
Im Folgenden sei $S$ ein Schema.
Zuerst werden wir in diesem Kapitel Gruppenschemata einführen, also
$S$-Schemata, deren Schnittfunktor Gruppen liefert. Als Spezialfälle
werden abelsche Schemata und abelsche Varietäten aufgeführt. Ein
wichtiges Beispiel für letzteres sind gerade elliptische Kurven.

Danach werden wir sehen, dass sich eigentliche Gruppenschemata
gutartig bzgl. rationalen Abbildungen verhalten,
das heißt bezüglich Morphismen, die nur auf offenen dichten Teilmengen
definiert sind.

\begin{Definition}[$S$"=Gruppenschema]\label{def:gruppenschema}
  Ein \emph{Gruppenschema über $S$} oder \emph{$S$-Gruppenschema} ist ein
  $S$-Schema ${\pi\colon G\to S}$ zusammen mit $S$-Morphismen
  \begin{align*}
    \mu\colon G\times_S G &\longto G
    &&\text{(Gruppenverknüpfung)}\\
    \epsilon\colon S&\longto G 
    &&\text{(Neutrales Element)}\\
    i\colon G &\longto G    
    &&\text{(Inverses)}
  \end{align*}
  für die folgende Diagramme kommutieren

  \begin{tabular}{>{\itshape(}l<{)}l}
    Assoziativität
    &\begin{tikzcd}
      G\times_S G\times_S G 
      \arrow[r, "{(\mu,\Id)}"] \arrow[d, "{(\Id,\mu)}"]
      & G\times_S G \arrow[d, "{\mu}"] \\
      G\times_S G \arrow[r, "\mu"]
      & G
    \end{tikzcd}\\
    Neutrales Element
    &\begin{tikzcd}
      G \arrow[r, "{(\pi,\Id)}"] \arrow[d, "{\Id}"]
      & S\times_S G \arrow[d, "{\epsilon\times\Id}"] \\
      G \arrow[from=r, "\mu"]
      & G\times_S G
    \end{tikzcd}
    und
    \begin{tikzcd}
      G \arrow[r, "{(\Id,\pi)}"] \arrow[d, "\Id"]
      & G\times_S S \arrow[d, "{\epsilon\times\Id}"] \\
      G \arrow[from=r, "\mu"]
      & G\times_S G
    \end{tikzcd}\\
    Inverses
    &\begin{tikzcd}
      G \arrow[r, "{(i,\Id)}"] \arrow[d, "{\pi}"]
      & G\times_S G \arrow[d, "{\mu}"] \\
      S \arrow[r, "\epsilon"]
      & G
    \end{tikzcd}
    und
    \begin{tikzcd}
      G \arrow[r, "{(\Id,i)}"] \arrow[d, "{\pi}"]
      & G\times_S G \arrow[d, "{\mu}"] \\
      S \arrow[r, "\epsilon"]
      & G
    \end{tikzcd}
  \end{tabular}
\end{Definition}
Die Folgende Äquivalenz zeigt, wo der Gruppenbegriff eines
Gruppenschemas zum tragen kommt:
\begin{Bemerkung}
  Die obige Definition ist äquivalent dazu, dass die $T$-wertigen Punkte
  $G(T)=\Hom_S(T,G)$ eine Gruppenstruktur tragen, die funktoriell in
  $T$ ist für $S$-Schemata $T$.\optcite[Chapter (4.15)]{wedhorn}
  In anderen Worten der Schnittfunktor faktorisiert über den
  Vergissfunktor
  \begin{align*}
    \Sch &\longto \Groups \longto \text{(Sets)}\\
    T &\longmapsto G(T)
  \end{align*}
  Die Gruppenstruktur auf $G(T)$ ist durch $\mu$ gegeben als
  \begin{align*}
    G(T)\times G(T) &\longto G(T)\\
    (\phi, \psi) &\longmapsto 
                   \left(
                   T\xrightarrow{(\phi,\psi)} G\times_S G
                   \overset{\mu}\longto G
                   \right)
  \end{align*}
  wobei der Schnitt $T\to S\xrightarrow{\epsilon} G$ die
  Identität ist und man Inverse durch Verknüpfung mit $i$ erhält.
  \optcite[Proposition IV.3.2]{silverman2}
  Die Gruppenstruktur liefert außerdem, dass für einen Schnitt
  $\sigma\in G(S)$ der $S$-Morphismus
  \begin{gather*}
    \tau_\sigma\colon G\cong G\times_S S
    \xrightarrow{\Id\times_S\sigma}
    G\times_S G
    \overset{\mu}\longto G
  \end{gather*}
  ein $S$-Automorphismus von $G$ ist mit Inversem
  $\tau_{(\sigma^{-1})}$, genannt Translation um $\sigma$.
\end{Bemerkung}

Anhand der Gruppenstruktur auf den Schnitten lassen sich
Morphismen zwischen Gruppenschemata und Untergruppenschemata
definieren.
\begin{Definition}[Morphismen von Gruppenschemata]
  \optcite[Definition 4.42]{wedhorn}
  Seien $G$ und $H$ $S$"=Gruppenschemata.
  Ein Morphismus von $S$"=Gruppenschemata ist ein Morphismus $f\colon
  G\to H$ von $S$-Schemata, für den die induzierten Abbildungen
  \begin{gather*}
    (\bullet\circ f)\colon G(T)\to H(T)
  \end{gather*}auf den $T$-wertigen Punkten
  Gruppenhomomorphismen sind für alle $S$-Schemata $T$.
  In anderen Worten
  \begin{gather*}
    f\circ\mu_G = \mu_H\circ(f\times_S f)
  \end{gather*}
\end{Definition}

% IDENTITÄTSKOMPONENTE
\begin{Definition}[Untergruppenschema]
  \optcite[Definition 4.45]{wedhorn}
  Sei $G$ ein $S$"=Gruppenschema.
  Ein $S$"=Untergruppenschema von $G$ ist ein geschlossenes
  $S$-Unterschema von $G$, dessen geschlossene Immersion ein
  Homomorphismus von Gruppenschemata ist.
\end{Definition}

Hier gleich ein wichtiges Beispiel eines Untergruppenschemas, das
durch die Topologie gegeben ist.
\begin{Definition}[Identitätskomponente]
  \optcite[Remark IV.6.1.2]{silverman2}
  Sei $G$ ein $S$"=Gruppenschema.
  Für $s\in S$ ist die Identitätskomponente einer Faser $G_s$ von $G$
  die Zusammenhangskomponente von $G_s$, die $\epsilon(s)$ enthält.
  Die Identitäts-~oder Zusammenhangskomponente von $G$ ist
  die Vereinigung aller Identitätskomponenten der Fasern von $G$.
  Die Identitätskomponente ist ein Untergruppenschema.
\end{Definition}

% ABELSCHE SCHEMATA
Wir werden nun einige wichtige Spezialfälle von Gruppenschemata
betrachten.
\begin{Bemerkung}\label{gruppenschemaaequivalenzen}
  Für ein $\K$"=Gruppenschema endlichen Typs über einem Körper $\K$
  sind glatt und geometrisch reduziert äquivalent.
  \optcite[8.5, Excercise 11]{bosch}
  Im glatten Fall sind auch irreduzibel und zusammenhängend äquivalent
  nach \cite[Exercise 3.16]{wedhorn}, \cite[Corollary 16.51]{wedhorn}.
  % , da
  % aufgrund der Glattheit alle Halme lokal und regulär, also
  % insb. integer, sind und $G$ lokal noethersch ist.
\end{Bemerkung}
\begin{Definition}[Abelsche Varietät]\label{def:abelschevarietaet}
  Eine abelsche Varietät $X/\K$ ist ein eigentliches, glattes,
  irreduzibles $\K$"=Gruppenschema.
  \optcite[9.6, Definition 1]{bosch}\optcite[Definition 16.52]{wedhorn}
\end{Definition}
\begin{Bemerkung}\label{thm:abvarietaetprojektiv}
  Abelsche Varietäten sind projektiv nach
  \cite[9.6, Proposition 4]{bosch}.
\end{Bemerkung}

\begin{Definition}[Abelsches Schema]
  Ein abelsches $S$-Schema ist ein eigentliches, glattes $S$"=Gruppenschema,
  dessen Fasern abelsche Varietäten sind 
  (also zusätzlich irreduzibel bzw. zusammenhängend,
  s. \ref{def:abelschevarietaet} und \ref{gruppenschemaaequivalenzen},
  nachdem eigentlich und glatt stabil unter Basiswechsel sind).
\end{Definition}

% S-RATIONALE ABBILDUNG
In manchen Fällen ist für ($S$-)Schemata $X$, $Y$ kein
($S$-)Morphismus $f\colon X\to Y$ gegeben, sondern nur ein Morphismus
auf einer offenen dichten Teilmenge von $X$ (z.\,B. der generischen
Faser). Solche Fälle sind besonders interessant, wenn man Aussagen
darüber treffen möchte, ob und wie sich dieser Morphismus auf ganz $X$
erweitert. Ein Beispiel sind rationale Abbildungen zwischen
projektiven Varietäten über algebraisch abgeschlossenen Körpern.
Eine Verallgemeinerung davon sind die ($S$-)rationalen Abbildungen
zwischen ($S$-)Schemata.
\begin{Definition}[rationale Abbildung]\label{def:ratabb}
  \optcite[Chapter 2.5]{neron}
  Sei $S$ Schema, und seien $X$ und $Y$ glatte $S$-Schemata.
  Eine rationale Abbildung $\phi\colon X\dra Y$ von
  $S$-Schemata ist eine Äquivalenzklasse der Menge
  \begin{gather*}
    \left\{
      \psi\colon U\to Y \text{ $S$-Morphismus}
      \;\middle|\;
      U\subset X \text{ dicht, offen in $X$}
    \right\}
  \end{gather*}
  wobei $\psi\colon U\to Y$ äquivalent zu $\psi'\colon U'\to Y$ ist,
  falls sie auf einer in $X$ dichten, offenen Teilmenge von $U\cap U'$
  übereinstimmen.
  
  Ein offenes Unterschema $U\subset X$ heißt $S$-dicht in $X$,
  falls für alle $s\in S$ die Faser $U_s$ %=U\times_S\k(s)$
  topologisch dicht in der Faser $X_s$ %=X\times_S\k(s)$
  liegt.
  Eine $S$-rationale Abbildung $\phi\colon X\dra Y$ ist eine
  rationale Abbildung, wobei alle dichten offenen Mengen aus obiger
  Definition $S$-dicht sind.
  % Eine \emph{$S$-rationale Abbildung} $\phi\colon X\dra Y$ ist eine
  % Äquivalenzklasse der Menge
  % \begin{gather*}
  %   \left\{
  %     \psi\colon U\to Y \text{ $S$-Morphismus}
  %     \;\middle|\;
  %     U\subset X \text{ $S$-dicht, offen in $X$}
  %   \right\}
  % \end{gather*}
  % wobei $\psi\colon U\to Y$ äquivalent zu $\psi'\colon U'\to Y$ ist,
  % falls sie auf einer $S$-dichten, offenen Teilmenge von $U\cap U'$
  % übereinstimmen.

  Eine ($S$-)rationale Abbildung $\phi\colon X\dra Y$ heißt definiert
  in $x\in X$, falls $\phi$ von einem Morphismus $\psi\colon U\to Y$
  präsentiert werden kann mit $x\in U$.
  Der Definitionsbereich $\Dom(\phi)$ von $\phi$ sind die
  Punkte, in denen $\phi$ definiert ist. Er ist offene, ($S$-)dichte
  Teilmenge von $X$. Ist $X/R$ separiert gibt es einen Morphismus
  $\Dom(\phi)\to Y$, der $\phi$ präsentiert.
  
  $\phi\colon X\dra Y$ heißt ($S$-)birational, falls es von einem
  $S$-Morphismus $\psi\colon U\to Y$ präsentiert werden kann, der
  die offene Immersion einer ($S$-)dichten, offenen Teilmenge von $Y$
  ist.
\end{Definition}

Nachdem unser Hauptaugenmerk auf Gruppenschemata liegt, ist für uns
besonders interessant, wie sich $S$-Gruppenschemata
bzgl. ($S$-)rationalen Abbildungen verhalten.
Tatsächlich verhalten sich eigentliche Gruppenschemata in folgender
Hinsicht gutartig:
\begin{Lemma}[{\cite[Proposition IV.6.2]{silverman2}}]
  \label{thm:rationalzumorphismus}
  Sei $G$ ein $R$"=Gruppenschema, $Y$ ein glattes $R$-Schema und
  $\phi\colon Y\dra G$ eine $R$-rationale Abbildung.
  Dann gilt
  \begin{enumerate}[label=(\roman*)]
  \item $Y\setminus \Dom(\phi)$ hat (reine) Kodimension 1.
  \item Ist $G$ eigentlich, so ist $\phi$ Morphismus.
  \end{enumerate}
\end{Lemma}

% -----

\section{Normale Komposition auf Schemata}
Sei $R$ im folgenden ein Dedekindring.
In manchen Fällen kann der Morphismus einer Gruppenstruktur nicht auf
dem gesamten Schema sondern nur als $R$-rationale Abbildung angegeben
werden. So kommt man zur Definition einer normalen Komposition, welche
eine $R$-rationale Gruppenstruktur liefert.
\begin{Definition}[normale Komposition]\label{def:normalekomposition}
  Sei $V$ glattes $R$-Schema mit nichtleeren Fasern.
  Eine normale Komposition auf $V$ ist eine $R$-rationale
  Abbildung
  \begin{gather*}
    \mu\colon V\times_R V\dra V
  \end{gather*}
  die folgende Bedingungen erfüllt
  \begin{description}[font=\normalfont\itshape]
  \item[(Assoziativität)] Es gilt, wo es definiert ist,
    \begin{gather*}
      \mu(\mu(x,y),z)=\mu(x,\mu(y,z))
    \end{gather*}
  \item[(Inverses)]
    Die $R$-rationalen Abbildungen
    \begin{align*}
      \phi\colon V\times_R V
      &\dra V\times_R V
      &\psi\colon V\times_R V
      & \dra V\times_R V\\
      (x,y)
      &\longmapsto (x,\mu(x,y))
      &(x,y)
      &\longmapsto (y,\mu(x,y))       
    \end{align*}
    haben dichtes Bild in jeder Faser und die Einschränkung auf jede
    Faser ist birational.
  \end{description}
\end{Definition}

Ein großer Vorteil einer normalen Komposition ist, dass zumindest
ein großer Teil, nämlich ein offenes dichtes Unterschema, offen und
dicht in einem richtigen Gruppenschema mit derselben Gruppenstruktur
liegt.
\begin{Satz}[{Weil,
  \cite[Theorem VIII.1.12]{artin},
  \cite[Theorem IV.6.9]{silverman2}}]\label{thm:weil}
  Sei $V$ glattes $R$-Schema mit nichtleeren Fasern
  und normaler Komposition $\mu\colon V\times_R V\dra V$.
  $V$ sei zusätzlich endlichen Typs und $V(R)$ sei dicht in jeder
  Faser. 
  % TODO!! Was heißt V(R) dicht in jeder Faser?
  Dann gibt es ein $R$-Gruppenschema $(G,\mu_G)$ endlichen Typs und
  ein gemeinsames, offenes, $R$-dichtes Unterschemata $V\hookleftarrow
  U\hookrightarrow G$ von $V$ und $G$, so dass $\mu|_U=\mu_G|_U$.
  % \begin{proof}
  %   \begin{enumerate}
  %   \item Fall zusammenhängende Fasern:
  %     \begin{enumerate}
  %     \item $X=X_\phi\cap X_\psi\cap Y_\phi\cap Y_\psi$:
  %       $V$ hat normale Komposition $\mu$ mit zugehörigen rationalen
  %       Abbildungen $\phi$ und $\psi$.
  %       Wähle $R$-dichte, offene Teilmengen
  %       $X_\phi,X_\psi,Y_\phi,Y_\psi\subset V\times_R V$,
  %       so dass $\phi\colon X_\phi\overset\sim\to Y_\phi$,
  %       $\psi\colon X_\psi\overset\sim\to Y_\psi$ Isomorphismen sind.
  %       Definiere $X=X_\phi\cap X_\psi\cap Y_\phi\cap Y_\psi\subset
  %       V^2$.
  %       $X$ ist wieder $R$-dicht und $\phi|_X,\psi|_X\colon X\cong X$
  %       sind per Definition Isomorphismen auf $X$.
  %     \item Es gibt eine offene, $R$-dichte Teilmenge $U\subset V$, so
  %       dass die Einschränkung der Projektionen
  %       $\proj_i\colon V\times_R V\to V$ auf $X$ surjektiv sind.
  %     \item Für $V=U$ gilt:
  %       Sei $a\in V$, $x\in V$ in dichter, offener Teilmenge der
  %       Faser, die $a$ enthält.
  %       Dann sind $\phi^{\pm1}, \psi^{\pm1}$ in $(x,a)\in V\times_R V$
  %       definiert.
  %     \item Sei $\Gamma_\mu\subset V^3$ der Graf von $\mu$
  %       ($R$-Schema), $\Gamma\coloneqq \overline{\Gamma_\mu}\subset
  %       V^3$.
  %       Dann sind die Einschränkungen der Projektionen
  %       $\proj_{1,2},\proj_{2,3},\proj_{1,3}\colon \Gamma\to V^2$
  %       offene Immersionen und dicht in jeder Faser von $V^2$.
  %     \item Verkleben entlang von $W_s=V\times_R s\times_R V$ für
  %       $s\in V$:
  %       …
  %     \item …
  %     \end{enumerate}
  %   \item Stein-Faktorisierung + „dense set of local sections“?:
  %     Erhalte offenes Unterschema 
  %   \end{enumerate}
  % \end{proof}
\end{Satz}

%%% Local Variables:
%%% mode: latex
%%% TeX-master: "neron_models"
%%% End:
