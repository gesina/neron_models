\chapter{Existenz von Néron-Modellen auf elliptischen Kurven}
\label{chap:exneronmodelle}
% EXISTENZSATZ ELL. KURVEN
\begin{Satz}\label{thm:exneronmodelle}
  % Der diskrete Bewertungsring $R$ sei strikt henselsch mit algebraisch
  % abgeschlossenem Quotientenkörper $\K$ und
  % $W$ ein minimales reguläres, eigentliches, flaches $R$-Modell der
  % elliptischen Kurve $E_\K$.
  % Dann ist das $R$-Unterschema der glatten Punkte von $W$ ein
  % Néron-Modell von $E_\K$.
  % \cite[1.5, Proposition 1, S.\,21]{neron}
  % \begin{proof}
  %   …
  % \end{proof}

  % Fehlt: \cite[Theorem IV.6.1]{silverman2}
  % (ähnlich, ohne Bed. an den Ring, dafür mehr an das Modell)
  Sei $R$ Dedekindring mit perfekten Quotientenkörpern,
  $\K=\Quot(R)$, $E_\K$ elliptische Kurve über $\K$ und $C$ minimales
  $R$-Modell von $E_\K$,
  das heißt reguläres, integeres, projektives,
  flaches $R$-Schema mit Minimalitätseigenschaft.
  Sei $E$ das offene Unterschema der glatten Punkte von $C$.
  Dann ist $E$ Néron-Modell von $E_\K$.

  Nachdem minimale Modelle elliptischer
  Kurven nach \ref{thm:exminmodell} existieren, hat jede
  elliptische Kurve über einem Dedekindring mit perfekten
  Restklassenkörpern ein Néron-Modell.
\end{Satz}
Dieses Kapitel ist dem Beweis des vorangegangenen Satzes gewidmet.

Von der Idee her werden wir uns vorerst in
\autoref{chap:gruppenstrukturhinreichend} klar machen, dass $E$ genau 
dann ein Néron-Modell von $E_\K$ ist, wenn sich die Gruppenverknüpfung
von $E_\K$ auf ganz $E$ erweitert (das \enquote{genau dann} ist
bereits aus \ref{thm:gruppenschemaerweiterung} bekannt).

Nun bleibt es zu zeigen, dass sich die Gruppenverknüpfung tatsächlich
auf $E$ erweitert.
Wir betrachten dies in \autoref{chap:gruppestrikthenselsch} vorerst für
den Spezialfall eines strikt henselschen diskreten Bewertungsrings
$R$. Dabei sehen wir, dass man eine normale Komposition auf $E$
erhält. Nach \ref{thm:weil} wissen wir nun, dass $E$ birational zu
einem $R$-Gruppenschema mit der gewünschten Struktur ist. Für den
Abschluss des Beweises stellen wir fest, dass
diese $R$-birationale Abbildung bereits ein $R$-Isomorphismus und $E$
daher ein $R$-Gruppenschema ist.

In \autoref{chap:exneronmodellebeweis} zeigen wir zuletzt, dass wir
uns ohne Einschränkung auf strikt henselsche Bewertungsringe
beschränken können, was dann mit den vorangegangenen Ergebnissen den
Beweis vollendet.


% \begin{proof}
  %   \begin{enumerate}
  %   \item \ref{thm:rationalzumorphismus}:
  %     Eine rationale Abbildung $X\dra G$ von einem glatten
  %     $R$-Schema in ein eigentliches $R$-Gruppenschema ist Morphismus.
  %   \item \ref{def:henselscheringe}: (strikt) Henselsche Ringe
  %   \item Eigenschaften henselscher Ringe auf Schnitteinschränkung:
  %     $(R,\m)$ diskreter Bewertungsring mit Restklassenkörper $k$, $X$
  %     glattes $R$-Schema, $X_k$ spezielle Faser.
  %     Dann
  %     \begin{enumerate}[label=(\roman*)]
  %     \item Falls $R$ henselsch, ist $X(R)\to X_k(k)$ surjektiv.
  %     \item Falls $R$ strikt henselsch, ist $X(R)\to X_k(k)$ surjektiv
  %       und hat dichtes Bild in $X_k$.
  %       (Beachte, dass $X_k$ glatt, also von endlichem Typ ist und $k$
  %       nach Voraussetzung algebraisch abgeschlossen ist. Daher
  %       gilt $X_k(k)\cong\{x\in X_k\text{ geschl.}\}\subset X_k$
  %       \cite[Example 4.1]{wedhorn}.)
  %     \end{enumerate}
  %     \cite[Proposition IV.6.4]{silverman2},
  %     \cite[2.3, Proposition 5]{neron}
  %   \item Henselisierung: \ref{thm:exhenselisierung}
  %   \item Fall $R$ strikt henselscher, diskreter BWR:
  %     \begin{enumerate}
  %     \item Ist $E$ Gruppenschema, so ist es Néron-Modell
  %       \ref{thm:fallstriktehenselisierung}
  %     \item Def. Normale Komposition \ref{def:normalekomposition}
  %     \item Für $R$ strikter henselscher, diskreter BWR
  %       definiert $E_\K$ eine normale Komposition auf $E$.
  %     \item (Theorem von Weil) Ein glattes $R$-Schema mit normaler
  %       Komposition ist birational zu einem $R$-Gruppenschema endlichen
  %       Typs: \ref{thm:weil}
  %     \item Also ist $E$ birational zu einem Gruppenschema $G$.
  %       Die rationalen Abbildungen sind Morphismen.
  %       \ref{thm:egruppenschema}
  %     \end{enumerate}
  %   \item Fall diskreter BWR:
  %     \begin{enumerate}
  %     \item Minimale eigentliche, reguläre Modelle sind stabil unter
  %       flachem, unverzweigtem Basiswechsel.
  %     \item $E^\text{sh}=E\times_R\Rsh$ ist glatter Teil von
  %       $C^\text{sh}$, also Néron-Modell von $E_\K^\text{sh}$.
  %     \item $E$ ist Néron-Modell wegen „faithfully flat descent“
  %     \end{enumerate}
  %   \item Allg. Dedekindring: lokal (\ref{thm:neronmodelllokal})
  %   \end{enumerate}
  % \end{proof}

  
\section{Gruppenstruktur hinreichend}\label{chap:gruppenstrukturhinreichend}
% ÄQ GRUPPENSTRUKTURERW.
% strikt henselsch nicht nötig!!
Die Hauptaussage dieses Kapitels wird sein, dass $E$ genau dann
Néron-Modell von $E_\K$ ist, wenn sich die Gruppenverknüpfung von
$E_\K$ auf ganz $E$ erweitert und $E$ damit zum $R$-Gruppenschema
macht.
\begin{Lemma}\label{thm:schnittearithmflglatt}
  Sei $R$ diskreter Bewertungsring mit perfektem Restklassenkörper
  $k$, $\K$, $E_\K$, $C$, $E$ wie in \ref{thm:exneronmodelle}.
  Sei $(R',\m')$ der Halm eines glatten $R$-Schemas und diskreter
  Bewertungsring
  und sei ein $R$-Morphismus $\phi\colon \Spec(R')\to C$ gegeben.
  Ist $\phi(\Spec(\K))\in E$, gilt $\im(\phi)\subset E$.
  \begin{proof}
    Für den generischen Punkt $\phi((0))$ ist dies klar, da die
    generische Faser von $C$ glatt ist und nach
    \cite[8.5, Proposition 17]{bosch} wegen Flachheit von $C$ in $E$
    liegt. 
    Betrachte also $\phi(\m')$ für den abgeschlossenen Punkt
    $\m'\in\Spec(R')$.
    $C_{R'}\coloneqq C\times_R R'$ ist nach
    \ref{thm:eigminmodell} wieder eine arithmetische Fläche über
    $R'$ mit Unterschema der glatten Punkte
    $E_{R'}\coloneqq E\times_R R'$.
    Nun ist der kanonische Morphismus
    $(\phi\times_{R}\Id_{R'})\colon\Spec(R')\to C_{R'}$ ein 
    $R'$-Schnitt, weshalb die spezielle Faser $(C_{R'})_{\m'}$ nach
    \ref{thm:arithflschnittbilderglatt} regulär in
    $(\phi\times\Id)(\m')$ ist.
    Als Schema endlichen Typs über einem
    perfekten Körper ist $(C_{R'})_{\m'}$ sogar glatt in diesem
    regulären Punkt.
    Da $C_{R'}$ als flaches, endlich präsentiertes $R'$-Schema wieder
    genau dann glatt in einem Punkt ist, wenn es die zugehörige Faser
    ist, ist $C_{R'}$ glatt in $(\phi\times\Id)(\m')$ bzw.
    $(\phi\times\Id)(\m')\in E\times_R R'$. Folglich muss bereits
    $\phi(\m')$ in $E$ liegen und $\phi\colon \Spec(R')\to E\subset C$
    faktorisiert über $E$.
  \end{proof}
\end{Lemma}  

\begin{Satz}\label{thm:fallstriktehenselisierung}
  Sei $R$ hier %strikt henselscher,
  diskreter Bewertungsring mit
  perfektem Restklassenkörper $k$
  und $\K$, $E_\K$, $C$, $E$ wie in \ref{thm:exneronmodelle}.
  Erweitert sich die $\K$-Gruppenschemastruktur von $E_\K$ auf
  $E=C^0$, dann ist $E$ Néron-Modell von $E_\K$.
  \begin{proof}
    Da $E$ bereits nach Voraussetzung glattes $R$-Gruppenschema ist,
    ist noch die \NAbbEig zu prüfen.
    Sei also $Y$ ein glattes $R$-Schema mit einem $\K$-Morphismus
    $\phi_\K\colon Y_\K\to E_\K$ der generischen Fasern.
    % , bzw. nach \ref{thm:dichtefaser}
    % einer rationalen Abbildung $\phi\colon Y\dra E$, die von $\phi_\K$
    % präsentiert wird.
    Nach \ref{thm:erwrrational} erweitert sich $\phi_\K$ auf eine
    $R$-rationale Abbildung $\phi\colon Y\dra C$, deren Bild nach dem
    vorhergehenden Lemma~\ref{thm:schnittearithmflglatt} in $E$
    liegt.
    
    Wir werden den Beweis durch Widerspruch führen.
    Angenommen, $\phi\colon Y\dra E$ sei kein Morphismus.
    Da $E$ ein $R$-Gruppenschema, $Y$ glatt und $\phi$ eine
    $R$-rationale Abbildung ist, hat $Y\setminus\Dom(\phi)$
    Kodimension~1 nach \ref{thm:rationalzumorphismus}.
    Das heißt es gibt ein irreduzibles, abgeschlossenes Unterschema
    $Z\subset Y$ der Kodimension~1, auf dessen generischem Punkt
    $\eta_Z$ die $R$-rationale Abbildung $\phi$ nicht definiert ist.
    % Dies ist äquivalent dazu, dass $\phi$ nicht im generischen Punkt
    % $\eta_Z$ von $Z$ definiert ist, da jede offene Teilmenge $U$ von
    % $Z$ Umgebung von $\eta_Z$ ist.
    $\O_{Y,\eta_Z}$ ist diskreter Bewertungsring, da er von
    Dimension~1, regulär ($Y$ ist glatt, also
    regulär) und lokal ist.
    Sein Quotientenkörper ist der Quotienten-~bzw. Restklassenkörper
    eines generischen Punkts $\eta$ der irreduziblen Komponente von
    $X$, in der $\eta_Z$ liegt. Wir erhalten somit ein kommutatives
    Diagramm
    \begin{center}
      \begin{tikzcd}
        \eta\arrow[r, equal]
        &\Spec(\k(\eta))
        \arrow[r, "{\phi_\K|_{\eta}}"] \arrow[d]
        & E_\K \arrow[r, hook]
        & C\arrow[d]\\
        \eta\cup\eta_Z\arrow[r, equal]
        &\Spec(\O_{Y,\eta_Z}) \arrow[rr]
        \arrow[urr, dashed, "{\exists!}", out=20, in=215]
        && \Spec(R)
      \end{tikzcd}
    \end{center}
    $C$ ist nach Voraussetzung eigentlich, daher erweitert sich
    $\phi_\K|_{\eta}$ nach dem Bewertungskriterium für
    Eigentlichkeit auf einen Morphismus
    $\phi|_{\O_{Y,\eta_Z}}\colon\Spec(\O_{Y,\eta_Z})\to C$.
    \ref{thm:morphismuserweiterung} besagt, dass sich
    dieser auf eine offene Umgebung $U_Z$ von $\eta_Z$ erweitert,
    was bedeutet, dass $\phi|_{U_Z}\colon U_Z\dra E\hookrightarrow C$
    Morphismus ist.
    Nun ist aber $\eta\in E_\K$, also
    $\phi|_{\O_{Y,\eta_Z}}(\eta)\in E$, weshalb nach
    \ref{thm:schnittearithmflglatt}
    $\im(\phi|_{\O_{Y,\eta_Z}})\subset E$.
    Wir können demnach eine glatte, offene Umgebung von
    $\im(\phi|_{\O_{Y,\eta_Z}})$ wählen, deren Urbild $U$ unter
    $\phi|_{U_Z}$ entsprechend wieder offen in $U_Z$, also $Y$, ist.
    Insgesamt ist $\phi|_U\colon U\to E$ ein Morphismus, was
    bedeutet, dass $\phi$ entgegen der Annahme in $\eta_Z$ definiert
    ist.
    Damit muss $\phi\colon Y\to E$ Morphismus sein, sprich die
    \NAbbEig für $E$ ist erfüllt.
    
    % \item Also kann $\phi(\eta_Z)$ nicht in $E$ liegen. Ansonsten
    %   wäre $\phi\colon Y\dra E$ auf dem Urbild einer genügend kleinen
    %   Umgebung von $\eta_Z$ in $E$ Morphismus und entgegen der Annahme in
    %   $\eta_Z$ definiert.
    %   %   C integer, urbilder dichter Mengen dicht
    % \item[?] $\phi(\eta_Z)\not\in E$ ist äquivalent dazu, dass für
    %   jeden geschlossenen (also $k$-wertigen) Punkt $x\in Z\setminus
    %   X_\K$, in dem $\phi$ definiert ist, $\phi(x)\not\in E$ gilt.
    % \item Da $R$ strikt henselsch ist, hat die Einschränkung $Y(R)\to
    %   Y_k(k)$ der $R$-wertigen Punkte auf die spezielle Faser ein
    %   dichtes Bild in $Y_k$ nach \ref{thm:eigstrikthenselsch}.
    % \item[?] Es gibt daher einen $R$-wertigen Punkt $x\in Y(R)$,
    %   dessen Einschränkung $x_0\in Y_k(k)$ einem geschlossen Punkt in
    %   $Z$ entspricht und in dem $\phi\colon Y\to C$ definiert ist.
    % \item Wir erhalten das kommutative Diagramm
    %   \begin{center}
    %     \begin{tikzcd}
    %       \Spec(\K) \arrow[r, "{x|_{\Spec(\K)}}"] \arrow[d]
    %       & Y_\K \arrow[r, "{\phi_\K}"]
    %       & E_\K \arrow[r, hook]
    %       & C \arrow[d]\\
    %       \Spec(R) \arrow[rrr, "\Id"{swap}]
    %       \arrow[urrr, dashed, "{\exists!}", in=210, out=15]
    %       &&&\Spec(R)
    %     \end{tikzcd}
    %   \end{center}
    %   und nach dem Bewertungskriterium für die Eigentlichkeit von $C$
    %   erweitert sich $\phi\circ x|_{\Spec(\K)}$ zu einem Morphismus
    %   $\Spec(R)\to C$, d.\,h.
    %   \begin{gather*}
    %     C(R)\ni\phi\circ x\colon
    %     \Spec(R)
    %     \overset{x}{\longto}
    %     \left\{ x|_{\Spec(\K)}, x_0 \right\}
    %     \overset{\phi}{\longto} C
    %   \end{gather*}
    %   ist Morphismus.
    % \item Es muss aber gelten, dass $\phi\circ x\not\in E(R)$, da
    %   ansonsten entgegen unserer Annahme $\phi$ in $x_0$ definiert
    %   wäre.
    % \item Dies ist ein Widerspruch, da nach
    %   \ref{thm:ratpkteregulaeremodelle} die Gleichheit
    %   $x\not\in E(R)=C(R)\ni x$ gilt.
    % \end{enumerate}
  \end{proof}
\end{Satz}

\section[Gruppenstruktur für strikt henselsche Ringe]%
{Nachweis der Gruppenstruktur für strikt henselsche Bewertungsringe}
\label{chap:gruppestrikthenselsch}
Für den Abschluss des Beweises zu \ref{thm:exneronmodelle} müssen wir
nach \ref{thm:fallstriktehenselisierung} nur noch nachweisen, dass
sich die Gruppenstruktur auf $E$ erweitert.
Dies werden wir in diesem Abschnitt vorerst für strikt henselsche
Bewertungsringe zeigen. Vorweg werden die Definition und Eigenschaften
strikt henselscher Ringe aufgeführt.
% Nur Erweiterung der Gruppenstruktur!
\subsection{Strikt henselsche Ringe}
% HENSELSCHE RINGE
\begin{Definition}[henselscher Ring]\label{def:henselscheringe}
  \optcite[2.3, Proposition 4]{neron}
  \optcite[Chapter IV.6]{silverman2}
  Ein lokaler Ring $(R,\m)$ heißt henselsch, wenn
  er eine der folgenden Bedingungen erfüllt
  \begin{enumerate}[label=(\roman*)]
  \item henselsches Lemma:
    Für jedes normierte Polynom $f\in R[X]$ und jede einfache Nullstelle
    $\bar a\in R/\m$ seiner Restklasse $\bar f\in (R/\m)[X]$
    gibt es eine eindeutige Nullstelle $a\in R$ von $f$, die ein Lift
    von $\bar a$ ist.
  \item Jede endliche $R$-Algebra ist ein Produkt lokaler Ringe.
  \end{enumerate}
  Ein henselscher Ring heißt strikt henselsch, falls sein
  Restklassenkörper $R/\m$ separabel abgeschlossen ist.
  Im Fall eines perfekten Restklassenkörpers ist dies äquivalent zu
  algebraisch abgeschlossen.
\end{Definition}

% \begin{Definition}[Henselisierung]\label{def:henselisierung}
%   \optcite[2.3, Definition 6]{neron}
%   Die Henselisierung eines lokalen Rings $R$ ist ein henselscher
%   lokaler Ring $\Rh$ mit einem lokalen Homomorphismus
%   $i^\text{h}\colon R\to\Rh$, s.\,d. folgende universelle Eigenschaft
%   erfüllt ist:
  
%   Für jeden weiteren lokalen Homomorphismus $u\colon R\to A$ in einen
%   henselschen lokalen Ring $A$ gibt es einen eindeutigen lokalen
%   Homomorphismus $u^\text{h}\colon\Rh\to A$, so dass das folgende Diagramm
%   kommutiert
%   \begin{center}
%     \begin{tikzcd}
%       R \arrow[r, "{u}"]\arrow[d, "{i^\text{h}}"]
%       & A \\
%       \Rh\arrow[ur, dashed, "{\exists!u^\text{h}}"{swap}]
%     \end{tikzcd}
%   \end{center}

%   Die Henselisierung ist nach Definition eindeutig bis auf
%   Isomorphismus und hat denselben Restklassenkörper wie $R$.
%   \optcite[Chapter 2.3, S. 47]{neron}
% \end{Definition}

\begin{Definition}[strikte Henselisierung]\label{def:striktehenselisierung}
  Die strikte Henselisierung eines lokalen Rings $(R,\m)$ mit
  Restklassenkörper $k=R/\m$ ist ein strikt henselscher lokaler Ring
  $\Rsh$, dessen Restklassenkörper der separable Abschluss $k_s$ von
  $k$ ist, zusammen mit einem lokalen Ringhomomorphismus
  $i^\text{sh}\colon R\to\Rsh$, so dass folgende universelle
  Eigenschaft erfüllt ist
  \begin{quote}
    Für jeden weiteren lokalen Ringhomomorphismus $u\colon R\to A$ in
    einen strikt henselschen lokalen Ring $A$ mit Restklassenkörper
    $k_A$ und für jeden $k$"=Körperhomomorphismus $\alpha\colon k_s\to k_A$
    gibt es einen eindeutigen lokalen Ringhomomorphismus
    $u^\text{sh}\colon\Rsh\to A$, so dass das folgende Diagramm
    kommutiert und so dass $\alpha$ die Projektion von $u^\text{sh}$
    auf die Restklassenkörper ist
    \begin{center}
      \begin{tikzcd}
        R \arrow[r, "{u}"]\arrow[d, "{i^\text{sh}}"]
        & A \\
        \Rsh\arrow[ur, dashed, "{\exists!u^\text{sh}}"{swap}]
      \end{tikzcd}
    \end{center}
  \end{quote}
  Die strikte Henselisierung ist nach Definition eindeutig bis auf
  Isomorphismus und hat $k_s$ als Restklassenkörper.
\end{Definition}

\begin{Satz}[{\cite[Proposition IV.6.5]{silverman2}}]
  \label{thm:exhenselisierung}
  Jeder diskrete Bewertungsring $(R,\m)$ hat eine
  % Henselisierung $\Rh$ und eine
  strikte Henselisierung $\Rsh$.
  \begin{proof}
    Siehe
    \cite[Proposition IV.6.5]{silverman2},
    \cite[Remark IV.6.6.2]{silverman2}
    oder \cite[Chapter 2.3, S. 48]{neron}.
  \end{proof}
\end{Satz}

Die für uns besonders wichtige Eigenschaft eines strikt henselschen
Rings ist folgende Aussage über die Einschränkung von Schnitten:
\begin{Satz}[{\cite[Proposition IV.6.4]{silverman2}}]
  \label{thm:eigstrikthenselsch}
  Sei $(R,\m)$ ein diskreter Bewertungsring mit perfektem
  Restklassenkörper $k=R/\m$ und $X$ glattes $R$-Schema mit spezieller
  Faser $X_k$.
  Die Einschränkung
  \begin{gather*}
    X(R)\longto X_k(k)\hookrightarrow X_k
  \end{gather*}
  ist
  \begin{enumerate}[label=(\roman*)]
  \item surjektiv auf $X_k(k)$, falls $R$ henselsch ist.
  \item surjektiv mit dichtem Bild in der speziellen Faser $X_k$,
    falls $R$ strikt henselsch ist.
  \end{enumerate}
\end{Satz}

\subsection{Nachweis der Gruppenstruktur für strikt henselsche Ringe}
Um die Erweiterung der Gruppenverknüpfung auf ganz $E$ für strikt
henselsche Ringe nachzuweisen, werden wir erst zeigen, dass wir eine
normale Komposition auf $E$ erhalten.
Diese ist dann nach \ref{thm:weil} $R$-birational zu dem gesuchten
Gruppenschema und wir werden zeigen, dass diese Birationalität ein
Isomorphismus ist.
% ERWEITERUNG DER GRUPPENSTRUKTUR
\begin{Satz}\label{thm:egruppenschema}
  Sei $R$ strikt henselscher diskreter Bewertungsring mit perfektem
  Restklassenkörper $k$ und $\K$, $E_\K$, $C$, $E$ wieder wie in
  \ref{thm:exneronmodelle}.
  Dann erweitert sich die Gruppenstruktur von $E_\K$ auf $E$ und macht
  $E$ zum Gruppenschema.
  \begin{proof}[Beweis zu \ref{thm:egruppenschema}]
    Die Aussage folgt direkt aus den folgenden beiden Lemmata.
  \end{proof}
\end{Satz}
\begin{Lemma}
  Die Gruppenstruktur $\mu_\K$ auf $E_\K$ erweitert sich zu einer
  normalen Komposition $\mu$ auf $E$.
  % (Hier muss $R$ nicht notwendigerweise strikt henselsch sein.)
  \begin{proof}
    $E\times_R E$ ist glattes $R$-Schema und $C$ ist eigentlich,
    weshalb sich die Gruppenverknüpfung $\mu_\K\colon E_\K\times_\K
    E_\K\to E_\K$ auf $E_\K$ nach \ref{thm:erwrrational} eindeutig zu
    einer $R$-rationalen Abbildung $\tilde\mu\colon E\times_R E\dra C$
    erweitert.
    Die generischen Punkte $\eta_k$ der speziellen Fasern von
    $E\times_R E$ haben wegen \autoref{eq:stalkdimension}
    Dimension~1, sind also diskrete Bewertungsringe, da
    $E\times_R E$ glatt und somit regulär ist. Nach
    \ref{thm:schnittearithmflglatt} liegt dann
    $\tilde\mu(\eta_k)$ und entsprechend das Bild einer offenen
    Umgebung von $\eta_k$ unter $\tilde\mu$ in $E$. Daher erhalten
    wir (unter Umständen durch Einschränkung von $\Dom(\tilde\mu)$)
    eine $R$-rationale Abbildung $\mu\colon E\times_R E\dra E$.

    Für eine normale Komposition bleiben Assoziativität und Inverse zu
    zeigen. Seien hierfür $\phi,\psi\colon E\times_R E\to E\times_R E$
    wie in der Definition \ref{def:normalekomposition}.

    \paragraph{Assoziativität}
    Für Assoziativität müssen für alle $x, z\in E$, für die es
    wohldefiniert ist, die folgenden $R$-rationalen 
    Abbildungen gleich sein
    \begin{gather*}
      \mu(\mu(x,\bullet),z), \mu(x,\mu(\bullet,z))
      \colon E \dra E
    \end{gather*}
    Da $E\times_R E$ reduziert ist, sind aber beide die nach
    \ref{thm:erweindeutig} eindeutige Erweiterung von
    \begin{gather*}
      \mu_\K(\mu_\K(x,\bullet),z)=\mu_\K(x,\mu_\K(\bullet,z))
      \colon E_\K \dra E_\K
    \end{gather*}
    Somit sind sie gleich als $R$-rationale Abbildungen.
    
    \paragraph{Inverses $R$-rational}
    Für $R$-Rationalität von $\phi$ und $\psi$ muss noch gezeigt
    werden, dass $\Dom(\phi)$ bzw. $\Dom(\psi)$ dicht in der
    speziellen Faser von $E$ ist, also in jedem generischen Punkt
    der speziellen Faser definiert ist.
    Wir werden den Fall für $\phi$ behandeln, $\psi$ folgt analog.
    
    Sei also $\eta\in E\times_R (R/\p)=E_k$ ein generischer Punkt der
    speziellen Faser von $E$ und $R'=\O_{E,\eta}$ der Halm in
    $\eta$ mit $\K'\coloneqq\Quot(R)$. % und $k'$ Restklassenkörper.
    Wegen \ref{eq:stalkdimension} hat $R'$
    Dimension 1, ist demnach ein diskreter Bewertungsring und
    $C_{R'}\coloneqq C\times_R R'$ nach \ref{thm:eigminmodell}
    minimales $R'$-Modell der elliptischen Kurve
    $E_\K\times_{\K'}\K'\eqqcolon E_{\K'}$ mit offenem Unterschema
    $E_{R'}=E\times_R R'$ der glatten Punkte.
    Wir wissen, dass per Konstruktion $\Spec(R')\subset E$, also dass
    es einen $R'$-wertigen Punkt $s\colon\Spec(R')\to E_{R'}$ gibt.
    Wir erhalten einen zugehörigen Translationsmorphismus auf der
    generischen Faser, der aufgrund der Gruppenstruktur Automorphismus
    ist:
    \begin{gather*}
      \tau_{\K'}\colon
      E_{\K'}\cong E_{\K'}\times_{\K'}\K'
      \xrightarrow{\Id\times s}
      E_{\K'}\times_{\K'}E_{\K'}
      \xrightarrow{\mu_{\K'}}
      E_{\K'}
    \end{gather*}
    Nach \ref{thm:eigminmodell} erweitert sich $\tau_{\K'}$ zu
    einem $R'$-Automorphismus $\tau_{R'}$ auf dem minimalen Modell
    $C_{R'}$ und auf dem glatten Unterschema $E_{R'}$.
    Für ein $e\in\Spec(R')$, $x=s(e)$ und einen Punkt in der Faser
    $y\in(E_{R'})_e$ sieht $\tau_{R'}$ wie folgt aus
    \begin{center}
      \begin{tikzcd}[row sep=0pt]
        R'\times_R E \arrow[r, equal]
        & R'\times_{R'}(R'\times_R E) \arrow[r, "{s\times\Id}"]
        & (R'\times_R E)\times_{R'}(R'\times_R E)
        \arrow[r, "{\mu_{R'}}", dashed]
        & R'\times_R E \\
        (e,y)\arrow[r, mapsto]
        & (e,(e,y))\arrow[r, mapsto]
        & ((e,x),(e,y))\arrow[r, mapsto]
        & (e,\mu(x,y))
      \end{tikzcd}
    \end{center}
    bzw. $\tilde\tau_{R'}\colon(e,y)\mapsto(e,\mu(y,x))$, wenn man
    $R'\times_{R'}(R'\times_R E)$ durch $(R'\times_R E)\times_{R'}R'$ ersetzt.
    Wir erhalten das kommutative Diagramm
    \begin{center}
      \begin{tikzcd}
        (e,y) \arrow[d, mapsto]\arrow[rrr, bend left=20, mapsto]
        & R'\times_R E \arrow[r, "\tau_{R'}", "\sim"{swap}]
        \arrow[d, "s\times\Id", hook]
        & R'\times_R E\arrow[d, "s\times\Id", hook]
        & (e,\mu(x,y))\arrow[d, mapsto]\\
        (x,y)\arrow[rrr, bend right=20, mapsto]
        & E\times_R E\arrow[r, "\phi", dashed]
        & E\times_R E
        & (x,\mu(x,y))
      \end{tikzcd}
    \end{center}
    und sehen, dass $\phi$ auf dem generischen Punkt
    $(\eta,\eta)\in\Spec(R')\times_R\Spec(R')\subset E\times_R E$
    von $(E\times_R E)_k=E_k\times_k E_k$ definiert ist.
    Mit $\tilde\tau$ statt $\tau$ erhält man dasselbe für $\psi$.
    \paragraph{Inverses $R$-birational}
    Für die Birationalität von $\phi$ bzw. $\psi$ siehe
    \cite[Propsition IV.6.10]{silverman2}.
    % TODO!! Warum birational?
  \end{proof}
\end{Lemma}

\begin{Lemma}\label{thm:äqgruppenschema}
  Sei nach \ref{thm:weil} $G$ das zu $E$ $R$-birationale
  Gruppenschema. Dann ist $G$ isomorph zu $E$.
  \begin{proof}
    Es seien $f\colon G\dra E$ und $g\colon G\dra E$ die
    zueinander inversen $R$-birationalen Abbildungen.
    Es ist zu zeigen, dass beide Morphismen sind.
    \begin{description}
    \item[$g$ Morphismus] Angenommen, $g$ ist kein
      Morphismus. Dann besagt \ref{thm:weil}, dass es eine
      irreduzible, abgeschlossene Teilmenge $Z\subset E$ gibt, so
      dass $g$ nicht in deren generischen Punkt $\eta_Z$ definiert
      ist. Die spezielle Faser $E_k$ von $E$ ist eindimensional und
      abgeschlossen, weshalb $\eta_Z\in E_k$ bedeuten würde, dass
      $\eta_Z$ ein generischer Punkt von $E_k$ ist.
      Nachdem $g$ als $R$-rationale Abbildung auf auf einer
      dichten Teilmenge von $E_k$, insbesondere auf dessen
      generischen Punkten, definiert ist, kann $\eta_Z$ also nicht in
      $E_k$ liegen.
      Allerdings ist $g$ auf der gesamten generischen Faser und daher
      dann auch in $\eta_Z$ definiert, was ein Widerspruch ist.
    \item[$f$ Morphismus] Eine Translation $\tau_P^G\colon G\to G$
      auf $G$ um einen Schnitt $P$ liefert eine Translation
      $\tau_P\coloneqq\tau_P^G|_{E_\K}$, also einen
      $\K$-Automorphismus auf $E_\K$ mit Inversem
      $\tau_{(-P)}=\tau_{(-P)}^G$. Diese 
      erweitern sich nach \ref{thm:eigminmodell} zu zueinander
      inversen $R$-Automorphismen
      $\tau_P^E,\tau_{(-P)}^E\colon E\to E$ auf $E$.
      Überall, wo es definiert ist, gilt
      \begin{gather*}
        f = \tau_{(-P)}^E\circ f\circ \tau_P^G
      \end{gather*}
      da beides die eindeutigen Erweiterungen von
      $\Id_{E_\K}=f|_{E_\K}$ sind, wo
      $\tau_P=\tau_P^E|_{E_\K}=\tau_P^G|_{E_\K}$ $\K$-Automorphismus
      ist und $\Id_{E_\K}=\tau_{(-P)}\circ\tau_P$ gilt.
      
      Für jedes $\beta\in G$, so dass es einen Schnitt $P$ gibt mit
      $\tau_P^G(\beta)\in\Dom(f)$, ist $\tau_{(-P)}^E\circ f\circ
      \tau_P^G$ also eine eindeutige Erweiterung von $f|_{E_\K}$
      bzw. von $f$ auf eine Umgebung von $\beta$.
      Wir können also durch Translation den Definitionsbereich von
      $f$ für passende $\beta$ erweitern.

      Jetzt können wir nutzen, dass $R$ nach Voraussetzung strikt
      henselsch ist. Das bedeutet nach
      \ref{thm:eigstrikthenselsch}, dass die Menge der Bildpunkte
      aller Schnitte $P\colon\Spec(R)\to G$ dicht in der
      speziellen Faser liegt.
      Insbesondere findet man in jeder offenen Umgebung eines
      Punkts $\beta\in G$ einen Punkt, um den man mithilfe einer
      Translation den Defintionsbereich von $f$ auf eine Umgebung von
      $\beta$ ausweiten kann, wie oben beschrieben.
      Damit ist $g$ in allen Punkten von $G$ definiert und somit
      Morphismus.
    \end{description}
  \end{proof}
\end{Lemma}


\section{Allgemeine Existenz}\label{chap:exneronmodellebeweis}
Zu allerletzt bleibt noch zu sehen, dass wir uns von Dedekindringen
auf Bewertungsringe und von dort auf strikt henselsche Bewertungsringe
beschränken können.
\begin{proof}[Beweis von \ref{thm:exneronmodelle}]
  Wegen \ref{thm:neronmodelllokal} können wir uns auf
  diskrete Bewertungsringe $R$ beschränken.
  Betrachte also einen diskreten Bewertungsring $R$ mit perfektem
  Restklassenkörper, eine elliptische Kurve $E_\K$ über $\K=\Quot(R)$,
  ein minimales Modell $X$ von $E_\K$ über $R$ und dessen Unterschema
  der glatten Punkte $E$.
  Sei $\Rsh$ die strikte Henselisierung von $R$.
  Nach \cite[Chapter 2.4, Corollary 9]{neron} ist $\Rsh$ treuflach
  über $R$ und unverzweigt per Konstruktion, vergleiche auch
  \cite[Chapter 2.3, Proposition 11]{neron}.
  Daher ist $\Rsh$ insbesondere étale über $R$ und nach
  \ref{thm:eigminmodell} ist $C\times_R\Rsh$ ein minimales Modell von
  $E_\Ksh=E_\K\times_\K\Ksh$ über $\Rsh$ mit Unterschema
  $E\times_R\Rsh$ der glatten Punkte.
  Wir wissen also aus \ref{thm:egruppenschema}, dass $E\times_R\Rsh$
  das Néron-Modell von $E_\Ksh$ über $\Rsh$ ist.

  Für die \NAbbEig von $E$ sei $Y$ glattes $R$-Schema und
  $\phi_\K\colon Y_\K\to E_\K$ Morphismus auf den generischen Fasern.
  $Y\times_R\Rsh$ ist glattes $\Rsh$-Schema, da Glattheit stabil unter
  Basiswechsel ist. Daher erweitert sich der Basiswechsel
  $\phi_\Ksh\colon Y_\Ksh\to E_\Ksh$ von $\phi_\K$ wegen der \NAbbEig
  von $E\times_R\Rsh$ auf einen Morphismus $\phi_\Rsh\colon
  Y\times_R\Rsh\to E\times_R\Rsh$.
  Nach \cite[Chapter 6.1, Theorem 6]{neron} ist wegen Treuflachheit
  von $\pi\colon\Spec(\Rsh)\to\Spec(R)$ der Basiswechselfunktor
  \begin{gather*}
    \text{(Sch/$R$)}\longto\text{(Sch/$\Rsh$)}\;,
    \qquad
    X\longmapsto \pi^*X=X\times_R\Rsh
  \end{gather*}
  volltreu. Deshalb erhalten wir einen
  treuflachen Abstieg $\phi\colon Y\to E$ von $\phi_\Rsh$, der
  $\phi_\K$ erweitert.

  Somit ist $E$ Néron-Modell von $E_\K$.
\end{proof}

%%% Local Variables:
%%% mode: latex
%%% TeX-master: "neron_models"
%%% End:
